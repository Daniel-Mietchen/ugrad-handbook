\noindent Online: \url{http://www.cpe.virginia.edu/ugrads/}

\mysection{Introduction}

Computer Engineering is an exciting field that spans topics across
electrical engineering and computer science.  Students learn and
practice the design and analysis of computer systems, including both
hardware and software aspects and their integration. Careers in
Computer Engineering (CpE) are as wide and varied as computer systems
themselves, which range from embedded computer systems found in
consumer products or medical devices, to control systems for
automobiles, aircraft, and trains, to more wide-ranging applications
in entertainment, telecommunications, financial transactions, and
information systems.

%\begin{quotation}
%Computer Engineering gives you a great working knowledge
%  and balance in both CS \& ECE. With the freedom to choose electives
%  in either department, you are in full control of your educational
%  experience and how you wish to enhance your knowledge. -- Kevin Chang,
%  08
%\end{quotation}

\subsection{Program Objectives}

Graduates of the Computer Engineering program at the
University of Virginia utilize their academic preparation
to become successful practitioners and innovators in
computer engineering and other fields. They analyze,
design and implement creative solutions to problems
with computer hardware, software, systems and
applications. They contribute effectively as team
members, communicate clearly and interact responsibly
with colleagues, clients, employers and society.

Faculty from the Computer Science and Electrical \& Computer
Engineering departments jointly administer the CpE undergraduate
degree program at the University of Virginia.

The Computer Engineering program does not offer a minor.

%\begin{quotation}
%It's the future.  Everything is digitized and computer
%  engineering allows you to keep up with changing technology. It's a
%  complex field with many great opportunities for advancement.  -- Rob
%  Yip, '08
%\end{quotation}

\mysection{Application Process}
\label{bscpeapplicationprocess}

The application process for the Computer Engineering degree are the
exact same as with the BS Computer Science degree (section
\ref{bscsapplicationprocess}, page~\pageref{bscsapplicationprocess}),
and thus it is not repeated here.


\mysection{Curriculum} % called ``Disciplines'' in the original

The curriculum has been carefully designed to ensure that students
obtain an excellent background in both Computer Science and Electrical
Engineering, providing breadth across these disciplines as well as
depth in at least one. All Computer Engineering students work through
an extended sequence of introductory, intermediate and advanced
courses:

\begin{itemlist}
\item CS 1110 Introduction to Computer Science
\item CS 2110 Software Development Methods
\item CS 2102 Discrete Math
\item ECE 2630 Introductory Circuit Analysis
\item ECE 2660 Electronics I
\item CS 2150 Program and Data Representation
\item ECE/CS 2330 Digital Logic Design
\item ECE 3750 Signals \& Systems I
\item CS 3240 Advanced Software Development
\item CS 3430 Introduction to Embedded Computer Systems
\item CS 4414 Operating Systems
\item ECE 4435 Computer Architecture \& Design
\item ECE 4440 Embedded Systems Design
\item CS/ECE 4457 Computer Networks
\end{itemlist}

%\paragraph{Please Note:} Course numbers changed in 2009. The old 
%course numbers are shown in parentheses.

\noindent In addition to providing breadth across the two areas,
this core of the Computer Engineering program provides
depth in the following areas:

%\ifthenelse{\boolean{lettersize}}{\begin{multicols}{2}}{}

\paragraph{Circuits}
\begin{itemlist}
\item ECE 2630: Introductory Circuit Analysis
\item ECE 2660: Electronics I
\end{itemlist}

\paragraph{Software Engineering}
\begin{itemlist}
\item CS 2110: Software Development Methods
\item CS 3240: Advanced Software Development
\end{itemlist}

\paragraph{Digital Logic}
\begin{itemlist}
\item ECE/CS 2330: Digital Logic Design
\item CS 2102: Discrete Math
\end{itemlist}

\paragraph{Computer Systems}
\begin{itemlist}
\item CS 2150: Program and Data Representation
%\item CS 3330: Computer Architecture (see \S\ref{embedded})
\item CS 3430 Introduction to Embedded Computer Systems
\item CS 4414: Operating Systems
\item ECE 4435: Computer Architecture \& Design
\item ECE 4436: Embedded Systems Design
\item CS/ECE 4457: Computer Networks
\end{itemlist}

%\ifthenelse{\boolean{lettersize}}{\end{multicols}}{}

\subsection{Grade Requirement}
To complete their program of study, computer engineering majors must
achieve a C average or better in their Computer Science and Electrical
Engineering courses.


%\begin{quotation}
%I decided to major in CPE because it gave me an opportunity
%  to combine two majors into one. I came to UVA interested in computer
%  science, but decided that I wanted to know more about the hardware
%  and doing CPE was the perfect choice for me.  In the long run having
%  this major can make the student more marketable because he or she
%  can take on careers in many paths. -- Alla Aksel, '04
%\end{quotation}

\subsection{Sample BS CpE Course Schedule}

Below is the recommended course of study for the bachelor's degree. If
one has already completed some of these classes (through AP credit,
for example), then your course of study would deviate from what is
shown below~-- consult your academic advisor for details.

\vspace{0.15in}

%\ifthenelse{\boolean{lettersize}}{\begin{multicols}{2}}{}

\samplescheduletableheader
\und{First semester} & & \und{15} \\
APMA 1110 & Single Variable Calculus & 4 \\
CHEM 1610/1611 & Chemistry for Engineers \& Lab & 4 \\
%CHEM 1611 & Chemistry Lab & 1 \\
ENGR 1620 & Introduction to Engineering & 3 \\
ENGR 1621 & Intro.\ to Engineering Lab & 1 \\
STS 1500 & Engineering, Tech.\ \& Society & 3 \\
\\
\end{tabular}

\samplescheduletableheader
\und{Second semester} & & \und{17} \\
APMA 2120 & Multivariate Calculus & 4 \\
PHYS 1425/1429 & Physics I \& Lab & 4 \\
%PHYS 1429 & Physics I Workshop & 1 \\
CS 1110 & Intro to Computer Science & 3 \\
SCI & Science elective$^2$ & 3 \\
HSS & HSS elective$^1$ & 3 \\
\\
\end{tabular}

\samplescheduletableheader
\und{Third semester} & & \und{16} \\
APMA 2130 & Ordinary Differential Eq. & 4 \\
CS 2110 & Software Develop.\ Methods & 3 \\
CS 2102 & Discrete Mathematics & 3 \\
ECE 2630 & Introductory Circuit Analysis & 3 \\
HSS & HSS elective$^1$ & 3 \\
\\
\end{tabular}

\samplescheduletableheader
\und{Fourth semester} & & \und{16} \\
CS 2150 & Prog.\ \& Data Representation & 3 \\
CS/ECE 2330 & Digital Logic Design & 3 \\
ECE 2660 & Electronics I & 4 \\
UE & Unrestricted elective$^3$ & 3 \\
STS & 2xxx/3xxx elective & 3 \\
\\
\end{tabular}

\samplescheduletableheader
\und{Fifth semester} & & \und{16} \\
ECE 3430 & Intro to Embed.\ Systems (see \S\ref{embedded}, p.~\pageref{embedded})& 3 \\
ECE 3750 & Signals \& Systems & 3 \\
CS/ECE & CS/ECE elective $^4$ & 3 \\
PHYS 2415/2419 & General Physics II \& Lab & 4 \\
%PHYS 2419 & General Physics Lab I & 1 \\ 
UE & Unrestricted elective$^3$ & 3 \\
\\
\end{tabular}

\samplescheduletableheader
\und{Sixth semester} & & \und{16.5} \\
APMA 3100 & Probability & 3 \\
ECE 4435 & Computer Arch.\ \& Design & 4.5 \\
CS 3240 & Advanced Software Develop. & 3 \\
CS/ECE & CS/ECE elective $^4$ & 3 \\
HSS & HSS elective$^1$ & 3 \\
\\
\end{tabular}

\samplescheduletableheader
\und{Seventh semester} & & \und{16.5} \\
ECE 4440 & Embedded Systems Design & 4.5 \\
CS/ECE 4457 & Computer Networks & 3 \\
CS/ECE & CS/ECE elective $^4$ & 3 \\
UE & Unrestricted elective$^3$ & 3 \\
STS 4500 & Western Tech and Culture & 3 \\
\\
\end{tabular}

\samplescheduletableheader
\und{Eighth semester} & & \und{15} \\
CS 4414 & Operating Systems & 3 \\
CS/ECE & CS/ECE elective $^4$ & 3 \\
UE & Unrestricted elective$^3$ & 3 \\
UE & Unrestricted elective$^3$ & 3 \\
STS 4600 & Engineer in Society & 3 \\
\end{tabular}

%\ifthenelse{\boolean{lettersize}}{\end{multicols}}{}

\paragraph{Footnotes:}

\label{sec:cpeunrestricted}

\begin{numlist}
\item Chosen from the approved list available in A122 Thornton Hall.
\item Chosen from: among BIOL 2010, BIOL 2020, CHEM 1620, ECE 2066, ENGR
  2500, MSE 2090, and PHYS 2620.
\item \label{cpeunrestricted} Unrestricted electives may be chosen
  from any graded course in
  the University except mathematics courses below MATH 1310 including
  STAT 1100 and 1120, and courses that substantially duplicate any
  others offered for the degree, including PHYS 2010, 2020; CS 1010,
  1020; or any introductory programming course. Students in doubt as
  to what is acceptable to satisfy a degree requirement should get the
  approval of their advisor and the dean's office, located in Thornton
  Hall, Room A122. APMA 1090 counts as a three-credit unrestricted
  elective.
\item Chosen from CS/ECE course at the 3000 level or higher. Two
  CS/ ECE electives must be 4000 level or above.
\end{numlist}

\subsection{Embedded Systems Course Requirement}
\label{embedded}

In the spring semester of 2013, the curriculum was modified to require
ECE 3430 (Introduction to Embedded Computing Systems) instead of CS
3330 (Computer Architecture); CS 3330 is no longer required.  In
addition, ECE 4435 (Computer Architecture \& Design) was restructured
to include many of the topics in CS 3330, and thus both courses cannot
be taken for credit.  
% Note that this change also affects the Electrical Engineering
% curriculum, but that is not being covered here.

Any CS courses that require CS 3330 as a prerequisite, ECE 4435 is
also allowed.  The one exception is CS/ECE 4457 (Computer Networks),
which allows either CS 3330 or ECE 3430 as prerequisites (i.e., not
CS 4435).

%Questions to resolve by asking Joanne:
%\begin{itemlist}
%\item what are the name changes?  they are listed differently in
%  different places
%\item is that prerequisite structure really correct? what about
%  cross-listed courses?  what about cs-only courses?
%\item what other changes to 4435 and 4440 are there?  name changes
%  also?
%\item how does the when-they-take-what table change?  i can replace
%  3330 with 3430, but does it move up a semester?  what about 4435 and
%  4440?
%\item where does ece 3430 go a few pages back? (page 26) i have it in
%  the general category
%\end{itemlist}

\mysection{Miscellaneous Information}

There are three CS capstone courses: CS 4970 (Capstone Practicum I), 
CS 4971 (Capstone Practicum II), and CS 4980 (Capstone Research).
Only the first one (CS 4970) counts as a CS/ECE elective for the
Computer Engineering degree; the other two can only count as an
unrestricted elective.

Please refer to the Undergraduate
Record\myurl{http://records.ureg.virginia.edu/} for detailed
information about SEAS Academic Rules and Regulations including HSS
electives. In addition, guidelines such as course load, academic
probation and academic suspension can also be found in the Record.

%The Registrar web site provides a Course Renumbering Crosswalk to
%assist with the transition from 3 to 4 digit course
%numbers\myurl{http://www.virginia.edu/registrar/search.php}.

\subsection{Double BS CS \& BS CpE majors}
\label{bscscpedoublemajors}

Due to substantial overlap with CS 3330 (Computer Architecture), ECE
4435 (Computer Architecture \& Design) can NOT count as a CS elective.
However, double majors may have ECE 4435 count as their CS 3330
requirement, although this will require a manual SIS exception to do
so; see section~\ref{sec:sisexceptions}
(page~\pageref{sec:sisexceptions}) for the SIS exception process.

ECE 4440 (Embedded Systems Design) can count as a CS elective, but
this also requires a SIS exception to be entered~-- see
section~\ref{sec:sisexceptions} (page~\pageref{sec:sisexceptions})
and section~\ref{sec:sisece4435issue} (page~\pageref{sec:sisece4435issue})
for details.  Note that even though ECE 4440 is a 4.5 credit course,
it can only count as one CS elective (i.e., only 3 credits).

The BS CpE web site\myurl{http://www.cpe.virginia.edu/ugrads/} has a
sample course schedule for double majors.

%\clearpage
%\mysection{Course Requirements Flowchart}
%\begin{figure}[h!]
%\epsfig{figure=flowcharts/cpe-flowchart.png,width=4.5in}
%\end{figure}

\ifthenelse{\boolean{lettersize}}{}{\clearpage}

\begin{figure*}[h!]
\mysection{Course Requirements Flowchart}
{\em (Updated October 2013)}
\begin{center}
\ifthenelse{\boolean{useflowchartimages}}{
\ifthenelse{\boolean{lettersize}}
{\epsfig{figure=diagrams/bs-cpe.png,width=6in}}
{\epsfig{figure=diagrams/bs-cpe.png,width=4.5in}}
}{
\ifthenelse{\boolean{lettersize}}
{\epsfig{figure=diagrams/bs-cpe.pdf,width=6in}}
{\epsfig{figure=diagrams/bs-cpe.pdf,width=4.5in}}
}
\end{center}
\end{figure*}
