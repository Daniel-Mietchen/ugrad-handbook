\documentclass[10pt,letter]{book}

\usepackage{hyperref}
\usepackage{palatino}
\usepackage{epsfig}
\usepackage{fancyhdr}

\usepackage[margin=0.5in, paperwidth=5.5in, paperheight=8.5in]{geometry}
\setlength{\paperheight}{8.5in}
\setlength{\paperwidth}{5.5in}

%\usepackage[margin=0.75in, paperwidth=8.5in, paperheight=11in]{geometry}
%\setlength{\paperheight}{11in}
%\setlength{\paperwidth}{8.5in}

\setlength{\pdfpagewidth}{\paperwidth}
\setlength{\pdfpageheight}{\paperheight}

\newenvironment{itemlist}{
\begin{itemize}
\setlength{\itemsep}{0pt}
\setlength{\parskip}{0pt}}
{\end{itemize}}

\newenvironment{numlist}{
\begin{enumerate}
\setlength{\itemsep}{0pt}
\setlength{\parskip}{0pt}}
{\end{enumerate}}

\newcommand{\und}[1]{\underline{\smash{#1}}}

\newcommand{\csminoradvisor}{Mark Sherriff and Tom Horton}
\newcommand{\csminoradvisoremail}{minoradvisor@cs.virginia.edu}
\newcommand{\badup}{David Evans}
\newcommand{\badupemail}{evans@virginia.edu}
\newcommand{\baduppronoun}{his\ } % 'his' or 'her'
\newcommand{\acmadvisor}{Aaron Bloomfield}
\newcommand{\acmadvisoremail}{aaron@virginia.edu}
\newcommand{\badmp}{Westley Weimer}
\newcommand{\badmpemail}{weimer@virginia.edu}
\newcommand{\sisexceptionenterer}{Brenda Perkins}

\newcommand{\course}[7]{\noindent {\bf #1 (#2)~-- #3} (#4): #5 #6 (#7)\vspace{0.1in}}

\newcommand{\mychapter}[2]{\chapter{#1}\renewcommand{\leftmark}{\textsc{#2}}}
\newcommand{\mysection}[1]{\section{#1}\renewcommand{\rightmark}{#1}}

\newcommand{\myurl}[1]{\footnote{\scriptsize\url{#1}}}

% The next two adapted from
% http://anthony.liekens.net/index.php/LaTeX/MultipleFootnoteReferences
\newcommand{\myurlremember}[2]{\footnote{\scriptsize\url{#2}}\newcounter{#1}\setcounter{#1}{\value{footnote}}}
\newcommand{\myurlrecall}[1]{\footnotemark[\value{#1}]} 

\raggedbottom
\setlength{\footskip}{0pt}

\begin{document}
\pagestyle{empty}

\vspace*{0.5in}

\begin{figure}[h!]
\begin{center}
\epsfig{figure=images/Engineering-4-color.pdf,width=3.5in}
\end{center}
\end{figure}

\vspace{0.25in}

\begin{center}
{\huge Department {\em of} Computer Science}

{\huge Undergraduate Handbook}

\vspace{1in}

{\Large Bachelor of Science in Computer Science}

{\Large Bachelor of Arts in Computer Science}

{\Large Bachelor of Science in Computer Engineering}

{\Large Minor in Computer Science}

\vspace{1in}

{\large http://www.cs.virginia.edu}

{\large Valid for the 2011--2012 academic year}
\end{center}


\clearpage

\vspace*{1.5in}

\begin{center}
\parbox{3in}
{This undergraduate handbook was last updated in the fall of 2011.
\newline

Any version of this handbook dated during or after the summer or fall
of 2011 is valid for the 2011-2012 academic year.
\newline

Any updates, both errata and addendums, to this version of the
handbook will be listed at \url{http://www.cs.virginia.edu/bscs/}.
\newline

Any updates to degree requirements to any of the three programs will
be reflected both on the individual degree programs' websites shown
below, as well as the above URL.
\newline

The BS CS undergraduate website is at
\url{http://www.cs.virginia.edu/bscs/}; the BA CS website is at
\url{http://www.cs.virginia.edu/ba/}, and the BS CpE website is at
\url{http://www.cpe.virginia.edu/ugrads/}.
\newline

}
\end{center}

\clearpage
\pagestyle{fancy}
\setcounter{page}{1}
\pagenumbering{roman}

\tableofcontents

\cleardoublepage
\setcounter{page}{1}
\pagenumbering{arabic}

\mychapter{Introduction}{Introduction}

\mysection{Introduction}

Through the development of sophisticated computer systems, processors,
and embedded applications, computer scientists have the opportunity to
change society in ways unimagined several years ago. Our goal is the
education and training of a diverse body of students who can lead this
information technology revolution. To this end, the computing programs
orient students toward the pragmatic aspects of computing and provides
the learning and practices to make them proficient computing
professionals. Computational thinking is rooted in solid mathematics
and science, and grounding in these fundamentals is essential. Our
laboratory environment exposes students to many commercial software
tools and systems, and introduces modern software development
techniques.  In the context of the practice of computing, this early
grounding forms the basis for an education that prepares students for
a computing career.

%With funding from the National Science Foundation, the Department of
%Computer Science has designed and developed a curriculum focused on
%the practice of computing, yet grounded in the mathematical and
%scientific fundamentals of computer science. The curriculum is
%structured around the introduction of modern software development
%techniques in the very beginning courses, and is supported by a set of
%``closed laboratories''.

%In order to provide an environment appropriate to our courses, the
%department has several laboratories with hundreds of workstations.
%These machines have high-resolution graphics and are connected to
%large file handlers, as well as to the University network.  The lab
%courses expose students to many commercial software tools and systems,
%and introduce modern software development techniques via
%object-oriented design and implementation.

%The Department of Computer Science co-offers, with the Department of
%Electrical and Computer Engineering, a degree in Computer Engineering. 

Students have opportunities to participate in cutting-edge research
with department faculty members. From the senior thesis research
project to independent study, students can pursue research in any
conceivable area. Our former students are enrolled in top graduate
programs across the country. Our undergraduates have won many research
awards, including multiple CRA (Computing Research Association)
research awards in the previous academic year. In fact, of all the
institutions, UVa is third in overall CRA research awards won.

All graduates of our three computing programs will have the knowledge
and skills to be practitioners and innovators in computing and other
fields.  They will be able to apply computational thinking in the
analysis, design and implementation of computing solutions, whether
working alone or as part of a team. The knowledge and skills acquired
from our degree programs will give students the ability to make
contributions after graduation in their own field as well as to
society at large.

A recent Bureau of Labor Statistics Occupational Outlook Handbook
states that ``very favorable opportunities'' (more numerous job
openings compared to job seekers) can be expected for college
graduates with at least a bachelor’s degree in computer
engineering. It also projects an employment increase of over 38\% by
2016 for occupations available to graduates with a bachelor’s degree
in computer engineering\myurl{http://www.bls.gov/oco}.

\subsection{Diversity Statement}

The members of the department envision an environment where a
diversity of capable, inspired individuals congregate, interact and
collaborate, to learn and advance knowledge, without barriers. We
embrace this vision because:

\begin{itemlist}
\item We wish to be leaders and role models in reaping and sharing the
 benefits of diversity.
\item We seek to improve the intellectual environment and creative
 potential of our department.
\item We expect to produce happier, more capable and more broadly
 educated computer science graduates.
\item We wish to contribute to social justice and economic well being
 for all citizens.
\end{itemlist}

\mysection{Degrees Offered}

The Department of Computer Science offers three computing degrees, as well as a minor.

\begin{itemlist}
\item Bachelor of Science (BS in CS) in Computer Science, available to
  students in the School of Engineering and Applied Sciences (SEAS).
\item Bachelor of Arts in Computer Science (BA in CS), available to
  students in the College of Liberal Arts and Sciences (CLAS).
\item Bachelor of Science in Computer Engineering (BS in CpE),
  available to students in the School of Engineering and Applied
  Sciences (SEAS). This degree is shared with the Department of
  Electrical and Computer Engineering.
\item Minor in Computer Science, available to students in either SEAS
  or CLAS.
\end{itemlist}

Details of the degrees are provided later in this document, but in
this section we explain the differences between computer science and
computer engineering. This explanation is adopted from the ACM and
IEEE's Computing Curricula 2005: The Overview
Report\myurl{http://www.acm.org/education/curricula-recommendations}. We
also give a high-level overview of the difference between our BS and
BA degrees in computer science.

\subsection{What is Computer Science?}

Computer science spans a wide range, from its theoretical and
algorithmic foundations to cutting-edge developments in graphics,
intelligent systems, cybersecurity, and other exciting areas. We can
think of the work of computer scientists as falling into three
categories.

\begin{itemlist}
\item They design and implement software. Computer scientists take on
  challenging programming jobs. They also supervise other programmers,
  keeping them aware of new approaches.
\item They devise new ways to use computers. Progress in the CS areas
  of networking, database, and human-computer-interface enabled the
  development of the World Wide Web. Now CS researchers are working
  with scientists from other fields to develop control physical
  sensors and devices, to use databases to create new knowledge, and
  to use computers to help doctors solve complex problems in medical
  care.
\item They develop effective ways to solve computing problems. For
  example, computer scientists develop the best possible ways to store
  information in databases, send data over networks, and display
  complex images. Their theoretical background allows them to
  determine the best performance possible, and their study of
  algorithms helps them to develop new approaches that provide better
  performance.
\end{itemlist}

Computer science spans the range from theory through
programming. While some universities offer computing degree programs
that are more specialized (such as software engineering,
bioinformatics, etc.), a degree in computer science offers a
comprehensive foundation that permits graduates to adapt to new
technologies and new ideas.

\subsection{Comparison of the BA \& BS Computer Science
  Degrees}

At the University of Virginia, we offer two different computer science
degrees:

\begin{itemlist}
\item the Bachelor of Science (BS) degree, through the School of
  Engineering and Applied Sciences (SEAS), and
\item the Interdisciplinary Major in Computer Science, a Bachelor of
  Arts (BA) degree, through the College of Liberal Arts and Sciences
  (CLAS).
\end{itemlist}

The following gives a high-level comparison of these two degrees.

The BS in Computer Science degree program includes the set of core
courses required of every other engineering degree in SEAS. These
include an introduction to engineering, physics, chemistry, calculus,
courses focused on the engineer's role in society, and at least five
courses in the humanities or social sciences. Like other engineering
majors, all students in our BS program complete a year-long project
leading to a senior thesis in their fourth year. Students in the BS
program can minor in another engineering discipline or applied
math. It is also possible to minor in a subject from the College of
Arts and Sciences (but it's more difficult to have a second major in a
College subject). Students in the BS program must complete at least 46
credits of computer science courses. The Bachelor of Science in
Computer Science is accredited by the Computing Accreditation
Commission of ABET\myurl{http://www.abet.org}.

The BA in Computer Science degree program includes the same general
requirements (known as core and competency requirements) as all other
liberal arts and science degrees in CLAS. These include courses in
foreign language, writing, historical studies, social science,
humanities, and non-western perspectives. These general requirements
also include natural science and mathematics, but fewer courses than
are required for the BSCS in engineering. Students in the BA program
are in a good position to major or minor in another subject in
CLAS. Students with a GPA of 3.4 or better may apply to the
Distinguished Majors Program, in which students complete a thesis
based on two semesters of empirical or theoretical research. Students
in the BA program must complete at least 27 credits of computer
science courses along with 12 additional credits of "integration
electives", which are computing-related courses taught by another
department other than the CS department. Students in the BA have the
option of taking a version of the first two computing courses that
differ from those taken by the BS students, but otherwise students
from both degree programs share the same CS courses.

Graduates of both programs have been accepted to the best graduate
programs, have received job offers from leading companies, etc. A few
employers have shown a preference for graduates from one program or
the other, but in general both degrees prepare students for excellent
opportunities after graduation.

Students who apply to the University of Virginia must choose to apply
for admission to either SEAS (the engineering school) or CLAS (the
College of Liberal Arts and Sciences). It is possible to transfer from
one unit to the other after admission, and since we offer degrees in
both units a student can major in computer science in either.

\subsection{What is Computer Engineering?}

Computer engineering is concerned with the design and construction of
computers and computer-based systems. It involves the study of
hardware, software, communications, and the interaction among
them. Its curriculum focuses on the theories, principles, and
practices of traditional electrical engineering and mathematics and
applies them to the problems of designing computers and computer-based
devices.

Computer engineering students study the design of digital hardware
systems including communications systems, computers, and devices that
contain computers. They study software development, focusing on
software for digital devices and their interfaces with users and other
devices. At the University of Virginia, the CpE degree has a balanced
emphasis on hardware and software.

At the University of Virginia, computer engineering degrees are
jointly designed and administered by the Department of Computer
Science and the Department of Electrical and Computer Engineering. The
degree program is composed of courses from both departments. 

\subsection{ABET accreditation}

The Bachelor of Science in Computer Science is accredited by the
Computing Accreditation Commission of
ABET\myurlremember{abet}{http://www.abet.org}.  The Bachelor of
Science in Computer Engineering is accredited by the Engineering
Accreditation Commission of ABET\myurlrecall{abet}.


\mysection{Course Numbering}

Starting with the fall 2009 semester, the University of Virginia
changed all course numbers to 4-digit numbers from the old 3-digit
number system. Whenever possible, the course numbers in this version
of the handbook will use both the 3-digit number and the four digit
number, in the form of ``CS 1110 (101)'' to allow people to transition
from the old numbers to the new numbers.

The table below lists the mapping of the old course numbers to the new
course numbers; it is sorted by the old course numbers.

\begin{tabular}{llp{2.5in}} \hline
\bf Old & \bf New & \bf Title \\ \hline
CS 100T & CS 1000T & Non-UVa Transfer/Test Credit \\
CS 101 & CS 1110 & Introduction to Computer Science \\
CS 101E & CS 1111 & Introduction to Computer Science \\
CS 101X & CS 1112 & Introduction to Computer Science \\
CS 110 & CS 1010 & Introduction to Information Technology \\
CS 120 & CS 1020 & Introduction to Business Computing \\
CS 150 & CS 1120 & Introduction to Computing: Language, Logic, and
Machines \\
CS 200T & CS 2000T & Non-UVa Transfer/Test Credit \\
CS 201 & CS 2110 & Software Development Methods \\
CS 202 & CS 2102 & Discrete Mathematics I \\
CS 205 & CS 2220 & Engineering Software \\
CS 216 & CS 2150 & Program and Data Representation \\
CS 230 & CS 2330 & Digital Logic Design \\
CS 290 & CS 2190 & Computer Science Seminar I \\
CS 300T & CS 3000T & Non-UVa Transfer/Test Credit \\
CS 302 & CS 3102 & Theory of Computation \\
CS 305 & CS 3205 & HCI in Software Development \\
CS 333 & CS 3330 & Computer Architecture \\
CS 340 & CS 3240 & Advanced Software Development Techniques \\
CS 400T & CS 4000T & Non-UVa Transfer/Test Credit \\
CS 414 & CS 4414 & Operating Systems \\
CS 415 & CS 4610 & Programming Languages \\
CS 416 & CS 4710 & Artificial Intelligence \\
CS 425 & CS 4630 & Defense against the Dark Arts \\
CS 432 & CS 4102 & Algorithms \\
CS 433 & CS 4330 & Advanced Computer Architecture \\
CS 434 & CS 4434 & Fault-tolerant Computing \\
CS 441 & CS 4240 & Principles of Software Design \\
CS 444 & CS 4444 & Introduction to Parallel Computing \\
CS 445 & CS 4810 & Introduction to Computer Graphics \\
%CS 446 & CS 4820 & Real Time Rendering \\
%CS 447 & CS 4830 & Image Synthesis \\
%CS 448 & CS 4840 & Computer Animation \\
CS 453 & CS 4753 & Electronic Commerce Technologies \\
CS 457 & CS 4457 & Computer Networks \\
CS 458 & CS 4458 & Internet Engineering \\
CS 462 & CS 4750 & Database Systems \\
CS 471 & CS 4620 & Compilers \\
CS 493 & CS 4993 & Independent Study \\
CS 494 & CS 4501 & Special Topics in Computer Science \\
CS 495 & CS 4998 & Distinguished BA Majors Research \\ \hline
\\
\end{tabular}

The registrar provides an online course renumbering
website\myurl{http://www.virginia.edu/registrar/search.php}
to help determine the mapping between old 3-digit and new 4-digit
course numbers.

Courses that have been created since the change from 3-digit to
4-digit course numbers, such as CS 4720 (Web and Mobile Systems) and
CS 4730 (Computer Game Design), are not listed in the table.



\subsection{Course Numbering Methodology}

The new 4-digit course numbers follow a system developed by the
department. The first digit is the year that the course is expected
to be taken. The second digit specifies the type of course, as shown
below. The third and fourth digits attempted to keep the previous
last two digits of the 3-digit course number, although that was not
always possible. 


The 2nd digit numbering scheme is:

\begin{itemlist}
\item x000 – service courses, courses for non-majors, general interest
\item x100 – core, fundamentals, theoretical (a broad category)
\item x200 – SW development-oriented courses (note in ECE, this will
 be for electronics courses)
\item x300 – hardware, architecture, etc.
\item x400 – computer systems
\item x500 – by University rule: ``special-topics and variable one-time
 offerings''
\item x600 – languages, compilation, etc.
\item x700 – application areas including AI, databases, etc.
\item x800 – computer graphics
\item x900 – by University rule: thesis, dissertation, independent
 study, capstone, etc.
\end{itemlist}

Note that currently cross-listed courses with ECE fall in the x300 and
x400 categories.

%\clearpage
%\mysection{Venn Diagram}
%\begin{figure}[h!]
%\epsfig{figure=flowcharts/venn-diagram.png,width=4.5in}
%\end{figure}

\clearpage

\mysection{Major Course Requirements Comparison}

\begin{figure}[h!]
\begin{center}
\epsfig{figure=diagrams/venn-diagram.pdf,width=4in}
\end{center}
\end{figure}

\noindent
\begin{tabular}{p{2in}p{2.25in}}

The SEAS school requirements consist of:
\begin{itemlist}
\item APMA 1110 \& 2120
\item CHEM 1610 \& 1611
\item ENGR 1620
\item PHYS 1425 \& 1429
\item PHYS 2415 \& 2419
\item Science elective
\item STS 1500
\item STS 2xxx/3xxx elective
\item STS 4500 \& 4600
\end{itemlist}

&

The CLAS school requirements consist of:
\begin{itemlist}
\item First \& second writing requirements
\item Foreign language requirement
\item 6 credits of social sciences
\item 6 credits of humanities
\item 3 credits of historical studies
\item 3 credits of non-western perspectives
\item 12 credits of natural science and math
\end{itemlist}

\end{tabular}

\noindent A ``CS 1 class'' is CS 111x for SEAS majors, or CS 1120
(preferred) or CS 111x for CLAS majors; placement is avaialble (see
sections \ref{applacement} \& \ref{101placement}).

\clearpage
\mychapter{Bachelors of Science in Computer Science}{BS CS Degree}


\mysection{Introduction}

% this intro drafted by Mark, and sent via e-mail on 9-6-11

The Bachelor of Science degree in Computer Science is a wide-ranging
program, encompassing both the theoretical and the practical.  The BS
program builds upon the engineering and mathematical principles
introduced in the Engineering school's core curriculum.  Our students
are taught to apply computing to the world around them by building
faster, smaller, and more secure software systems, exploring emerging
technologies, and working on real-world problems.  Our courses focus
on teaching students how to recognize computational challenges, create
elegant and efficient algorithms, and then use rigorous development
methodologies to build systems that can solve pressing
problems. Graduates of the BS program find successful careers with
traditional software companies, government agencies, consulting firms,
academia, and companies in other fields that have software needs.
Computing professions are often ranked near the top in ``Best Job''
lists put together by news organizations for job availability, pay,
and satisfactions.

Course work in the BS CS program starts with several courses that
introduce the basic principles of software creation, from learning
programming languages to advanced development techniques.  Once
students have mastered the basics, the bulk of our program opens up,
offering electives in several exciting fields, including networking,
security, game design, web programming, e-commerce, parallel
computing, and much more.  Students have the opportunity to take
several electives each semester, as our department offers more
electives than the other departments in the Engineering school. 

\mysection{Curriculum}

\subsection{Recommended Course of Study}

Below is the recommended course of study for the bachelor's degree. If
you have already completed some of these classes (through AP credit,
for example), then your course of study would deviate from what is
shown below~-- consult your academic advisor for details.

There are a total of 8 electives that the student can choose
from. These electives are indicated by the footnotes below, and are
described in detail beginning on page~\pageref{sec:electiveinfo}. Note
that some of these requirements are for all SEAS students, while
others are required for the CS bachelor's degree. Please be aware of
when the classes are offered! Some are only offered once per year, or
in a particular semester. See page~\pageref{sec:coursedesc} for details
as to when courses are offered.

The recommended schedule shown below has changed slightly each year as
the degree requirements have evolved. As discussed in the Degree
Requirement Revisions (page~\pageref{sec:degreerevisions}), a student
can graduate using any set of requirements that were in effect when
they became a declared computer science major. Thus, as long as all
the major requirements are met, students can follow any version of the
recommended course schedule.

Academic requirements are managed by SIS (the Student Information
System, UVa’s system for handling academic requirements and
registration) A sample of the BS CS requirement listing can be found
online\myurl{http://www.cs.virginia.edu/bscs/bscs-reqs-in-sis.pdf};
your individual one can be found via SIS. You may also want to see the
FAQ question about how HSS requirements list in the SIS report (see
page~\pageref{sec:sishssissue}).


\subsection{Elective Information}
\label{sec:electiveinfo}

The numbers in the list below correspond to the footnote numbers from
the sample course schedule shown starting on
page~\pageref{sec:bscsschedule}.

\begin{enumerate}

\item Science elective (1 required): Students must choose one of BIOL
  2010 (201) (Introduction to Biology: Cell Biology and Genetics),
  BIOL 2020 (202) (Introduction to Biology: Organismal and
  Evolutionary Biology), CHEM 1620 (152) (Introductory Chemistry for
  Engineers), ECE 2066 (200) (Science of Information), ENGR 2500
  (Introduction to Nano\-science and Technology), MSE 2090 (209)
  (Introduction to the Science and Engineering of Materials), or PHYS
  2620 (252) (Introductory Physics IV: Quantum Physics). Additional
  courses in this list can count as an unrestricted elective.

\item HSS electives (5 required): Studies in the humanities and social
  sciences serve not only to meet the objectives of a broad education,
  but also to meet the objectives of the engineering profession. Such
  course work must meet the generally accepted definitions that the
  humanities are the branches of knowledge concerned with humankind
  and its culture, while the social sciences are the studies of
  society. See the full list of allowed courses in the SEAS
  Undergraduate Handbook. This list can be found
  online\myurl{http://www.seas.virginia.edu/advising/undergradhandbook.php\#hss}. Note
  that there are a number of courses that do not count as HSS
  electives, but would count as an unrestricted elective. See that URL
  for details.

\item Unrestricted elective (5 required): Any graded course in the
  University, with a few exceptions. From the SEAS Undergraduate
  Student Handbook\myurl{http://www.seas.virginia.edu/advising/undergradhandbook.php}:
  All Unrestricted Electives may be chosen from any graded course in
  the University except mathematics courses below MATH 1310 (131),
  including STAT 1100 (110) and 1120 (112), and courses that
  substantially duplicate any others offered for the degree, including
  PHYS 2010 (201), PHYS 2020 (202), CS 1010 (110), or any introductory
  programming course. Students in doubt as to what is acceptable to
  satisfy a degree requirement should obtain the approval of their
  advisor and the dean's office, Thornton Hall, Room A122. APMA 1090
  (109) counts as a three credit unrestricted elective for
  students. Band classes (such as marching band) and ROTC classes can
  count for the unrestricted elective.

\item APMA elective (2 required): Must choose two from: APMA 2130
  (213) (Ordinary Differential Equations), APMA 3080 (308) (Linear
  Algebra) or APMA 3120 (312) (Statistics). Note that APMA 3100 (310)
  (Probability) is a required course in addition to the two APMA
  electives.

\item CS electives (5 required): Any 3 credit CS class at the 3000
  level or above. A course that is fulfilling another requirement can
  not also count as a CS elective. If you take more than five CS
  electives, you can count additional CS elective course(s) as
  unrestricted electives. ECE 4435 (435) (Computer Organization \&
  Design) and ECE 4440 (436) (Advanced Digital Design) also count as a
  CS electives (this is not the case for CpE majors, as they are both
  required courses for CpE). Note, however, that those two courses
  only count as {\em one} CS elective each; those 9 credits (each is
  worth 4.5 credits) do not count as 3 CS electives.  And in order for
  them to be counted, a SIS exception must be entered~-- see
  section~\ref{sec:sisece4435issue} on
  page~\pageref{sec:sisece4435issue} for details.  CS 4993 (493)
  (Independent Study) can be used at most once for a CS elective
  (i.e. no more than 3 credits); additional CS 4993 (493) credits can
  be used as unrestricted electives. Note that for a class that does
  not meet these requirements to count as a CS elective requires
  approval by the CS undergraduate curriculum committee (NOT by the
  student's academic advisor). Due to substantial overlap, one cannot
  get credit for both ECE 4435 (435) and CS 4330 (433). Thus, if a
  student takes both of those classes, the other one can ONLY count as
  a unrestricted elective.

\item STS 2xxx/3xxx elective (1 required): Any STS course at the
  2000-level or 3000-level.
\end{enumerate}
 
Note that classes that receive no grade (including classes that are
audited) do not count toward your degree requirements.

\clearpage
\subsection{Degree Requirements Checklist}

\small 
\begin{tabular}{|l|l|l|l|} \hline
\bf Required computing \& math courses & \bf Grade & \bf Semester &
\bf Comments? \\ \hline \hline
CS 1110: Intro. to Computer Science & & & \\ \hline
CS 2110: Software Development Methods & & & \\ \hline
CS 2102: Discrete Mathematics I & & & \\ \hline
CS 2150: Program \& Data Representation & & & \\ \hline
CS/ECE 2330: Digital Logic & & & \\ \hline
CS 2190: CS Seminar I & & & \\ \hline
CS 3102: Theory of Computation & & & \\ \hline
CS/ECE 3330: Computer Architecture & & & \\ \hline
CS 3240: Advanced SW Devel. Tech. & & & \\ \hline
CS 4414: Operating Systems & & & \\ \hline
CS 4102: Analysis of Algorithms & & & \\ \hline
APMA 3100: Probability & & & \\ \hline
APMA 2130 or 3080 or 3120 (circle one) & & & \\ \hline
APMA 2130 or 3080 or 3120 (circle one) & & & \\ \hline
\end{tabular}

\noindent\begin{tabular}{@{}ll}
\noindent\begin{tabular}{@{}l}
\\
SEAS required courses \\
\begin{tabular}{|l|l|l|}\hline
\bf Course & \bf Grade & \bf Semester \\ \hline \hline
APMA 1110 & & \\ \hline
APMA 2120 & & \\ \hline
CHEM 1610 & & \\ \hline
CHEM 1611 & & \\ \hline
ENGR 1620 & & \\ \hline
PHYS 1425 & & \\ \hline
PHYS 1429 & & \\ \hline
PHYS 2415 & & \\ \hline
PHYS 2419 & & \\ \hline
\end{tabular} \\
\\
Science elective \\
\begin{tabular}{|l|l|l|} \hline
\bf Course & \bf Grade & \bf Semester \\ \hline \hline
\hspace{0.7in} & & \\ \hline
\end{tabular}
\end{tabular}

&

\begin{tabular}{@{}l}
\\
STS courses \\
\begin{tabular}{|l|l|l|} \hline
\bf Course & \bf Grade & \bf Semester \\ \hline \hline
STS 1010/1500 & & \\ \hline
STS 2xxx/3xxx & & \\ \hline
STS 4010/4500 & & \\ \hline
STS 4020/4600 & & \\ \hline
\end{tabular} \\
\\
CS Electives (5) \\
\begin{tabular}{|l|l|l|l|} \hline
& \bf Course & \bf Grade & \bf Semester \\ \hline \hline
1) & \hspace{0.55in} & & \\ \hline
2) & & & \\ \hline
3) & & & \\ \hline
4) & & & \\ \hline
5) & & & \\ \hline
\end{tabular}
\end{tabular}
\\
& \\
HSS electives (5) & Unrestricted electives (5) \\
\begin{tabular}{|l|l|l|l|} \hline
& \bf Course & \bf Grade & \bf Semester \\ \hline \hline
1) & \hspace{0.45in} & & \\ \hline
2) & & & \\ \hline
3) & & & \\ \hline
4) & & & \\ \hline
5) & & & \\ \hline
\end{tabular}
& 
\begin{tabular}{|l|l|l|l|} \hline
& \bf Course & \bf Grade & \bf Semester \\ \hline \hline
1) & \hspace{0.575in} & & \\ \hline
2) & & & \\ \hline
3) & & & \\ \hline
4) & & & \\ \hline
5) & & & \\ \hline
\end{tabular}
\\
\end{tabular}

\normalsize

\clearpage

\subsection{Sample BS CS Course Schedule}
\label{sec:bscsschedule}

\noindent \begin{tabular}{llc}
\und{First semester} & & \und{15} \\
APMA 1110 (111) & Single Variable Calculus & 4 \\
CHEM 1610 (151) & Chemistry for Engineers & 3 \\
CHEM 1611 (151L) & Chemistry Lab & 1 \\
ENGR 1620 (162) & Introduction to Engineering & 4 \\
STS 1010/1500 (101) & Engineering, Technology \& Society & 3 \\
& & \\
\und{Second semester} & & \und{17} \\
APMA 2120 (212) & Multivariate Calculus & 4 \\
PHYS 1425 (142E) & Physics I & 3 \\
PHYS 1429 (142W) & Physics I Workshop & 1 \\
CS 1110 (101) & Intro to Computer Science & 3 \\
SCI & Science elective$^1$ & 3 \\
HSS/UE & HSS or unrestricted elective$^{2,3}$ & 3 \\
& & \\
\und{Third semester} & & \und{16} \\
APMA & APMA elective$^4$ or APMA 3100 (310) & 3 \\
CS 2110 (201) & Software Development Methods & 3 \\
CS 2102 (202) & Discrete Mathematics & 3 \\
PHYS 2415 (241E) & General Physics II & 3 \\
PHYS 2419 (241W) & General Physics Lab I & 1 \\ 
HSS/UE & HSS or unrestricted elective$^{2,3}$ & 3 \\
& & \\
\und{Fourth semester} & & \und{16} \\
CS 2150 (216) & Program and Data Representation & 3 \\
CS/ECE 2330 (230) & Digital Logic Design & 3 \\
CS 3102 (302) & Theory of Computation & 3 \\
CS 2190 (290) & CS Seminar & 1 \\
STS STS & 2xx/3xx elective$^6$ & 3 \\
HSS/UE & HSS or unrestricted elective$^{2,3}$ & 3 \\
& & \\
\und{Fifth semester} & & \und{15} \\
CS/ECE 3330 (333) & Computer Architecture & 3 \\
CS 4102 (432) & Algorithms & 3 \\
APMA & APMA elective$^4$ or APMA 3100 (310) & 3 \\
HSS/UE & HSS or unrestricted elective$^{2,3}$ & 3 \\
HSS/UE & HSS or unrestricted elective$^{2,3}$ & 3 \\
\end{tabular}
 
\noindent \begin{tabular}{llc}
\und{Sixth semester} & & \und{15} \\
CS 3240 (340) & Advanced Software Development & 3 \\
CS & CS elective$^5$ & 3 \\
APMA & APMA elective$^4$ or APMA 3100 (310) & 3 \\
HSS/UE & HSS or unrestricted elective$^{2,3}$ & 3 \\
HSS/UE & HSS or unrestricted elective$^{2,3}$ & 3 \\
& & \\
\und{Seventh semester} & & \und{15} \\
STS 4010/4500 (401) & Western Tech and Culture & 3 \\
CS & CS elective$^5$ & 3 \\
CS & CS elective$^5$ & 3 \\
CS 4414 (414) & Operating Systems & 3 \\
HSS/UE & HSS or unrestricted elective$^{2,3}$ & 3 \\
& & \\
\und{Eighth semester} & & \und{15} \\
STS 4020/4600 (402) & Engineer in Society & 3 \\
CS & CS elective$^5$ & 3 \\
CS & CS elective$^5$ & 3 \\
HSS/UE & HSS or unrestricted elective$^{2,3}$ & 3 \\
HSS/UE & HSS or unrestricted elective$^{2,3}$ & 3 \\
\end{tabular}

\mysection{Miscellaneous Information}
\subsection{CS 2190 (290) Specific Details}

While students can take courses in any semester, there is an issue to
consider with CS 2190 (290): this course should be taken in the second
year or (less preferably) the third year. If a student reaches
his/her fourth year without taking the course, then s/he must take a 3
credit in ethics and technology in its place (even though CS 2190
(290) is only 1 credit). This course taken in place of CS 2190 (290)
does not count towards any other requirement except to replace CS 2190
(290).


%\clearpage
%\mysection{Course Requirements Flowchart}
%\begin{figure}[h!]
%\epsfig{figure=flowcharts/Cs-course-flowchart-summer-2010.png,width=4.5in}
%\end{figure}

\clearpage
\mysection{Course Requirements Flowchart}

\begin{figure}[h!]
\epsfig{figure=diagrams/bs-cs.pdf,width=4.5in}
\end{figure}


\clearpage
\mychapter{Bachelors of Arts in Computer Science}{BA CS Degree}

\mysection{Introduction}

Computer Science is the study of information processes. Computer
scientists learn how to describe information processes, how to reason
about and predict properties of information processes, and how to
implement information processes elegantly and efficiently in hardware
and software. The Computer Science major concentrates on developing
the deep understanding of computing and critical thinking skills that
will enable graduates to pursue a wide variety of possible fields and
to become academic, cultural, and industrial leaders in areas that
integrate the arts and sciences with computing. The Computer Science
major is designed to provide students entering the University without
previous background in computing with an opportunity to major in
Computer Science, while taking courses in arts, humanities, and
sciences to develop broad understanding of other areas and their
connections to computing. Computing connects closely with a wide range
of disciplines including, but not limited to, the visual arts, music,
life sciences including biology and cognitive science, the physical
sciences, linguistics, mathematics, and the social sciences. The core
curriculum focuses on developing methods and tools for describing,
implementing, and analyzing information processes and for managing
complexity including abstraction, specification, and recursion. 


\mysection{Curriculum}

\paragraph{Prerequisites}

Before declaring the computer science major, all students should have
taken one introductory computer science course (either CS 1120 (150),
CS 1110 (101), CS 1111 (101-E), or CS 1112 (101-X)) with a grade of C+
or better, or have comparable experience. Students may be permitted to
declare the major while they are currently taking the introductory
course.

The major requires the College Competency and Area
Requirements\myurl{http://artsandsciences.virginia.edu/college/requirements/index.html}
as well as at least 27 credits in Computer Science courses and 12
credits in Integration Electives.

\subsection{Required ``Core'' Courses}

The following courses are required for all BA CS majors.  Full
descriptions can be found in the Course Descriptions section
(page~\pageref{sec:coursedesc}).

\begin{itemlist}
\item CS 2110 (201), Software Development Methods, or CS 2220 (205),
  Engineering Software
\item CS 2102 (202), Discrete Math
\item CS 2150 (216), Program and Data Representation
\item CS 3330 (333), Computer Architecture
\item CS 4102 (432), Algorithms
\end{itemlist}

Note that the CS1 class, either CS 111x (101/101E/101X), Introduction
to Programming, or CS 1120 (150), Introduction to Computing: Language,
Logic, and Machines, is required to enroll in CS 2110 (201) or CS
2220 (205), respectively.

\subsection{CS Electives}

Four computing-intensive electives are to be selected from a list of
approved courses. The list of approved courses will initially comprise
current Computer Science courses at 3000-level or above as well as CS
2330. Additional courses that may be jointly offered by CLAS and CS
departments will be added to the list of approved computing electives
based on approval by the BA committee.


\subsection{Integration Electives}

Four courses selected with the approval of the student's advisor from
the list of computing-related courses approved by the BA CS
committee. These courses are offered by departments other than
Computer Science, and should either provide fundamental computing
depth and background or explore applications of computing to arts and
sciences fields. 

This is a list of the courses that are generally approved as
integration electives. This list is not meant to be exhaustive: if you
find a course that is not on the list that appears to satisfy the
goals of an integration elective, discuss with your advisor or the BA
Program Director if it should count as an integration elective for
you.

Some of these courses are not offered regularly, and some courses may
have prerequisites. Courses listed in {\bf bold} are courses that are
offered regularly and are among the most commonly taken integration
electives.

See online\myurl{http://www.cs.virginia.edu/ba/integration.html}
for an up-to-date list of approved integration electives. 

\paragraph{Arts}
\begin{itemlist}
\item ARCH 5420 (Digital Animation \& Storytelling)
\item ARCH 5710 (Photography and Digital Media)
\item DRAM 2620/2630 (Sound Design and Sound Lab)
\item DRAM 2110 (Lighting Design)
\item DRAM 3210 (Scene Design)
\item {\bf MDST 2010 (Introduction to Digital Media)}
\item MDST 3050 (History of Media)
\item MDST 3703 (Introduction to the Digital Liberal Arts)
\item MUSI 2350 (Technosonics and Digital Music)
\item MUSI 3390 (Introduction to Music and Computers)
\item MUSI 7350 (Interactive Media)
\item MUSI 4543 (Sound Studio)
\end{itemlist}

\paragraph{Mathematics and Logic}
\begin{itemlist}
\item {\bf ECE 2066 (Science of Information)}
\item MATH 1160 (Algebra, Number Systems, and Number Theory)
\item MATH 3000 (Transition to Higher Mathematics)
\item MATH 3351 (Elementary Linear Algebra)
\item MATH 3354 (Survey of Algebra)
\item MATH 5653 (Number Theory)
\item MATH 5110 (Introduction to Stochastic Processes)
\item PHIL 1410 (Forms of Reasoning)
\item {\bf PHIL 2420 (Introduction to Symbolic Logic)}
\item PHIL 5420 (Symbolic Logic)
\item STAT 2120 (Introduction to Statistical Analysis)
\item STAT 5000 (Introduction to Applied Statistics)
\item STAT 5330 (Data Mining/Machine Learning)
\end{itemlist}

(Note: MATH 4040 (Discrete Mathematics) is not included because of
overlap with CS 2102.)

\paragraph{Life Sciences}
\begin{itemlist}
\item BIOL 3170 (Introduction to Neurobiology)
\item BIOL 3240 (Introduction to Immunology)
\item BIOL 4050 (Developmental Biology)
\item BIOL 4130 (Population Ecology and Conservation Biology)
\item BIOL 4160 (Functional Genomics)
\item BIOL 4170 (Cellular Neurobiology)
\item BIOL 4250 (Human Genetics)
\item BIOL 4480 (Macromolecular Structure)
\item BIOL 5080 (Developmental Mechanisms)
\item BIOM 3310 (Biomedical Systems Analysis and Design)
\item BIOM 3315 (Computational Biomedical Engineering)
\item BME 3636 (Neural Network Models)
\item PHIL 2330 (Computers Minds and Brains)
\item {\bf PSYC 2150 (Introduction to Cognition)}
\item PSYC 2200 (A Survey of the Neural Basis of Behavior)
\item {\bf PSYC 2300 (Introduction to Perception)}
\item PSYC 4200 (Neural Mechanisms of Behavior)
\item PSYC 4300 (Theories of Perception)
\item NESC 5330 (Neural Network Models)
\end{itemlist}

\paragraph{Physical Sciences}
\begin{itemlist}
\item EVSC 3020 (GIS Methods)
\item EVSC 5020 (Introduction to Geographic Information Systems)
\item EVSC 5030 (Applied Statistics for Environmental Scientists)
\item PHYS 2660 (Fundamentals of Scientific Computing)
\item {\bf PHYS 5630 (Computational Physics I)}
\item PHYS 5640 (Computational Physics II)
\end{itemlist}

\paragraph{Social Sciences}
\begin{itemlist}
\item ANTH 2430 (Languages of the World)
\item ANTH 3490 (Language and Thought)
\item ANTH 5040 (Linguistic Field Methods)
\item ANTH 5410 (Phonology)
\item ANTH 5420 (Theories of Language)
\item {\bf ECON 4010 (Game Theory)}
\item {\bf ECON 4020 (Auction Theory and Practice)}
\item ECON 4720 (Introductory Econometrics)
\item ECON 4880 (Seminar in Policy Analysis)
\item HIST 4510 (From Vellum to Very Large Databases)
\item LNGS 3250 (Introduction to Linguistic Theory and Methodology)
\item PSYC 4110 (Psycholinguistics)
\end{itemlist}

\paragraph{Using other courses.}  If a student would like to use a
course not on the above list as an integration elective, they should
first contact their academic advisor.  Their advisor can work with the
student to come up with a good argument as to why the course should
qualify, and once the advisor approves it, send it to the BA CS
Director of Undergraduate Programs (DUP) (currently
\badup\ (\badupemail)).  Alternatively, if the advisor prefers, s/he
can just send the student to DUP to get approval for a requirement
exception.

\mysection{Miscellaneous Information}

\subsection{Declaring the Major}

Before declaring the computer science major, students should have
taken one introductory computer science course (CS 111x (101, 101E,
101x), Introduction to Programming, or CS 1120 (150), Introduction to
Computing: Language, Logic, and Machines) with a grade of C+ or
better, or have comparable experience. Students may be permitted to
declare the major while they are currently taking the introductory
course.

To declare the major:

\begin{numlist}

\item Satisfy the major prerequisite by taking one of the introductory
  computer science courses. CS 1120 (150) is the recommended course
  for most BA CS majors, but the other introductory courses (CS 1110
  CS1111, and CS 1112) can also be used to satisfy the
  prerequisite. You may declare the major before completing the course
  as long as you are on track to complete the course successfully. If
  you believe you have comparable experience in some other way, you
  may also be able to declare the major.

\item Pick up a Major Declaration Form from the Dean's office, and
  fill out the top half.

\item Arrange to meet with \badup\ (\badupemail), Director of the
  Undergraduate Program (DUP). You can email to arrange a meeting
  time, or drop by \baduppronoun office hours.

\end{numlist}


\subsection{Distinguished Majors Program}


%Bachelor of Arts Computer Science 
BS CS
%
majors who have completed 18 credit hours towards their major and who
have a cumulative GPA of 3.4 or better may apply to the Distinguished
Majors Program. Students who are accepted must complete a thesis based
on two semesters of empirical or theoretical research. The
Distinguished Majors Program features opportunities for students and
advisors to collaborate on creative research; it is not a lock-step
thesis program with strict content requirements. Upon successful
completion of the program, students will likely be recommended for a
baccalaureate award of Distinction, High Distinction, or Highest
Distinction.

Students applying to the DMP must have a minimum cumulative GPA of 3.4
and have completed 18 credit hours towards their Computer Science
major by the end of the semester in which they apply. These 18 credit
hours can can come from any course used to fulfill the ``Major Subject
Requirements'', ``Computing Electives'' or ``Integration Electives''
of the 
%Interdisciplinary Major in Computer Science Curriculum.  curriculum.
(Exceptions to the 18 credit hours rule may be granted at the
discretion of the Distinguished Majors Program Director.)

In addition to the normal requirements for the computer science major,
they must register for two semesters of supervised research (CS 4998 (495)
for 3 credits each semester). Students may apply to the DMP before
completing this supervised research, but students must complete the
supervised research to complete the DMP. Based on their independent
research, students must complete, to the satisfaction of their advisor
and the Distinguished Major Program Director, a project at least one
month prior to graduation.

Please note: The CS 4998 DMP credits do not apply towards the credit
hours required for the major. That is, they cannot be used to fulfill
any requirement listed on the BA CS curriculum.

For more information on the DMP, see
online\myurl{http://www.cs.virginia.edu/ba/dmp/}. You may
also contact \badmp\ (\badmpemail), who is in charge of the BA DMP
program.

\subsection{Double majors in CLAS}

From the CLAS website on majors
(\myurl{http://artsandsciences.virginia.edu/college/major/major_type.html}),
regarding double majors:

\begin{quotation}
\noindent You may major in two subjects, in which case the application
for a degree must be approved by both departments or inter\-departmental
programs. Students who double major must submit at least 18 credits in
each major; these credits may not be duplicated in the other
major. There is no triple major.
\end{quotation}


\clearpage

\mysection{Course Requirements Flowchart}

\begin{figure}[h!]
\epsfig{figure=diagrams/ba-cs.pdf,width=4.5in}
\end{figure}

\noindent Notes:

\begin{itemlist}
\item CS 2102 requires {\em either} CS 111x or CS 1120 as a
  prerequisite.
\item Some instructors may not enforce CS 2330 as a prerequisite for
  CS 3330; check with the individual instructors to be sure.
\end{itemlist}


\clearpage
\mychapter{Bachelors of Science in Computer Engineering}{BS CpE Degree}

\mysection{Introduction}

Computer Engineering is an exciting field that spans topics across
electrical engineering and computer science.  Students learn and
practice the design and analysis of computer systems, including both
hardware and software aspects and their integration. Careers in
Computer Engineering (CpE) are as wide and varied as computer systems
themselves, which range from embedded computer systems found in
consumer products or medical devices, to control systems for
automobiles, aircraft, and trains, to more wide-ranging applications
in entertainment, telecommunications, financial transactions, and
information systems.

%\begin{quotation}
%Computer Engineering gives you a great working knowledge
%  and balance in both CS \& ECE. With the freedom to choose electives
%  in either department, you are in full control of your educational
%  experience and how you wish to enhance your knowledge. -- Kevin Chang,
%  08
%\end{quotation}

\subsection{Program Objectives}

Graduates of the Computer Engineering program at the
University of Virginia utilize their academic preparation
to become successful practitioners and innovators in
computer engineering and other fields. They analyze,
design and implement creative solutions to problems
with computer hardware, software, systems and
applications. They contribute effectively as team
members, communicate clearly and interact responsibly
with colleagues, clients, employers and society.

Faculty from the Computer Science and Electrical \& Computer
Engineering departments jointly administer the CpE undergraduate
degree program at the University of Virginia.

The Computer Engineering program does not offer a minor.

%\begin{quotation}
%It's the future.  Everything is digitized and computer
%  engineering allows you to keep up with changing technology. It’s a
%  complex field with many great opportunities for advancement.  -- Rob
%  Yip, ‘08
%\end{quotation}

\mysection{Curriculum} % called ``Disciplines'' in the original

The curriculum has been carefully designed to ensure that the students
obtain an excellent background in both Computer Science and Electrical
Engineering, providing breadth across these disciplines as well as
depth in at least one. All Computer Engineering students work through
an extended sequence of introductory, intermediate and advanced
courses:

\begin{itemlist}
\item CS 1110 (101) Introduction to Computer Science
\item CS 2110 (201) Software Development Methods
\item CS 2102 (202) Discrete Math
\item ECE 2630 (203) Introductory Circuit Analysis
\item ECE 2660 (204) Electronics I
\item CS 2150 (216) Program and Data Representation
\item ECE/CS 2330 (230) Digital Logic Design
\item ECE 3750 (323) Signals \& Systems I
\item CS/ECE 3330 (333) Computer Architecture
\item CS 3240 (340) Advanced Software Development
\item CS 4414 (414) Operating Systems
\item ECE 4435 (435) Computer Organization \& Design
\item ECE 4440 (436) Advanced Digital Design
\item CS/ECE 4457 (457) Computer Networks
\end{itemlist}

Please Note: Course numbers changed in 2009. The old course numbers
are shown in parentheses.

In addition to providing breadth across the two areas,
this core of the Computer Engineering program provides
depth in the following areas:

\paragraph{Circuits}
\begin{itemlist}
\item ECE 2630 (203): Introductory Circuit Analysis
\item ECE 2660 (204): Electronics I
\end{itemlist}

\paragraph{Software Engineering}
\begin{itemlist}
\item CS 2110 (201): Software Development Methods
\item CS 3240 (340): Advanced Software Development
\end{itemlist}

\paragraph{Digital Logic}
\begin{itemlist}
\item ECE/CS 2330 (230): Digital Logic Design
\item CS 2102 (202): Discrete Math
\end{itemlist}

\paragraph{Computer Systems}
\begin{itemlist}
\item CS 2150 (216): Program and Data Representation
\item CS/ECE 3330 (333): Computer Architecture
\item CS 4414 (414): Operating Systems
\item ECE 4435 (435): Computer Organization \& Design
\item ECE 4436 (436): Advanced Digital Design
\item CS/ECE 4457 (457): Computer Networks
\end{itemlist}

\subsection{Grade Requirement}
In completing their program of study, computer
engineering majors must achieve a “C” average or
better in their Computer Science and Electrical
Engineering courses.


%\begin{quotation}
%I decided to major in CPE because it gave me an opportunity
%  to combine two majors into one. I came to UVA interested in computer
%  science, but decided that I wanted to know more about the hardware
%  and doing CPE was the perfect choice for me.  In the long run having
%  this major can make the student more marketable because he or she
%  can take on careers in many paths. -- Alla Aksel, ‘04
%\end{quotation}

\subsection{Sample BS CpE Course Schedule}

\noindent \begin{tabular}{llc}
\und{First semester} & & \und{15} \\
APMA 1110 (111) & Single Variable Calculus & 4 \\
CHEM 1610 (151) & Chemistry for Engineers & 3 \\
CHEM 1611 (151L) & Chemistry Lab & 1 \\
ENGR 1620 (162) & Introduction to Engineering & 4 \\
STS 1010/1500 (101) & Engineering, Technology \& Society & 3 \\
& & \\
\und{Second semester} & & \und{17} \\
APMA 2120 (212) & Multivariate Calculus & 4 \\
PHYS 1425 (142E) & Physics I & 3 \\
PHYS 1429 (142W) & Physics I Workshop & 1 \\
CS 1110 (101) & Intro to Computer Science & 3 \\
SCI & Science elective$^2$ & 3 \\
HSS & HSS elective$^1$ & 3 \\
& & \\
\und{Third semester} & & \und{16} \\
APMA 2130 & Ordinary Differential Equations & 4 \\
CS 2110 (201) & Software Development Methods & 3 \\
CS 2102 (202) & Discrete Mathematics & 3 \\
ECE 2630 () & Introductory Circuit Analysis & 3 \\
HSS & HSS elective$^1$ & 3 \\
\end{tabular}
 
\noindent \begin{tabular}{llc}
\und{Fourth semester} & & \und{16} \\
CS 2150 (216) & Program and Data Representation & 3 \\
CS/ECE 2330 (230) & Digital Logic Design & 3 \\
ECE 2660 () & Electronics I & 4 \\
CS/ECE & CS/ECE elective $^{3,5}$ & 3 \\
STS STS & 2xx/3xx elective & 3 \\
& & \\
\und{Fifth semester} & & \und{16} \\
CS/ECE 3330 (333) & Computer Architecture & 3 \\
ECE 3750 () & Signals \& Systems & 3 \\
APMA 3100 & Probability & 3 \\
PHYS 2415 (241E) & General Physics II & 3 \\
PHYS 2419 (241W) & General Physics Lab I & 1 \\ 
UE & Unrestricted elective$^4$ & 3 \\
& & \\
\und{Sixth semester} & & \und{15} \\
CS 3240 (340) & Advanced Software Development & 3 \\
CS/ECE & CS/ECE elective $^{3,5}$ & 3 \\
CS 4414 (414) & Operating Systems & 3 \\
HSS & HSS elective$^1$ & 3 \\
UE & Unrestricted elective$^4$ & 3 \\
& & \\
\und{Seventh semester} & & \und{16.5} \\
ECE 4435 & Computer Org. \& Design & 4.5 \\
CS/ECE 4457 () & Computer Networks & 3 \\
CS/ECE & CS/ECE elective $^{3,5}$ & 3 \\
UE & Unrestricted elective$^4$ & 3 \\
STS 4010/4500 (401) & Western Tech and Culture & 3 \\
& & \\
\und{Eighth semester} & & \und{16.5} \\
ECE 4440 & Advanced Digital Design & 4.5 \\
CS/ECE & CS/ECE elective $^{3,5}$ & 3 \\
UE & Unrestricted elective$^4$ & 3 \\
STS 4020/4600 (402) & Engineer in Society & 3 \\
UE & Unrestricted elective$^4$ & 3 \\
\end{tabular}

\paragraph{Footnotes:}

\begin{numlist}
\item Chosen from the approved list available in A122 Thornton Hall.
\item Chosen from: among BIOL 2010, BIOL 2020, CHEM 1620, ECE 2066, ENGR
  2500, MSE 2090, and PHYS 2620.
\item Students interested in selected advanced CS electives should
  consider CS 3102.  Students interested in selected advanced ECE
  electives can delay this elective until the sixth semester and take
  another elective instead.
\item Unrestricted electives may be chosen from any graded course in
  the University except mathematics courses below MATH 1310 including
  STAT 1100 and 1120, and courses that substantially duplicate any
  others offered for the degree, including PHYS 2010, 2020; CS 1010,
  1020; or any introductory programming course. Students in doubt as
  to what is acceptable to satisfy a degree requirement should get the
  approval of their advisor and the dean's office, located in Thornton
  Hall, Room A122. APMA 1090 counts as a three-credit unrestricted
  elective.
\item Chosen from CS/ECE course at the 3000 level or higher. Two
  CS/ ECE electives must be 4000 level or above.
\end{numlist}


\mysection{Miscellaneous Information}

Please refer to the Undergraduate Record for detailed information
about SEAS Academic Rules and Regulations including HSS
electives. Guidelines such as Course Load, Academic Probation and
Academic Suspension can also be found in the Record.

The Registrar web site provides a Course Renumbering Crosswalk to
assist with the transition from 3 to 4 digit course
numbers\myurl{http://www.virginia.edu/registrar/search.php}.


%\clearpage
%\mysection{Course Requirements Flowchart}
%\begin{figure}[h!]
%\epsfig{figure=flowcharts/cpe-flowchart.png,width=4.5in}
%\end{figure}

\clearpage
\mysection{Course Requirements Flowchart}

\begin{figure}[h!]
\epsfig{figure=diagrams/bs-cpe.pdf,width=4.5in}
\end{figure}


\clearpage
\mychapter{Minor in Computer Science}{CS Minor}
\label{sec:csminor}

\mysection{Introduction}

The Department of Computer Science provides a minor program for
qualified students. The courses in the minor program provide a solid
foundation in computer science. The minor program is a six course,
eighteen credit curriculum. The curriculum consists of the four
required courses and two elective courses. Full course descriptions
are at the end of this document, beginning on
page~\pageref{sec:coursedesc}.

In the past, there were separate requirements for the minor for SEAS
students and non-SEAS students.  These requirements have been
streamlined into a single set of requirements for everybody.
 
\mysection{Curriculum}

All SEAS (School of Engineering and Applied Science) students are
required to take (or place out of) CS 1110 (101), as part of the SEAS
first-year curriculum. This course is also the first required course
for the minor.

The following are the first four courses required for the minor.

\begin{itemlist} 
\item CS 1110 (101), CS 1111 (101E), or CS 1112 (101X): Introduction
  to Computer Science
\item CS 2110 (201): Software Development Methods
\item CS 2102 (202): Discrete Mathematics
\item CS 2150 (216): Program and Data Representation
\end{itemlist}

Note that CS 1120 (From Ada and Euclid to Quantum Computing and the
World Wide Web) can replace the CS 111x requirement.  However, all
SEAS students are required to take a CS 111x course regardless, so
courses, so taking CS 1120 would not help at all.

Likewise, CS 2220 (Engineering Software) can replace CS 2110.

Note that if you place out of CS 1110 (101) via the placement exam,
you still have to take 6 CS courses; if you receive course credit for
it via the AP exam or transfer credit, then you need not substitute a
course in its place.

Furthermore, two additional computer science electives are
required. The elective courses must be computer science courses at the
3000 level or above. The only restriction on elective courses is a
limit to how many independent study courses one can count toward a
minor~-- contact the minor advisor for details at
\csminoradvisoremail.

Computer science courses typically build upon each other. In
particular, CS 1110 (101) is a prerequisite of both CS 2110 (201) and
CS 2102 (202). CS 2110 (201) and CS 2102 (202) are both prerequisites
of CS 2150 (216). In addition, CS 2150 (216) is a prerequisite for
almost all of the computer science electives. The Department of
Computer Science also requires that its courses be passed at a certain
level (typically a C- or higher) in order to take successive
courses. Be aware that the department strictly enforces its
prerequisite policy.

 
\mysection{Miscellaneous Information}
\subsection{Declaring the minor}

To declare the minor:

\begin{numlist}

\item A student should have completed CS 1110 (101) or 1120 (150), CS
  2110 (201) or 2220 (205), and CS 2102 (202). Furthermore, the
  student should have completed, or at least be enrolled in, CS 2150
  (216).

\item Complete the minor declaration form, which is available in the
  Computer Science department's front office (Rice Hall, room 527).
  The form has the title, ``School of Engineering and Applied Science,
  Minor Declaration''~-- this is the form for everybody; the SEAS
  school title at the top is because the CS department is in SEAS.

\item Meet with the CS department's minor advisor(s), currently
  \csminoradvisor\ (\csminoradvisoremail).  Bring a the minor
  declaration form and your transcript (unofficial is fine).

\item Assuming the form is approved, it will be processed by the
  department.

\end{numlist}


\clearpage
\mychapter{Masters in Computer Science}{Masters in CS}

% this information is via Wes who spoke with Kathy Thornton in January 2011

\mysection{Introduction}

There are multiple ways that one can pursue a graduate degree.
Typically, undergraduates are interested in completing a Masters
program in 5 years, 1 year beyond that for a Bachelors.

The department maintains graduate program information
online\myurl{http://www.cs.virginia.edu/grad_curriculum/};
that website contains more complete information than this chapter.
This section pertains solely to obtaining a Masters in 5 years.

Students are often better served going to a different school for
graduate work.  Every school has biases and ways of doing things, and
if one spends all of their academic career at one institution, then
they don't see any other way.  This is especially true for
Ph.D.\ degrees, but also important (although less so) for Masters.

Students are are aiming for a 5-year Masters must still follow the
same rules and guidelines for all Computer Science and SEAS graduate
students.  These rules and guidelines can be found in the graduate
handbook (see above), and
online\myurl{http://www.seas.virginia.edu/admissions/graduate.php}.

We would like to stress again that this chapter focuses solely on
earning a Masters degree in 5 years, where the Bachelors also was
earned at UVa.


\subsection{When to Apply}

Although you are earning both degrees in 5 years, there is still will
be a formal switch between undergraduate and graduate student status.
Also see the graduate applications web
page\myurl{http://www.seas.virginia.edu/admissions/graduate.php}.

A student applies to the UVa graduate program in CS like anybody else,
but mentions in the application packet (specifically, in the statement
of purpose) that they are going for a terminal 5-year Masters.  Here,
'terminal' means that you are not pursuing a degree beyond that (i.e.,
a Ph.D.).

Note that in the application process, you are NOT considered a
transfer student, even if you already have taken some graduate courses
at UVa.

The easiest time to apply is in the fall of your 4th year (i.e. during
your 7th semester).  This would allow one to finish up 8 full
semesters as an undergraduate, and have 2 full semesters (plus
summers, potentially) as a graduate student.

Often students will use the summer (either before or after their 5th
year) to complete some Masters requirements.  Note that it is very
possible to complete a Masters in 1 year, but it will be a heavy work
load if classes are not taken during the summers.  Check the course
availability to see what courses, if any, are being offered.

One can certainly take more than 5 year to complete the Masters~-- you
are paying tuition, after all~-- but typically students aim to
complete their Masters after 5 full years.

However, nothing requires that you apply at that time.  You can apply
in the spring of your 3rd year (i.e. 6th semester) or even earlier.
Should one decide to apply then (6th semester), they may graduate
early in 3.5 years (i.e. 7 semesters)~-- thus, they graduate 1
semester early from undergraduate, then enroll in their 8th semester
as a Masters student.

The benefit of applying a semester early is that one can have an
additional semester to work on graduate courses.  This must be
balanced with the concern of completing their undergraduate degree in
7 semesters.

Note that if you are a SEAS student and are graduating one semester
early, you must still write a senior thesis, and take STS 4500 and STS
4600.

One can apply earlier as well, such as in the fall of one's 3rd year
(i.e. in your 5th semester).  This would mean that one would complete
the undergraduate degree in 3 years (following the same time line as
completion of the undergraduate degree in 8 semesters, but accelerated
by 1 year), and complete the Masters in their 4th year.


\subsection{Degrees Offered}

The Department of Computer Science offers two different Masters
degrees.  The first is a Masters of Computer Science (MCS), and the
second is a Masters of Science in Computer Science (MS).  Both may be
obtained in 5 years, although most students will opt for the MCS.

From the perspective of employers, the two degrees are, for the most part,
equivalent.  The primary difference is that a MS requires a full
Masters thesis, with a complete faculty committee that looks for a
significant amount of work to have been accomplished.  A MS requires a
3-credit project, and that is judged only by the student's advisor.  A
faculty committee looking at a MS thesis will look for significantly
more work than what is required for an MCS project.  As a result, a
MCS is an easier degree to earn.

Students tend to complete the MCS instead of the MS, as the MS
requires a significant thesis, and the MCS requires a project that is
smaller in scope.


\mysection{Curriculum}

The full curriculum for a Masters degree is listed in the graduate
handbook.  As of publication of this version of this undergraduate
handbook, the graduate handbook is only a draft, and is not a final
version yet.  It can be found
online\myurl{http://www.cs.virginia.edu/~csgsg/graduate-guide/}. This
section is only intended as a summary.

A Masters curriculum generally consists of 30 credits (i.e. 10
courses) in computer science.  One or two of these courses will be the
MCS project course or the MS thesis course(s).

Any course that counts towards the graduation requirements for your
undergraduate degree may {\em\bf NOT} count towards the graduation
requirements for your Masters~-- even if it is as an unrestricted
elective.  Thus, if you want to take graduate class(es) as an
undergraduate, and you want it to count towards your Masters
graduation requirements, you must ensure that you take enough classes
so that you could have graduated {\em\bf without} those graduate
class(es).  Thus, one must carefully work out which courses will count
for which degree.

Masters students do not need to take (or pass) the qualification
exams, as those are required for the Ph.D.\ degree only.  If you
decide to later transfer into the Ph.D.\ degree, then you will need to
take (or have taken) the qualification exams.

\mysection{Miscellaneous Information}

Generally, Masters students are not funded.  Thus, students will pay
tuition (and room/board, as appropriate).  The costs are analogous to
undergraduate rates: lower for in-state residents, and higher for
out-of-state residents.

The ``easiest''~-- and most typical~-- path to a 5-year Masters is to
apply in 7th semester, and already have some graduate classes that are
NOT counting towards your undergraduate degree.  You will have had to
have talked to your undergraduate advisor about who you are going to
work with for your Masters.  You would then complete the degree in 1
full additional year (summer, fall, and spring), aiming for a May
graduation date.


\clearpage
\mychapter{Common Information}{Common Information}

\mysection{Major Focal Paths}

A focal path is a selection of courses that a student can take to
fulfill the various elective requirements, which are described in
detail in the sections on elective information for the various
majors. They do not change any of the requirements, and students are
not required to follow a focal path. They are included simply to give
prospective majors an idea about various classes that they can take to
fulfill an interest that they may have in computing. Not all focal
paths have classes to fulfill each elective requirement. And some will
have more classes than are needed for the given requirement.

In an effort to keep down the space for each listing, the reason for
each class is not listed – if interested, speak to a CS faculty member
in that particular area. Also, as BA CS students may be interested in
these focal paths, a line listing the BA CS requirements is also shown
below.

There are a number of other areas for which focal paths are being
developed, and we expect to include them in future editions of this
handbook. Those areas are: Systems, Parallel \& Distributed Computing,
Graphics, Languages \& Compilers, Software Engineering, Hardware, and
Security \& Privacy.


\subsection{Game Design}
\begin{itemlist}
\item Science elective (1): N/A
\item HSS electives (5): anything relating to asset development
  (sound, images, video, etc.)
\item Unrestricted elective (5): digital art classes, such as ARTS
  2220, 2222, 3220, 3222, 4220, and 4222; sound design courses, such
  as DRAM 2620 and DRAM 3640; modeling classes such as ARCH 3410.
  Also consider additional CS electives.
\item APMA electives (2): linear algebra (APMA 3080)
\item CS electives (5): game development courses (offered as special
  topics courses, CS 4501); graphics (CS 4810), artificial
  intelligence (CS 4710), networks (CS 4457), databases (CS 4750),
  parallel computing (CS 4444)
\item STS 2xxx/3xxx elective (1): N/A
\item Notes: You will need a lot of C++ experience upon graduation
\end{itemlist}

\subsection{Theory}
\begin{itemlist}
\item Science elective (1): ECE 2066 (200) (Science of Information)
\item HSS electives (5): mathematical economics (ECON 3090),
  psycho-linguistics (PSYC 4110)
\item Unrestricted electives (5): game theory (ECON 401), various math
  courses (MATH 4452, MATH 5700, STAT 3010)
\item Unrestricted elective (1): N/A
\item APMA electives (2): linear algebra (APMA 3080)
\item CS electives (5): programming languages (CS 4610), artificial
  intelligence (CS 4710), cryptography (offered as special topics
  courses, CS 4501)
\item STS 2xxx/3xxx elective (1): N/A
\item Notes: BA students need to take CS 302 which is critical for a
  theory focal path, but is not (at this time) a required course for
  the BA
\end{itemlist}

\subsection{Networks (including wireless networks)}
\begin{itemlist}
\item Science elective (1): ECE 2066 (Science of Information)
\item HSS (5): N/A
\item Unrestricted electives (5): ECE 2630 (Circuits), ECE 3750
  (Signals), ECE 4710 (Communications), ECE 4290 (Wireless Systems),
  ECE 4785 (Optical Communications)
\item APMA electives (2): APMA 3120 (Statistics)
\item CS electives (5): CS 4457 (Networks), CS 4458 (Internet
  Networks), all wireless sensor networks courses offered as special
  topics, CS 4753 (Electronic Commerce), CS 4720 (Web and Mobile
  Systems), CS 4444 (Parallel Computing)
\item STS 2xxx/3xxx elective (1): N/A
\item Notes: The wireless networking class is often offered as a graduate class (called wireless sensor networks) and can be added with instructor permission. 
%\item Science elective (1): N/A
%\item HSS electives (5): N/A
%\item Unrestricted electives (5): electronic commerce (SYS 2057),
%  hardware based communications (ECE 4710, ECE 4784)
%\item Unrestricted elective (1): electronic commerce (COMM 4240)
%\item APMA electives (2): N/A
%\item CS electives (5): networks (CS 4457), Internet networks (CS
%  4458), wireless networks (offered as special topics courses, CS
%  4501), electronic commerce (CS 4753), parallel computing (CS 4444)
%\item STS 2xxx/3xxx elective (1): N/A
%\item Notes: the wireless networking class is often offered as a
%  graduate class (called wireless sensor networks), and can be added
%  with instructor permission
\end{itemlist}

\subsection{Web technologies}
\begin{itemlist}
\item Science elective (1): ECE 2066 (Science of Information)
\item HSS (5): Digital art classes, ECON 2010 (Micro Economics), ECON 2020 (Macro Economics), PSYC 1010 (Intro to Psychology)
\item Unrestricted electives (5): STS 4110 (Business of New Product Development)
\item APMA electives (2): N/A
\item CS electives (5): CS 4753 (Electronic Commerce), CS 4457 (Networks), CS 4720 (Web and Mobile Systems), CS 4750 (Database Systems), CS 4240 (Software Design)
\item STS 2xxx/3xxx elective (1): STS 2160 (Intellectual Property)
\item Notes: There are a number of IT classes that are relevant, including courses in web design, technology, and marketing.  However, these are not allowed as unrestricted electives per SEAS policy. 
%\item Science elective (1): ECE 200
%\item HSS and unrestricted electives (10): digital art classes
%\item Unrestricted electives (5): COMM 424 (if no 453), TMP 351
%\item APMA electives (2): N/A
%\item CS electives (5): Electronic commerce (CS 5753); networks (CS
%  4457); Internet networks (CS 4458); web-based courses (offered as
%  special topics courses, CS 4501)
%\item STS 2xxx/3xxx elective (1): STS 2160
%\item Notes: There are a number of IT classes that are relevant,
%  including IT 323 (Web design), IT 332 (Web Tech), IT 334 (Web
%  marketing). However these are not allowed as unrestricted electives
%  as per SEAS policy.
\end{itemlist}

\subsection{Software Engineering}
\begin{itemlist}
\item Science elective (1): Science of Information (ECE 2066)
\item HSS electives (5): See note below regarding the Engineering
  Business Minor.
\item Unrestricted elective (5): See note below regarding the
  Engineering Business Minor.
\item APMA electives (2): any choices are suitable
\item CS electives (5): Principles of Software Design (CS 4240) and
  HCI in Software Development (CS 3205) have a strong software
  engineering focus. Other courses that include significant development
  projects would be appropriate, such as Databases (CS 4750), Web and
  Mobile Systems (CS 4720), and Computer Game Design (CS 4730).
\item STS 2xxx/3xxx elective (1): Any offerings related to technology
  in society or ethics would be appropriate
\item Notes: If special topics courses were offered in software
  testing, software quality, or formal methods, these would be good
  choices for this focal path. Also, the Engineering Business Minor
  would be a good addition for this focus. Finally, experience with in
  a software company through a summer internship will increase your
  understanding of this area.
\end{itemlist}

\mysection{Frequently Asked Questions}

\subsection{What computer science student groups exist?}

There are three main computer science student groups at UVa.

The Association for Computing Machinery Chapter at the University of
Virginia is a student chapter of the parent Association for Computing
Machinery. The Chapter is a Contracted Independent Organization (CIO)
at the University of Virginia, and serves students, faculty, and staff
of the University as well as members of the Charlottesville /
Albermarle community. Any member of the University or Charlottesville
/ Albermarle community may become a Member of the Chapter.  Also see
their website\myurl{http://acm.cs.virginia.edu/}.

ACM-W is the ACM committee on Women in Computing. It celebrates,
informs and supports women in computing, and works with the ACM-W
community of computer scientists, educators, employers and policy
makers to improve working and learning environments for women. Also
see their website\myurl{http://www.cs.virginia.edu/~acm-w/}.

The Student Game Developers seeks to bring together students who are
interested in learning and experiencing the art of computer game
development. They have resources available for programmers as well as
non-programmers, weekly informative meetings, and many industry
contacts for lectures, resume building, and networking. Also see their
website\myurl{http://gamedevatuva.blogspot.com/}. 

\subsection{What is ICPC, the International Collegiate Programming
  Contest, and how do I get involved?}

The International Collegiate Programming Contests, abbreviated ICPC,
is a world-wide contest of computer programming for college
students. UVa has a very active programming contest team. Regional
contests occur in the fall – our region is the nearest 6 (or so)
states and D.C. The top team(s) from each regional contest advance to
the world finals, which consists of the top 100 teams from around the
world. UVa has qualified for the world finals twice in the recent
years: for the 2009 world finals in Stockholm, Sweden, and the 2010
world finals in Harbin, China. We typically have seven teams (of three
students each) compete in the regional contest. Our programming
contest teams practice throughout the year. If you are interested in
more information, you can either contact UVa’s local ACM
chapter\myurl{http://acm.cs.virginia.edu} or ACM’s advisor,
\acmadvisor\ (\acmadvisoremail).

\subsection{What kind of advanced placement credit is available?}
\label{applacement}

Advanced placement (AP) credit is awarded by the University for most
AP tests in which the grade is a 4 or a 5. This section only deals
with the AP computer science test. A student's SIS report will
always list which courses qualify for the AP test scores (both in
computer science and in other fields).

A 5 on the computer science AB test will receive course credit for CS
1110 (101) and CS 2110 (201). A 4 on the computer science AB test OR a
5 on the computer science A test will receive course credit for CS
1110 (101). If the AP exam was not in Java, proficiency in Java must
be demonstrated prior to taking CS 2110 (201). Note that CS 2110 (201)
is required for other majors: computer engineering, systems
engineering, and electrical engineering. There is also a placement
exam before the fall semester that will allow the student to place out
of CS 1110 (101), but does not allow credit to be received for the
course~-- the student must then take another 3 hour CS or technical
course (see your advisor for details about a ‘technical course’)
instead. See the next question and answer for information about the CS
1110 (101) placement exam.

\subsection{Can I place out of CS 1110 (101)? What about CS 2110
  (201)?}
\label{101placement}

There is a placement exam for CS 1110 (101), which covers all the
topics taught in the course. For the current semester's syllabus, see
the CS 1110 (101) course website. Successful completion will allow a
student to place out of the course, but does NOT give course credit~--
only a sufficient score on the AP test or transfer credit can give
course credit for CS 1110 (101). A student must still take CS 2110
(201) or a technical course (see your advisor for details about a
‘technical course’) to fulfill the SEAS CS 1110 (101) requirement. The
test is offered before the beginning of the fall semester. Note that
any student who has enrolled in CS 1110 (101) or equivalent (1111,
1112) and got a letter grade – including a ‘W’~-- is not allowed to
take the placement exam (in other words, if you enroll and then drop
the course without a ‘W’, you may still take the placement exam). The
exam may be taken by visiting the departmental office in Rice Hall,
room 527.

More information about the CS 1110 (101) placement exam may be found online
\myurl{http://www.cs.virginia.edu/cs1110/placement.php}.

For information about the placement exam for CS 2110 (201), please
contact the current CS 2110 (201) instructor.

\subsection{How does SEAS handle transfer credit?}

The Engineering school handles transfer credit, such as from an AP
course or transfer from another school. The credit will appear on your
SIS report, along with the UVa courses that you received credit
for. Note that the credit amounts need to match - so if you are
getting credit for APMA 2120 (212) (Multivariate Calculus), which is a
4 credit course, the number of credits you transfer in should
(ideally) also be 4 credit hours. If it does not (your equivalent
course at another school was only 3 credits), you will have to take
another math or technical course (see your advisor for details about a
‘technical course’) to make up for the discrepancy. Note that placing
out of a course (such as CS 1110 (101), APMA 2120 (212), etc.) through
the respective placement exam does not give credit – and thus the
credits need to be made up through other courses (in the case of CS
1110 (101), 3 credits of a technical course will fill that spot; in
the case of APMA 2120 (212), 4 credits of math or a technical elective
will fill that spot). AP exams do give course credit.

Note that half of the 128 credits that one uses to graduate must be
earned at UVa. Thus, if you transfer with more than 64 credits, you
must still take 64 credits at UVa.

\subsection{Can CS courses from another college receive credit?}

We officially discourage taking major courses elsewhere. This policy
is especially true for the lab-based and required courses. If, in
spite of this departmental policy, you still want to take a course
elsewhere, then the student needs an advisor signature AND the
signature of the current instructor of that course from UVa. To
receive the required signatures, you must bring in a detailed
syllabus, so that faculty can make informed decisions. Note that to
receive credit for CS 2150 (216) elsewhere, you need a course (or
multiple courses) that cover(s) data structures, C++, and assembly
language programming.

\subsection{What are the Rodman Scholar requirements?}

Rodman scholars have slightly different requirements for graduation.

\begin{itemlist}
\item Rodman Scholars are not required to take STS 1010/1500 (101); an HSS
elective is substituted
\item Rodmans fulfill the STS 2xxx/3xxx elective by taking STS 2000
  (formerly known as STS 200R). For the class of 2013, Rodmans will
  take STS 1010/1500 in the Spring, with a special discussion section for
  Rodmans only.
\item In place of ENGR 1620, Rodmans instead take the two-course
  sequence, ENGR 1410 (141R) and ENGR 1420 (142R)
\item Rodmans take PHYS 1427 (142R) during their first semester,
  instead of PHYS 1425 (142E) and PHYS 1429 (142W) in the second
  semester
\item Rodmans take a different Physics class during their second
  semester, instead of PHYS 2415 (241E) and PHYS 2419 (241W). This
  class used to be called PHYS 241R; as of print time, the new 4-digit
  course number is not yet known.
\end{itemlist}

Furthermore, Rodman Scholars are required to complete 4 seminars~--
ENGR 3580 (formerly 307 and 308) prior to graduation. First-year
students joining the Rodman program at mid-year are required to take
three seminars prior to graduation.

\subsection{Why are ECE 4435 and ECE 4440 not showing up in my list of
  fulfilled CS electives?}
\label{sec:sisece4435issue}

This has to do with a restriction in how SIS handles the CS elective
requirements.  While ECE 4435 (435) (Computer Organization \& Design)
and ECE 4440 (436) (Advanced Digital Design) can both count as a CS
elective each, in order for this to happen a SIS exception will need
to be entered.  Your academic advisor can request to have such an
exception entered by contacting \sisexceptionenterer.  Note that both
of those courses only count as one elective each, even though they are
4.5 credits per course (meaning that taking both of those classes~--
worth 9 credits total~-- does not count as three CS electives, but
only as two).

\subsection{Why do the SIS requirements for the BS CS major list 6 HSS
  electives, and not 5?}
\label{sec:sishssissue}

This has to do with how SIS (the Student Information System, UVa’s
system for handling academic requirements and registration) handles
major requirements, and is done to allow for people to place out of
STS 1500 (previously STS 101, and STS 1010). If one does not
place out of STS 1500, then STS 1500 will list both in the STS 1500
requirement, and in the HSS requirement, thus requiring students to
take 5 additional HSS courses. If one does place out of STS 1500, they
need to take an additional HSS course in its place. So the credit to
place out of STS 1500 will appear in the STS 1500 requirements, and
will still require 6 (not 5) HSS courses. We think this is all a bit
bizarre as well, but that is how SIS handles requirements.

A sample of the BS CS requirements can be found
online\myurl{http://www.cs.virginia.edu/bscs/bscs-reqs-in-sis.pdf}~--
your individual one can be found via SIS.

\subsection{Can CS students study abroad?}

Yes! To get more information about studying abroad, see
online\myurl{http://www.cs.virginia.edu/curriculum/study\_abroad/}
for more details.
 
\subsection{How do I transfer into the CS program?}

Like other SEAS students, transfer students must formally apply to,
and be approved by, the Department of Computer Science to enroll in
the computer science program of study. To minimize loss of credit upon
transfer, students must take a rigorous program in mathematics and the
sciences. The School of Engineering and Applied Science expects a
minimum of 63 credits in the first two years, instead of the 60-credit
minimum that is customary in the College of Arts and Sciences. The
additional credits are often completed through summer
courses. Detailed information on curriculum requirements may be
obtained from the Office of the Dean of the School of Engineering and
Applied Science.

There is also the Bachelor of Arts in Computer Science, offered
through the College; also see their main
web page\myurl{http://www.cs.virginia.edu/ba/}. Students
outside of the School of Engineering and Applied Science with an
interest in obtaining a BS (as opposed to a BA) degree in computer
science must transfer to the Engineering school.


\subsection{Where can I find out about the Business minor?}

The courses for the Engineering Business Minor can be worked into the
various electives for the BS CS. More details can be found
online\myurl{http://www.seas.virginia.edu/advising/businessminor.php}.


\subsection{What CS electives can be taken without having completed CS
  2150 (216)?}

There are a few CS electives that one can take having only taken CS
2110 (201).  They include:
\begin{itemlist}
\item CS 3102 (302), Theory of Computation
\item CS 3205 (305), Human-Computer Interaction
\item For the BA CS, CS 2330 (230), Digital Logic Design, counts as an
  elective
\item CS 3330 (333), Computer Architecture.  This course requires CS
  2330 (230), Digital Logic Design, as a prerequisite.  There are a
  few courses that require CS 3330 (333) as a prerequisite, but do not also
  require CS 2150 (216):
  \begin{itemlist}
  \item CS 4434 (434), Fault-tolerant Computing (note that this course
    has other prerequisites)
  \item CS 4457 (457), Computer Networks
  \end{itemlist}
\end{itemlist}

\subsection{Why is CS 2330 (230), Digital Logic Design, not offered
  in the spring?}
\label{cs2330}

CS 2330 (230), Digital Logic Design, is cross-listed with ECE 2330
(230).  Either course counts for this requirement, and it does not
matter which one you take.  For unknown reasons, it is not
cross-listed with CS 2330 (230) in the spring, but it is in the fall.
We don't understand why, either.  But you can take ECE 2330 (230) to
fulfill this requirement, as it's all the same course.



\clearpage
\mysection{Course Descriptions}
\label{sec:coursedesc}

These course listings are from the undergraduate
record\myurl{http://records.ureg.virginia.edu/content.php?catoid=11\&navoid=189}. The
frequency code for each class specifies how often it is offered. (S)
means offered each (spring and fall) semester; (Y) means offered once
each academic year, and (SI) means offered upon sufficient student
interest.

\subsection{1000 Level Courses}

\course{CS 1010}{110}{Introduction to Information Technology}{3
  credits}{Provides exposure to a variety of issues in information
  technology, such as computing ethics and copyright. Introduces and
  provides experience with various computer applications, including
  e-mail, newsgroups, library search tools, word processing, Internet
  search engines, and HTML. Not intended for students expecting to do
  further work in CS. Cannot be taken for credit by students in SEAS
  or Commerce.}{}{S}

\course{CS 1110}{101}{Introduction to Programming}{3
  credits}{Introduces the basic principles and concepts of
  object-oriented programming through a study of algorithms, data
  structures and software development methods in Java. Emphasizes both
  synthesis and analysis of computer programs.}{}{S}

\course{CS 1111}{101E}{Introduction to Programming}{3
  credits}{Introduces the basic principles and concepts of
  object-oriented programming through a study of algorithms, data
  structures and software development methods in Java. Emphasizes both
  synthesis and analysis of computer programs.}{Prerequisite: Prior
  programming experience.}{S}

\course{CS 1112}{101X}{Introduction to Programming}{3
  credits}{Introduces the basic principles and concepts of
  object-oriented programming through a study of algorithms, data
  structures and software development methods in Java. Emphasizes both
  synthesis and analysis of computer programs. Note: No prior
  programming experience.}{}{SI}

\course{CS 1120}{150}{Introduction to Computing: Language, Logic, and
  Machines}{3 credits}{Introduction to computer science with no
  previous background. Focuses on describing and reasoning about
  information processes using language and logic. Uses motivating
  examples from liberal arts and sciences areas such as art, biology,
  economics, narrative, physics, and sociology.}{}{Y}

\subsection{2000 Level Courses}

\course{CS 2102}{202}{Discrete Mathematics}{3 credits}{Introduces
  discrete mathematics and proof techniques involving first order
  predicate logic and induction. Application areas include sets
  (finite and infinite), elementary combinatorial problems, and
  probability. Development of tools and mechanisms for reasoning about
  discrete problems. Cross-listed as APMA 202.}{Prerequisite: CS 1110
  (101) or 1120 (150) with a grade of C- or higher.}{S}

\course{CS 2110}{201}{Software Development Methods}{3 credits}{A
  continuation of CS 1110 (101), emphasizing modern software
  development methods. An introduction to the software development
  life cycle and processes. Topics include requirements analysis,
  specification, design, implementation, and verification. Emphasizes
  the role of the individual programmer in large software development
  projects.}{Prerequisite: CS 1110 (101) with a grade of C- or
  higher.}{S}

\course{CS 2150}{216}{Program and Data Representation}{3 credits}
  {Introduces programs and data representation at the machine
  level. Data structuring techniques and the representation of data
  structures during program execution. Operations and control
  structures and their representation during program
  execution. Representations of numbers, arithmetic operations,
  arrays, records, recursion, hashing, stacks, queues, trees, graphs,
  and related concepts.)}{Prerequisite: CS 2102 (202) and either CS
  2110 (201) or CS 2220 (205) with all grades of C- or higher.}{S}

\course{CS 2190}{290}{Computer Science Seminar I}{1 credit}{Provides
  cultural capstone to the undergraduate experience. Students make
  presentations based on topics not covered in the traditional
  curriculum. Emphasizes learning the mechanisms by which researchers
  and practicing computer scientists can access information relevant
  to their discipline, and on the professional computer scientist's
  responsibility in society.}{Prerequisite: CS 2110 (201) or 2220
  (205) with a grade of C- or higher, as well as a computing major
  (BA CS, BS CS, or BS CpE).}{Y}

\course{CS 2220}{205}{Engineering Software}{3 credits}{Covers tools
  and techniques used to manage complexity and to build, analyze, and
  test complex software systems including abstraction, analysis, and
  specification.  Notes: Students may not receive credit for both CS
  2110 (201) and CS 2220 (205).}{Prerequisite: CS 150 with a grade of
  C- or higher.}{Y}

\course{CS/ECE 2330}{230}{Digital Logic Design}{3 credits}{Includes
  number systems and conversion; Boolean algebra and logic gates;
  minimization of switching functions; combinational network design;
  ip-ops; sequential network design; arithmetic networks. Introduces
  computer organization and assembly language. Cross-listed as ECE
  2330 (230).}{}{S}

\course{CS 2501}{251}{Selected Topics in Computer Science}{1 to 3
  credits}{Content varies annually, depending on students needs and
  interests. Recent topics include the foundations of computation,
  artificial intelligence, database design, real-time systems,
  Internet engineering, and electronic design
  automation.}{Prerequisite: Instructor permission.}{SI}

\subsection{3000 Level Courses}

\course{CS 3102}{302}{Theory of Computation}{3 credits}{Introduces
  computation theory including grammars, finite state machines and
  Turing machines; and graph theory. Cross-listed as APMA 3102
  (302).}{Prerequisite: CS 2102 (202) and either CS 2110 (201) or 2220
  (205) all with grades of C- or higher.}{Y}

\course{CS 3205}{305}{HCI in Software Development}{3
  credits}{Human-computer interaction and user-centered design in the
  context of software engineering. Examines the fundamental principles
  of human-computer interaction. Includes evaluating a systems
  usability based on well-defined criteria; user and task analysis, as
  well as conceptual models and metaphors; the use of prototyping for
  evaluating design alternatives; and physical design of software
  user-interfaces, including windows, menus, and
  commands.}{Prerequisite: CS 2110 (201) or 2220 (205) with a grade of
  C- or higher.}{Y}

\course{CS 3240}{340}{Advanced Software Development Techniques}{3
  credits}{Analyzes modern software engineering practice for
  multi-person projects; methods for requirements specification,
  design, implementation, verification, and maintenance of large
  software systems; advanced software development techniques and large
  project management approaches; project planning, scheduling,
  resource management, accounting, configuration control, and
  documentation.}{Prerequisite: CS 2150 (216) with a grade of C- or
  higher.}{Y}

\course{CS/ECE 3330}{333}{Computer Architecture}{3 credits}{Includes
  the organization and architecture of computer systems hardware;
  instruction set architectures; addressing modes; register transfer
  notation; processor design and computer arithmetic; memory systems;
  hardware implementations of virtual memory, and input/output control
  and devices. Cross-listed as ECE 3330.}{Prerequisite: CS 2110 (201)
  or 2220 (205) with a grade of C- or higher, and CS/ECE 2330 (230)
  with a grade of C- or higher.}{S}

\course{CS 3501}{351}{Selected Topics in Computer Science}{1 to 3
  credits}{Content varies annually, depending on students needs and
  interests. Recent topics include the foundations of computation,
  artificial intelligence, database design, real-time systems,
  Internet engineering, and electronic design
  automation.}{Prerequisite: Instructor permission.}{SI}

\subsection{4000 Level Courses}

\course{CS 4102}{432}{Algorithms}{3 credits}{Introduces the analysis
  of algorithms and the effects of data structures on them. Algorithms
  selected from areas such as sorting, searching, shortest paths,
  greedy algorithms, backtracking, divide- and-conquer, and dynamic
  programming. Data structures include heaps and search, splay, and
  spanning trees. Analysis techniques include asymptotic worst case,
  expected time, amortized analysis, and reductions between
  problems.}{Prerequisite: CS 2150 (216) with a grade of C- or
  higher.}{Y}

\course{CS 4240}{441}{Principles of Software Design}{3
  credits}{Focuses on techniques for software design in the
  development of large and complex software systems. Topics will
  include software architecture, modeling (including UML),
  object-oriented design patterns, and processes for carrying out
  analysis and design. More advanced or recent developments may be
  included at the instructor's discretion. The course will balance an
  emphasis on design principles with an understanding of how to apply
  techniques and methods to create successful software
  systems.}{Prerequisite: CS 2150 (216) with a C- or higher.}{Y}

\course{CS 4330}{433}{Advanced Computer Architecture}{3
  credits}{Provides an overview of modern microprocessor design. The
  topics covered in the course will include the design of super-scalar
  processors and their memory systems, and the fundamentals of
  multi-core processor design.}{Prerequisite: CS 2150 (216) and CS/ECE
  3330 (333) with a C- or higher.}{Y}

\course{CS 4414}{414}{Operating Systems}{3 credits}{Analyzes process
  communication and synchronization; resource management; virtual
  memory management algorithms; file systems; and networking and
  distributed systems.}{Prerequisite: CS 2150 (216) and CS/ECE 3330
  (333) with grades of C- or higher.}{S}

\course{CS 4434}{434}{Fault-tolerant Computing}{3
  credits}{Investigates techniques for designing and analyzing
  dependable computer-based systems. Topics include fault models and
  effects, fault avoidance techniques, hardware redundancy, error
  detecting and correcting codes, time redundancy, software
  redundancy, combinatorial reliability modeling, Markov reliability
  modeling, availability modeling, maintainability, safety modeling,
  trade-off analysis, design for testability, and the testing of
  redundant digital systems. Cross-listed as ECE 434.}{Prerequisite:
  CS/ECE 3330 (333), APMA 2130 (213), and APMA 3100 (310), each with
  grades of C- or higher.}{SI}

\course{CS 4444}{444}{Introduction to Parallel Computing}{3
  credits}{Introduces the student to the basics of high-performance
  parallel computing and the national cyber-infrastructure. The course
  is targeted for both computer science students and students from
  other disciplines who want to learn how to significantly increase
  the performance of applications.}{Prerequisite: CS 2150 (216) and
  CS/ECE 3330 (333) with a C- or higher.}{Y}

\course{CS/ECE 4457}{457}{Computer Networks}{3 credits}{Intended as a
  first course in communication networks for upper-level undergraduate
  students. Topics include the design of modern communication
  networks; point-to-point and broadcast network solutions; advanced
  issues such as Gigabit networks; ATM networks; and real-time
  communications. Cross-listed as ECE 457.}{Prerequisite: CS/ECE 3330
  (333) with a grade of C- or higher.}{Y}

\course{CS 4458}{458}{Internet Engineering}{3 credits}{An advanced
  course on computer networks on the technologies and protocols of the
  Internet. Topics include the design principles of the Internet
  protocols, including TCP/IP, the Domain Name System, routing
  protocols, and network management protocols. A set of laboratory
  exercises covers aspects of traffic engineering in a wide-area
  network.}{Prerequisite: CS 4457 (457) with a grade of C- or
  higher.}{Y}

\course{CS 4501}{451 and 494}{Selected Topics in Computer Science}{1
  to 3 credits}{Content varies annually, depending on students needs
  and interests. Recent topics include the foundations of computation,
  artificial intelligence, database design, real-time systems,
  Internet engineering, wireless sensor networks, and electronic
  design automation.}{Prerequisite: Instructor permission.}{SI}

\course{CS 4610}{415}{Programming Languages}{3 credits}{Presents the
  fundamental concepts of programming language design and
  implementation. Emphasizes language paradigms and implementation
  issues. Develops working programs in languages representing
  different language par\-a\-digms. Many programs oriented toward language
  implementation issues.}{Prerequisite: CS 2150 (216) with a grade of
  C- or higher.}{Y}

\course{CS 4620}{471}{Compilers}{3 credits}{Provides an introduction
  to the field of compilers, which translate programs written in
  high-level languages to a form that can be executed. The course
  covers the theories and mechanisms of compilation tools. Students
  will learn the core ideas behind compilation and how to use software
  tools such as lex/flex, yacc/bison to build a compiler for a
  non-trivial programming language.}{Prerequisite: CS 3240 (340) and
  CS/ECE 3330 (333) with grades of C- or higher.}{Y}

\course{CS 4630}{425}{Defense Against the Dark Arts}{3
  credits}{Viruses, worms, and other malicious software are an
  ever-increasing threat to computer systems. There is an escalating
  battle between computer security specialists and the designers of
  malicious software. This course provides an essential understanding
  of the techniques used by both sides of the computer security
  battle.}{Prerequisite: CS 2150 (216) with a grade of C- or
  higher.}{Y}

\course{CS 4710}{416}{Artificial Intelligence}{3 credits}{Introduces
  artificial intelligence. Covers fundamental concepts and techniques
  and surveys selected application areas. Core material includes state
  space search, logic, and resolution theorem proving. Application
  areas may include expert systems, natural language understanding,
  planning, machine learning, or machine perception. Provides exposure
  to AI implementation methods, emphasizing programming in Common
  LISP.}{Prerequisite: CS 2150 (216) with a grade of C- or higher.}{Y}

\course{CS 4720}{N/A}{Web and Mobile Systems}{3 credits}{With advances
  in the Internet and World Wide Web technologies, research on the
  design, implementation and management of web-based information
  systems has become increasingly important. Thus looks at the
  systematic and disciplined creation of web-based software
  systems. Students will be expected to work in teams on projects
  involving mobile devices and web applications.}{Prerequisite: CS
  2150 (216) with a grade of C- or higher.}{Y}

\course{CS 4730}{N/A}{Computer Game Design}{3 credits}{This course
  will introduce students to the concepts and tools used in the
  development of modern 2-D and 3-D real-time interactive computer
  video games.  Advanced CS topics such as graphics, parallel
  processing, human-computer interaction, networking, artificial
  intelligence, and software engineering play a large role in industry
  in general.}{Prerequisite: CS 2150 (216) with a grade of C- or
  higher.}{Y}

\course{CS 4750}{462}{Database Systems}{3 credits}{Introduces the
  fundamental concepts for design and development of database
  systems. Emphasizes relational data model and conceptual schema
  design using ER model, practical issues in commercial database
  systems, database design using functional dependencies, and other
  data models. Develops a working relational database for a realistic
  application.}{Prerequisite: CS 2150 (216) with a grade of C- or
  higher.}{Y}

\course{CS 4753}{453}{Electronic Commerce Technologies}{3
  credits}{Focuses on the history of the Internet and electronic
  commerce on the web; case studies of success and failure;
  cryptographic techniques for privacy, security, and authentication;
  digital money; transaction processing; wired and wireless access
  technologies; Java; streaming multimedia; XML; Bluetooth. Defining,
  protecting, growing, and raising capital for an
  e-business.}{Prerequisite: CS 2150 (216) with a grade of C- or
  higher.}{Y}

\course{CS 4810}{445}{Introduction to Computer Graphics}{3
  credits}{Introduces the fundamentals of three-dimensional computer
  graphics: rendering, modeling, and animation. Students learn how to
  represent three-dimensional objects (modeling) and the movement of
  those objects over time (animation). Students learn and implement
  the standard rendering pipeline, defined as the stages of turning a
  three-dimensional model into a shaded, lit, texture-mapped
  two-dimensional image.}{Prerequisites: CS 2150 (216) with a grade of
  C-.}{Y}

%\course{CS 4830}{447}{Image Synthesis}{3 credits}{Provides a broad
%  overview of the theory and practice of rendering. Discusses classic
%  rendering algorithms, although most of the course focuses on either
%  fundamentals of image synthesis or current methods for physically
%  based rendering. The final project is a rendering
%  competition.}{Prerequisite: Grade of C- or higher in CS 4810 (445)
%  or equivalent working knowledge.}{Y}

\course{CS 4993}{493}{Independent Study}{1 to 3 credits}{In-depth
  study of a computer science or computer engineering problem by an
  individual student in close consultation with departmental
  faculty. The study is often either a thorough analysis of an
  abstract computer science problem or the design, implementation, and
  analysis of a computer system (software or hardware).}{Prerequisite:
  Instructor permission.}{S}

\course{CS 4998}{495}{Distinguished BA Majors Research}{3
  credits}{Required for Distinguished Majors completing the Bachelor
  of Arts degree in the College of Arts and Sciences. An introduction
  to computer science research and the writing of a Distinguished
  Majors thesis.}{Prerequisite: in the BA program and instructor
  permission.}{S}
 

\mysection{Degree Requirement Revisions}
\label{sec:degreerevisions}

Computer science is an evolving field, and our undergraduate
curriculum reflects this. The department sometimes makes changes to
the requirements for the bachelor's degree. Note that you are allowed
to graduate using ANY SINGLE set of requirements that were in effect
when you were a declared computer science major~-- thus, if the
requirements change, you are allowed to complete the degree using the
older version of the requirements. You cannot ``mix and match''
requirements from the different sets. For example, a student using the
fall 2004 rules below (no general electives) is not allowed to take
something other than ECE 435 (Computer Organization and Design) for
the computer architecture elective. 

Any changes to the requirements will occur after the spring semester
and before the following fall semester, unless the change is
considered minor. A minor change is something that does not in any way
restrict the degree requirements. Examples of minor changes would be
expanding the allowed courses for one of the elective types, or
clarifying what counts as a given elective. Note that unless the
change to the requirements directly affects the third semester
(i.e. the first semester of the second year), a student cannot choose
to graduate using a set of requirements that were in effect during his
or her first year at UVa but that were not in effect during his or her
second year, as they were not a declared computer science major during
their first year.

The requirement revisions below describe which major changes occurred
during the previous years, and what courses students must complete to
graduate using that set of requirements. Note that the older sets are
kept for historical reasons, even though there may not be any more
students who are eligible to graduate with those sets.

The current set of requirements, which this document reflects, became
effective in January 2010.

\subsection{Requirements revision from spring 2010}

In January of 2010, the elective structure was changed. Previously,
majors were required to take 3 HSS electives, 3 general education
electives, 3 technical electives, and 1 unrestricted elective. With
the change, these 10 elective courses are now split into 5 HSS
electives and 5 unrestricted electives. Students wishing to graduate
using the old rules (if you were a declared major prior to 2010)
should see the previous editions of this handbook for the description
of what constitutes general education electives and technical
electives. However, the new requirements are more general, and we
expect most students to graduate using these updated requirements. The
old versions of this handbook are available
online\myurl{http://www.cs.virginia.edu/bscs/}.

\subsection{Requirements revision from fall 2009}

In addition to the course numbering change, the change in the
requirements was that the computer architecture elective was replaced
with an additional CS elective, to bring the total number of required
CS electives to 5. The previous computer architecture requirement had
the students take one class from a set of 3: CS 4444 (444)
(Introduction to Parallel Computing), CS 4330 (433) (Advanced Computer
Architecture) or ECE 4435 (435) (Computer Organization and
Design). Since all of those three courses count as CS electives,
students who have already fulfilled this old requirement will still
fulfill the CS elective that replaced it.

Focal paths were also added to the undergraduate handbook, although
they do not change the major requirements.


\clearpage
\pagestyle{empty}

\vspace*{2in}

\begin{center}
\parbox{2.5in}{{\em Enlighten the people generally, and tyranny and
    oppression of body and mind will vanish like evil spirits at the
    dawn of day … the diffusion of knowledge among the people is to be
    the instrument by which it is to be effected.\linebreak\linebreak
    --~Thomas Jefferson, 1816}}
\end{center}


\clearpage

\vspace*{2.75in}

\begin{center}
Department of Computer Science \linebreak
School of Engineering and Applied Science \linebreak
The University {\em of} Virginia \linebreak
151 Engineer’s Way \linebreak
P.O. Box 400740 \linebreak
Charlottesville, Virginia 22904-4740 \linebreak
434.982.2200 \linebreak
\url{http://www.cs.virginia.edu} \linebreak
\end{center}

\begin{figure}[h!]
\begin{center}
\epsfig{figure=images/uva-seal.png,width=2in}
\end{center}
\end{figure}

\end{document}

\clearpage

\mychapter{Todos}{Todos}

\begin{numlist}
\item Convert final version to web page
\item fix all three diagrams
\item a nice, shiny full-color cover
\item insert the grad handbook link once I get it
\item Rename the 'Common' section
\item When to remove 3 digit course numbers?  Probably next handbook
\item Put independent study in the course diagrams; and 4998 in the BA
  CS one.
\item Look though course-offerings.xls (an attachment Tom sent me on
  5-6-11) to make sure I list all the courses that we have taught
  recently.
\item Have Dave check the ``CLAS school requirements'' under the Venn
  diagram
\item Fix the links to the main BA, BS, CpE, and minor pages
\end{numlist}


\noindent Assignments

\begin{itemlist}
\item Jim: re-writing sections 3.1 and 4.1
\item Jim: re-writing the new last paragraph of 1.1 (the bureau of labor statistics one)
\item Jim will edit the theory one
\item Mark will write section 2.1
\item Mark will edit the networks and web focal paths
%\item Tom will create the SE focal path (DONE)
%\item Tom is re-writing and/or combining old sections 1.2 and 1.2.1 (DONE)
\item Aaron will edit the game design one 
\end{itemlist}


\end{document}
