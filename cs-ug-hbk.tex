\documentclass[10pt,letter]{book}

\usepackage{hyperref}
\usepackage{palatino}
\usepackage{epsfig}
\usepackage{fancyhdr}

% the following pacakge was added to use the strikeout, but it caused
% all emphasized text (i.e., \em) to be underlined instead.
%\usepackage{ulem} % for the CpE embedded requirement strikeout

\usepackage[margin=0.5in, paperwidth=5.5in, paperheight=8.5in]{geometry}
\setlength{\paperheight}{8.5in}
\setlength{\paperwidth}{5.5in}

%\usepackage[margin=0.75in, paperwidth=8.5in, paperheight=11in]{geometry}
%\setlength{\paperheight}{11in}
%\setlength{\paperwidth}{8.5in}

\setlength{\pdfpagewidth}{\paperwidth}
\setlength{\pdfpageheight}{\paperheight}

\newenvironment{itemlist}{
\begin{itemize}
\setlength{\itemsep}{0pt}
\setlength{\parskip}{0pt}}
{\end{itemize}}

\newenvironment{numlist}{
\begin{enumerate}
\setlength{\itemsep}{0pt}
\setlength{\parskip}{0pt}}
{\end{enumerate}}

\newcommand{\und}[1]{\underline{\smash{#1}}}

\newcommand{\csminoradvisoremail}{minoradvisor@cs.virginia.edu}
\newcommand{\bacsdirectoremail}{\mbox{bacsdirector@cs.virginia.edu}} % the \mbox prevents hypenation
\newcommand{\bscsdirectoremail}{\mbox{bscsdirector@cs.virginia.edu}} % the \mbox prevents hypenation
\newcommand{\acmadvisor}{Aaron Bloomfield}
\newcommand{\acmadvisoremail}{aaron@virginia.edu}
\newcommand{\handbookmaintainer}{Aaron Bloomfield}
\newcommand{\handbookmaintaineremail}{aaron@virginia.edu}
\newcommand{\handbookmaintainerpronoun}{him} % him or her

\newcommand{\course}[5]{\noindent {\bf #1~-- #2} (#3): #4 #5\vspace{0.1in}}

\newcommand{\mychapter}[2]{\chapter{#1}\renewcommand{\leftmark}{\textsc{#2}}}
\newcommand{\mysection}[1]{\section{#1}\renewcommand{\rightmark}{#1}}

\newcommand{\myurl}[1]{\footnote{\scriptsize\url{#1}}}

% The next two adapted from
% http://anthony.liekens.net/index.php/LaTeX/MultipleFootnoteReferences
\newcommand{\myurlremember}[2]{\footnote{\scriptsize\url{#2}}\newcounter{#1}\setcounter{#1}{\value{footnote}}}
\newcommand{\myurlrecall}[1]{\footnotemark[\value{#1}]} 

\raggedbottom
\setlength{\footskip}{0pt}

\begin{document}
\pagestyle{empty}

\vspace*{0.5in}

\begin{figure}[h!]
\begin{center}
\epsfig{figure=images/Engineering-4-color.pdf,width=3.5in}
\end{center}
\end{figure}

\vspace{0.25in}

\begin{center}
{\huge Department {\em of} Computer Science}

\vspace{12pt}

{\huge Undergraduate Handbook}

\vspace{1in}

{\Large Bachelor of Science in Computer Science}

{\Large Bachelor of Arts in Computer Science}

{\Large Bachelor of Science in Computer Engineering}

{\Large Minor in Computer Science}

\vspace{1in}

{\large http://www.cs.virginia.edu}

{\large Handbook version for the 2013--2014 academic year}
\end{center}


\clearpage

\vspace*{1.5in}

\begin{center}
\parbox{3in} {The last updates to this undergraduate handbook were
  made in the summer of 2013.  This version is valid for the
  2013--2014 academic year.
\newline

Any updates, both errata and addendums, to this version of the
handbook will be listed on the individual degree web pages to which
the errata or addendum applies.
\newline

The BS CS web site is at
\url{http://www.cs.virginia.edu/acad/bscs/}
\newline

The BA CS web site is at
\url{http://www.cs.virginia.edu/acad/ba/}
\newline

The BS CpE web site is at
\url{http://www.cpe.virginia.edu/ugrads/}.
\newline

This handbook was written by, and is currently maintained by,
\handbookmaintainer\ (\handbookmaintaineremail).  Please send any
errata to \handbookmaintainerpronoun.

}
\end{center}

\clearpage
\pagestyle{fancy}
\setcounter{page}{1}
\pagenumbering{roman}

\tableofcontents

\cleardoublepage
\setcounter{page}{1}
\pagenumbering{arabic}

\mychapter{Introduction}{Introduction}

\mysection{Introduction}

Through the development of sophisticated computer systems, processors,
and embedded applications, computer scientists have the opportunity to
change society in ways unimagined several years ago. Our goal is the
education and training of a diverse body of students who can lead this
information technology revolution. To this end, the computing programs
orient students toward the pragmatic aspects of computing and provides
the learning and practices to make them proficient computing
professionals. Computational thinking is rooted in solid mathematics
and science, and grounding in these fundamentals is essential. Our
laboratory environment exposes students to many commercial software
tools and systems, and introduces modern software development
techniques.  In the context of the practice of computing, this early
grounding forms the basis for an education that prepares students for
a computing career.

%With funding from the National Science Foundation, the Department of
%Computer Science has designed and developed a curriculum focused on
%the practice of computing, yet grounded in the mathematical and
%scientific fundamentals of computer science. The curriculum is
%structured around the introduction of modern software development
%techniques in the very beginning courses, and is supported by a set of
%``closed laboratories''.

%In order to provide an environment appropriate to our courses, the
%department has several laboratories with hundreds of workstations.
%These machines have high-resolution graphics and are connected to
%large file handlers, as well as to the University network.  The lab
%courses expose students to many commercial software tools and systems,
%and introduce modern software development techniques via
%object-oriented design and implementation.

%The Department of Computer Science co-offers, with the Department of
%Electrical and Computer Engineering, a degree in Computer Engineering. 

Students have many opportunities to participate in cutting-edge research
with department faculty members. From the senior thesis research
project to independent study, students can pursue research in any
conceivable area. Our former students are enrolled in top graduate
programs across the country. Our undergraduates have won many research
awards, including multiple CRA (Computing Research Association)
awards in recent academic years. In fact, of all US
institutions, UVa is third in overall CRA research awards won.

All graduates of our three computing programs will have the knowledge
and skills to be practitioners and innovators in computing and other
fields.  They will be able to apply computational thinking in the
analysis, design and implementation of computing solutions, whether
working alone or as part of a team. The knowledge and skills acquired
from our degree programs will give students the ability to make
contributions after graduation in their own field as well as to
society at large.

A recent Bureau of Labor Statistics Occupational Outlook Handbook
states that ``very favorable opportunities'' (more numerous job
openings compared to job seekers) can be expected for college
graduates with at least a bachelor's degree in computer
engineering. It also projects an employment increase of over 38\% by
2016 for occupations available to graduates with a bachelor's degree
in computer engineering\myurl{http://www.bls.gov/oco}.

\subsection{Diversity Statement}

The members of the department envision an environment where a
diversity of capable, inspired individuals congregate, interact and
collaborate, to learn and advance knowledge, without barriers. We
embrace this vision because:

\begin{itemlist}
\item We wish to be leaders and role models in reaping and sharing the
 benefits of diversity.
\item We seek to improve the intellectual environment and creative
 potential of our department.
\item We expect to produce happier, more capable and more broadly
 educated computer science graduates.
\item We wish to contribute to social justice and economic well-being
 for all citizens.
\end{itemlist}

\mysection{Degrees Offered}

The Department of Computer Science offers three computing degrees, as well as a minor.

\begin{itemlist}
\item Bachelor of Science (BS in CS) in Computer Science, available to
  students in the School of Engineering and Applied Sciences (SEAS).
\item Bachelor of Arts in Computer Science (BA in CS), available to
  students in the College of Liberal Arts and Sciences (CLAS).
\item Bachelor of Science in Computer Engineering (BS in CpE),
  available to students in the School of Engineering and Applied
  Sciences (SEAS). This degree is shared with the Department of
  Electrical and Computer Engineering.
\item Minor in Computer Science.  Note that the minor is restricted in
  whom can apply for it; see section~\ref{minorapplicationprocess}
  (page~\pageref{minorapplicationprocess}) for details.
\end{itemlist}

Details of the degrees are provided later in this document, but in
this section we explain the differences between computer science and
computer engineering. This explanation is adopted from the ACM and
IEEE's Computing Curricula 2005: The Overview
Report\myurl{http://www.acm.org/education/curricula-recommendations}. We
also give a high-level overview of the difference between our BS and
BA degrees in computer science.

\subsection{What is Computer Science?}

Computer science spans a wide range, from its theoretical and
algorithmic foundations to cutting-edge developments in graphics,
intelligent systems, cybersecurity, and other exciting areas. We can
think of the work of computer scientists as falling into three
categories.

\begin{itemlist}
\item They design and implement software. Computer scientists take on
  challenging programming jobs. They also supervise other programmers,
  keeping them aware of new approaches.
\item They devise new ways to use computers. Progress in the CS areas
  of networking, database, and human-computer-interface enabled the
  development of the World Wide Web. Now CS researchers are working
  with scientists from other fields to develop control physical
  sensors and devices, to use databases to create new knowledge, and
  to use computers to help doctors solve complex problems in medical
  care.
\item They develop effective ways to solve computing problems. For
  example, computer scientists develop the best possible ways to store
  information in databases, send data over networks, and display
  complex images. Their theoretical background allows them to
  determine the best performance possible, and their study of
  algorithms helps them to develop new approaches that provide better
  performance.
\end{itemlist}

Computer science spans the range from theory through
programming. While some universities offer computing degree programs
that are more specialized (such as software engineering,
bioinformatics, etc.), a degree in computer science offers a
comprehensive foundation that permits graduates to adapt to new
technologies and new ideas.

\subsection{Comparison of the BA \& BS Computer Science
  Degrees}

At the University of Virginia, we offer two different computer science
degrees:

\begin{itemlist}
\item The Bachelor of Science (BS) degree, through the School of
  Engineering and Applied Sciences (SEAS), and
\item The Interdisciplinary Major in Computer Science, a Bachelor of
  Arts (BA) degree, through the College of Liberal Arts and Sciences
  (CLAS).
\end{itemlist}

The following gives a high-level comparison of these two degrees.

The BS in Computer Science degree program includes the set of core
courses required of every other engineering degree in SEAS. These
include an introduction to engineering, physics, chemistry, calculus,
courses focused on the engineer's role in society, and at least five
courses in the humanities or social sciences. Like other engineering
majors, all students in our BS program complete a year-long project
leading to a senior thesis in their fourth year. Students in the BS
program can minor in another engineering discipline or applied
math. It is also possible to minor in a subject from the College of
Arts and Sciences (but it's more difficult to have a second major in a
College subject). Students in the BS program must complete at least 46
credits of computer science courses. The Bachelor of Science in
Computer Science is accredited by the Computing Accreditation
Commission of ABET\myurl{http://www.abet.org}.

The BA in Computer Science degree program includes the same general
requirements (known as core and competency requirements) as all other
liberal arts and science degrees in CLAS. These include courses in
foreign language, writing, historical studies, social science,
humanities, and non-western perspectives. These general requirements
also include natural science and mathematics, but fewer courses than
are required for the BS CS in engineering. Students in the BA program
are in a good position to major or minor in another subject in
CLAS. Students with a GPA of 3.4 or better may apply to the
Distinguished Majors Program, in which students complete a thesis
based on two semesters of empirical or theoretical research. Students
in the BA program must complete at least 27 credits of computer
science courses along with 12 additional credits of "integration
electives", which are computing-related courses taught by another
department other than the CS department. Students in the BA program
have the option of taking a version of the first two computing courses
that differ from those taken by the BS students, but otherwise
students from both degree programs share the same CS courses.

Graduates of both programs have been accepted to the best graduate
programs, have received job offers from leading companies, etc. A few
employers have shown a preference for graduates from one program or
the other, but in general both degrees prepare students for excellent
opportunities after graduation.

Students who apply to the University of Virginia must choose to apply
for admission to either SEAS (the engineering school) or CLAS (the
College of Liberal Arts and Sciences). It is possible to transfer from
one unit to the other after admission, and since we offer degrees in
both units a student can major in computer science in either.

\subsection{What is Computer Engineering?}

Computer engineering is concerned with the design and construction of
computers and computer-based systems. It involves the study of
hardware, software, communications, and the interaction among
them. Its curriculum focuses on the theories, principles, and
practices of traditional electrical engineering and mathematics and
applies them to the problems of designing computers and computer-based
devices.

Computer engineering students study the design of digital hardware
systems including communications systems, computers, and devices that
contain computers. They study software development, focusing on
software for digital devices and their interfaces with users and other
devices. At the University of Virginia, the CpE degree has a balanced
emphasis on hardware and software.

At the University of Virginia, computer engineering degrees are
jointly designed and administered by the Department of Computer
Science and the Department of Electrical and Computer Engineering. The
degree program is composed of courses from both departments. 

\subsection{ABET accreditation}

The Bachelor of Science in Computer Science is accredited by the
Computing Accreditation Commission of
ABET\myurlremember{abet}{http://www.abet.org}.  The Bachelor of
Science in Computer Engineering is accredited by the Engineering
Accreditation Commission of ABET\myurlrecall{abet}.


%\clearpage
%\mysection{Venn Diagram}
%\begin{figure}[h!]
%\epsfig{figure=flowcharts/venn-diagram.png,width=4.5in}
%\end{figure}

\clearpage

\mysection{Major Course Requirements Comparison}

\begin{figure}[h!]
\begin{center}
\epsfig{figure=diagrams/venn-diagram.pdf,width=3.9in}
\end{center}
\end{figure}

\noindent
\begin{tabular}{p{2in}p{2.25in}}

The SEAS school requirements consist of:
\begin{itemlist}
\item APMA 1110 \& 2120
\item CHEM 1610 \& 1611
\item ENGR 1620 \& ENGR 1621
\item PHYS 1425 \& 1429
\item PHYS 2415 \& 2419
\item Science elective
\item STS 1500
\item STS 2xxx/3xxx elective
\item STS 4500 \& 4600
\end{itemlist}

&

The CLAS school requirements consist of:
\begin{itemlist}
\item First \& second writing requirements
\item Foreign language requirement
\item 6 credits of social sciences
\item 6 credits of humanities
\item 3 credits of historical studies
\item 3 credits of non-western perspectives
\item 12 credits of natural science and math
\end{itemlist}

\end{tabular}

\noindent A ``CS 1 class'' is CS 1110, CS 1111, or CS 1112.  CLAS
majors can take CS 1120, provided they have Java experience.
Placement is available; see sections~\ref{applacement}
(page~\pageref{applacement}) \& \ref{101placement}
(page~\pageref{101placement}).

\clearpage
\mychapter{Bachelor of Science in Computer Science}{BS CS Degree}
\label{bscschapter}

\mysection{Introduction}

% this intro drafted by Mark, and sent via e-mail on 9-6-11

The Bachelor of Science degree in Computer Science is a wide-ranging
program, encompassing both the theoretical and the practical.  This
program builds upon the engineering and mathematical principles
introduced in the Engineering School's core curriculum.  Our students
are then taught to apply computing to the world around them by building
faster, smaller, and more secure software systems, exploring emerging
technologies, and working on real-world problems.  Our courses focus
on teaching students how to recognize computational challenges, create
elegant and efficient algorithms, and then use rigorous development
methodologies to build systems that can solve pressing
problems. Graduates of the BS program find successful careers with
traditional software companies, government agencies, consulting firms,
academia, and companies in other fields that have software needs.
Computing professions are often ranked near the top in ``Best Job''
lists put together by news organizations for job availability, pay,
and satisfaction.

Course work in the BS CS program starts with several courses that
introduce the basic principles of software creation, from learning
programming languages to advanced development techniques.  Once
students have mastered the basics, the bulk of our program opens up,
offering electives in several exciting fields, including networking,
security, game design, web programming, e-commerce, parallel
computing, and much more.  Students have the opportunity to take
several electives each semester, as our department offers more
electives than the other departments in the Engineering School. 

\mysection{Application Process}
\label{bscsapplicationprocess}

The application requirements in this section are the same as for the
Computer Engineering degree.

The Department of Computer Science is experiencing tremendous student
interest in our degree programs. Our goal is to accommodate as many
undergraduate majors as possible. However, because of current
resources, there is an enrollment cap on the total number of
majors in the BS in Computer Science (BS CS) and the BS in Computer
Engineering (BS CpE) degree programs.

{\bf First-year students:} All first-year Engineering students choose
their major in the Spring semester of their first year. At that time,
students seeking either the BS CS or BS CpE major will submit their
application information as part of the normal first-year SEAS major
declaration process, which the SEAS Dean's office manages. Since the
cap is on the {\em combined} number of BS CS and BS CpE majors, a
student's chance of acceptance will not be affected by choosing one
over the other. All applicants will be notified of admission decisions
by early summer.

{\bf Engineering students changing majors or seeking a second major:}
Declared BS CS or BS CpE majors can change from one major to the other
at any time, and they can declare the other major as their second
major without applying. However, other engineering students who want
to declare the BS CS or BS CpE as a second major, and students who want
to change their first major to either of these, must apply. The
available space given our target caps for each class year may be a
factor. Application deadlines for such requests will be October 15 and
February 15. The application form can be found
online\myurl{http://www.cs.virginia.edu/acad/bscs/}.

All third and fourth year students who want to declare the BS CS major
as a first or second major must have already completed CS 2190 or will
need to complete CS 2190 by the end of their third year.

At this time, SEAS students are not eligible to apply for the BA CS
(i.e., the College degree) as a second major.

{\bf Intra-University Transfer Students:} SEAS accepts transfer
app\-li\-ca\-tions for non-SEAS UVa students, currently once a year. The
application deadline and process is described on this SEAS
web page\myurl{http://www.seas.virginia.edu/advising/transferseas.php}.
Qualified applicants who want to transfer into SEAS to become BS CS or
BS CpE majors will be considered on a space-available basis given our
target caps for each class year. Such students should follow the SEAS
application process, and must contact the Computer Science department
contact person or the Electrical and Computer Engineering department
contact person listed in the SEAS web page before applying.

{\bf Transfer students from outside the University:} Students
transferring into the University from other institutions must apply to
the department to be allowed to declare the BS CS or BS CpE major.
Qualified applicants will be considered on a space-available basis,
given our target caps for each class year. Applications will be
considered the summer before a transfer student begins classes, and
the application process will be discussed during the summer
orientation session. If an incoming transfer does not attend summer
orientation, they must meet with a department advisor before classes
begin to discuss applying.

Due to prerequisite dependencies, it is difficult for rising third year
students who have not completed CS 2110 and CS 2102 to complete the
BS CS in the 4 remaining semesters. It is important that students
transferring to the University as third year students complete the equivalent
of these courses before coming to UVa. In exceptional cases, students
in this situation may apply for the major, but the ability to complete
the degree in a timely fashion is considered in determining if
you are accepted into the degree program.


\mysection{Curriculum}

The requirements for the computer science degree consist of a number
of required courses, as well as a series of elective choices for the
student to make.  A table of all the requirements is shown in
figure~\ref{fig:bscsreqs} (page~\pageref{fig:bscsreqs}).

\subsection{Elective Information}
\label{sec:electiveinfo}

The numbers in the list below correspond to the footnote numbers from
the sample course schedule shown in section~\ref{sec:bscsschedule}
(page~\pageref{sec:bscsschedule}).

\begin{enumerate}

\item Science elective (1 required): Students must choose one of BIOL
  2010 (Introduction to Biology: Cell Biology and Genetics),
  BIOL 2020 (Introduction to Biology: Organismal and
  Evolutionary Biology), CHEM 1620 (Introductory Chemistry for
  Engineers), ECE 2066 (Science of Information), ENGR 2500
  (Introduction to Nano\-science and Technology), MSE 2090
  (Introduction to the Science and Engineering of Materials), or PHYS
  2620 (Introductory Physics IV: Quantum Physics). Additional
  courses in this list can count as an unrestricted elective.

\item HSS electives (5 required): Studies in the humanities and social
  sciences serve not only to meet the objectives of a broad education,
  but also to meet the objectives of the engineering profession. Such
  course work must meet the generally accepted definitions that the
  humanities are the branches of knowledge concerned with humankind
  and its culture, while the social sciences are the studies of
  society. See the full list of allowed courses in the SEAS
  Undergraduate Handbook. This list can be found
  online\myurl{http://www.seas.virginia.edu/advising/undergradhandbook.php\#hss}. Note
  that there are a number of courses that do not count as HSS
  electives, but would count as an unrestricted elective. See the URL
  for details.

\item Unrestricted elective (5 required): Any graded course in the
  University, with a few exceptions. From the SEAS Undergraduate
  Student
  Handbook\myurl{http://www.seas.virginia.edu/advising/undergradhandbook.php}:
  ``Unrestricted Electives may be chosen from any graded course in the
  University except mathematics courses below MATH 1310, including
  STAT 1100 and 1120, and courses that substantially duplicate any
  others offered for the degree, including PHYS 2010, PHYS 2020, CS
  1010, CS 1020, or any introductory programming course. Students in
  doubt as to what is acceptable to satisfy a degree requirement
  should obtain the approval of their adviser and the dean's office,
  Thornton Hall, Room A122. APMA 1090 counts as a three credit
  unrestricted elective for students.''  Note that Band classes (such
  as marching band) and ROTC classes can count for the unrestricted
  elective.

\item APMA elective (2 required): Must choose two from: APMA 2130
  (Ordinary Differential Equations), APMA 3080 (Linear
  Algebra) or APMA 3120 (Statistics). Note that APMA 3100
  (Probability) is a required course in addition to the two APMA
  electives.

\item CS electives (5 required): Any 3 credit CS class at the 3000
  level or above. A course that is fulfilling another requirement
  cannot count as a CS elective. If you take more than five CS
  electives, you can count additional CS elective course(s) as
  unrestricted electives. ECE 4435 (Computer Architecture \& Design)
  and ECE 4440 (Embedded Systems Design) also count as a CS electives
  (this is not the case for CpE majors, as they are both required
  courses for CpE). Note, however, that those two courses only count
  as {\em one} CS elective each; those 9 credits (each is worth 4.5
  credits) do not count as 3 CS electives.  And in order for them to
  be counted, a SIS exception must be entered~-- see
  section~\ref{sec:sisece4435issue}
  (page~\pageref{sec:sisece4435issue}) for details.  CS 4993
  (Independent Study) can be used at the most once for a CS elective
  (i.e. no more than 3 credits); additional CS 4993 credits can be
  used as unrestricted electives. Note that for a class that does not
  meet these requirements to count as a CS elective requires approval
  by the CS undergraduate curriculum committee (NOT by the student's
  academic advisor); this process can be initiated by emailing the BS
  CS director at \bscsdirectoremail. Due to substantial overlap, one
  cannot get credit for both ECE 4435 and CS 4330. Thus, if a student
  takes both of those classes, the other one can ONLY count as a
  unrestricted elective.

\item STS 2xxx/3xxx elective (1 required): Any STS course at the
  2000-level or 3000-level.
\end{enumerate}
 
Note: classes that receive no grade (including classes that are
audited) do not count toward your degree requirements.

\subsection{Capstone Sequence}
\label{capstone-section}

All SEAS students must complete a senior thesis, which is encapsulated
in the STS 4500 and STS 4600 courses.  In addition to the STS courses,
BS CS students must complete one or two CS courses to fulfill the
computer science capstone sequence requirement.  There are two
``tracks'' to complete the capstone, described below, and students may
choose either track.

Formally, students must complete 3 credits of one of the two capstone
courses, depending on which track they choose to
fulfill: either CS 4971 (Capstone Practicum II) for the Capstone
Practicum track, or CS 4980 (Research Capstone) for the Research
Capstone track.  But note that CS 4970 (Capstone Practicum I), which
is a CS elective, is a strict pre-requisite to CS 4971!  Also note
that STS 4500 and STS 4600 must {\bf still} be taken; the CS courses
are in addition to, not instead of, the STS courses.

\subsubsection{Research Capstone Track}

This track is intended for students who are interested in performing
an independent project, either a research-based project or an
implementation-based project.  The student must seek out a faculty
member, who will agree to advise the project.  The requirements of the
project, workload, etc., are to be agreed upon by the student and
advisor.  Students will receive 3 credits for CS 4780 (Research
Capstone), which is what formally fulfills the capstone requirement
for the degree.  Faculty advisors may decide to assign the three
credits in a single semester, or spread the three credits across
multiple semesters; as long as three credits are eventually earned,
then the requirement will be fulfilled.  Group projects are up to the
discretion of the advisor, but are certainly permissible.  Large
projects may receive additional credit through CS 4993 (Independent
Study), but this is solely up to the advisor.  Note that there is a
maximum of 3 credits (1 course) of CS electives that that CS 4993 may
count toward; any additional credits count toward the unrestricted
elective requirement.

This track is essentially how the senior thesis technical requirement
worked previously, except that students now receive three credits for
the technical work performed.

\subsubsection{Capstone Practicum Track}

This track is intended for any students who are not planning on
performing an independent project.  Students under this track will
register for CS 4970 (Capstone Practicum I) in the fall of their 4th
year, and CS 4971 (Capstone Practicum II) in the spring of their 4th
year~-- specifically, those classes are to be taken concurrently with
STS 4500 and STS 4600, respectively.  Both of these CS courses are 3
credits.

These two courses form a year-long project implementation course.
Students will be grouped into teams, will have requirements, real
customers to interact with, real deadlines, and will need to complete
real deliverables.  While the domain of the course is up to the
instructor, the current implementation of the courses is a Service
Learning Practicum\myurl{http://www.cs.virginia.edu/~asb/slp/}.

Note that only CS 4971 counts toward the capstone requirement; CS 4970
is a CS elective.  However, CS 4970 is a {\bf strict} pre-requisite to
CS 4971.  In particular, because the projects are year-long
group-based projects, there will be absolutely no allowances for
students to join the sequence for just CS 4971 in the spring semester.


\subsubsection{Double Majors and the Capstone}

A SEAS student who is double-majoring with the BS CS and another SEAS
degree must complete the capstone or major design experience (MDE)
course requirements in their non-CS major {\bf in addition to} the
capstone sequence in the computer science major.  This means that they
will still have to take either CS 4980 (Research Capstone) or the
practicum capstone sequence (CS 4970 and CS 4971). Students can
negotiate with their capstone/MDE instructors and their STS instructor
which degree program's work will be used to satisfy the STS thesis
portfolio requirements. However, the student should be aware that the
capstone/MDE instructors in both departments will almost certainly
require some kind of documentation of the technical work done for that
program's requirement.



\subsection{Degree Requirements Checklist}

The degree requirement checklist is shown in figure~\ref{fig:bscsreqs}
(page~\pageref{fig:bscsreqs}), and is available
online\myurl{http://www.cs.virginia.edu/acad/bscs/} in a full
(letter-sized) page format.

\begin{figure}
\label{fig:bscsreqs}
\small 
\begin{tabular}{|l|l|l|l|} \hline
\bf Required Computing \& Math courses & \bf Grade & \bf Semester &
\bf Comments \\ \hline \hline
CS 1110: Intro. to Computer Science & & & \\ \hline
CS 2110: Software Development Methods & & & \\ \hline
CS 2102: Discrete Mathematics I & & & \\ \hline
CS 2150: Program \& Data Representation & & & \\ \hline
CS/ECE 2330: Digital Logic & & & \\ \hline
CS 2190: CS Seminar I & & & \\ \hline
CS 3102: Theory of Computation & & & \\ \hline
CS/ECE 3330: Computer Architecture & & & \\ \hline
CS 3240: Advanced SW Devel. Tech. & & & \\ \hline
CS 4414: Operating Systems & & & \\ \hline
CS 4102: Analysis of Algorithms & & & \\ \hline
Capstone course (circle: CS 4971 or 4980) & & & \\ \hline
APMA 3100: Probability & & & \\ \hline
APMA 2130 or 3080 or 3120 (circle one) & & & \\ \hline
APMA 2130 or 3080 or 3120 (circle one) & & & \\ \hline
\end{tabular}

\noindent\begin{tabular}{@{}ll}
\noindent\begin{tabular}{@{}l}
\\
SEAS required courses \\
\begin{tabular}{|l|l|l|}\hline
\bf Course & \bf Grade & \bf Semester \\ \hline \hline
APMA 1110 & & \\ \hline
APMA 2120 & & \\ \hline
CHEM 1610 & & \\ \hline
CHEM 1611 & & \\ \hline
ENGR 1620 & & \\ \hline
ENGR 1621 & & \\ \hline
PHYS 1425 & & \\ \hline
PHYS 1429 & & \\ \hline
PHYS 2415 & & \\ \hline
PHYS 2419 & & \\ \hline
\end{tabular} \\
\\
Science elective \\
\begin{tabular}{|l|l|l|} \hline
\bf Course & \bf Grade & \bf Semester \\ \hline \hline
\hspace{0.7in} & & \\ \hline
\end{tabular}
\end{tabular}

&

\begin{tabular}{@{}l}
\\
STS courses \\
\begin{tabular}{|l|l|l|} \hline
\bf Course & \bf Grade & \bf Semester \\ \hline \hline
STS 1500 & & \\ \hline
STS 2xxx/3xxx & & \\ \hline
STS 4500 & & \\ \hline
STS 4600 & & \\ \hline
\end{tabular} \\
\vspace{0.15in} \\
CS Electives (5) \\
\begin{tabular}{|l|l|l|l|} \hline
& \bf Course & \bf Grade & \bf Semester \\ \hline \hline
1) & \hspace{0.55in} & & \\ \hline
2) & & & \\ \hline
3) & & & \\ \hline
4) & & & \\ \hline
5) & & & \\ \hline
\end{tabular}
\end{tabular}
\\
& \\
HSS electives (5) & Unrestricted electives (5) \\
\begin{tabular}{|l|l|l|l|} \hline
& \bf Course & \bf Grade & \bf Semester \\ \hline \hline
1) & \hspace{0.45in} & & \\ \hline
2) & & & \\ \hline
3) & & & \\ \hline
4) & & & \\ \hline
5) & & & \\ \hline
\end{tabular}
& 
\begin{tabular}{|l|l|l|l|} \hline
& \bf Course & \bf Grade & \bf Semester \\ \hline \hline
1) & \hspace{0.575in} & & \\ \hline
2) & & & \\ \hline
3) & & & \\ \hline
4) & & & \\ \hline
5) & & & \\ \hline
\end{tabular}
\\
\end{tabular}
\caption{BS CS Degree Requirements Checklist}

\end{figure}

\normalsize

\subsection{Sample BS CS Course Schedule}
\label{sec:bscsschedule}

Below is the recommended course of study for the bachelor's degree. If
one has already completed some of these classes (through AP credit,
for example), then your course of study would deviate from what is
shown below~-- consult your academic advisor for details.

There are a total of six types of electives that the student can
choose from. These electives are indicated by the footnotes below, and
are described in detail in section~\ref{sec:electiveinfo}
(page~\pageref{sec:electiveinfo}). Note that some of these
requirements are for all SEAS students, while others are required for
the CS bachelor's degree. Please be aware of when the classes are
offered! Some are only offered once per year, or in a particular
semester. See section~\ref{sec:coursedesc}
(page~\pageref{sec:coursedesc}) for details as to when courses are
offered.

The recommended schedule shown below has changed slightly each year as
the degree requirements have evolved. As discussed in the Degree
Requirement Revisions (section~\ref{sec:degreerevisions},
page~\pageref{sec:degreerevisions}), a student can graduate using any
set of requirements that were in effect when they became a declared
computer science major. Thus, as long as all the major requirements
are met, students can follow a previous version of the recommended
course schedule.

Academic requirements are managed by SIS (UVa's Student Information
System), which is where your individual set of requirements can be found.
%A sample of the BS CS requirement listing can be found
%online\myurl{http://www.cs.virginia.edu/bscs/bscs-reqs-in-sis.pdf}.
You may also want to see the FAQ question about how HSS requirements
list in the SIS report (section~\ref{sec:sishssissue},
page~\pageref{sec:sishssissue}).

\vspace{0.15in}

\noindent \begin{tabular}{p{1.25in}p{2.5in}c}
\und{First semester} & & \und{15} \\
APMA 1110 & Single Variable Calculus & 4 \\
CHEM 1610 \& 1611 & Chemistry for Engineers \& Lab & 4 \\
%CHEM 1611 & Chemistry Lab & 1 \\
ENGR 1620 & Introduction to Engineering & 3 \\
ENGR 1621 & Introduction to Engineering Lab & 1 \\
STS 1500 & Engineering, Technology \& Society & 3 \\
\\
\end{tabular}

\noindent \begin{tabular}{p{1.25in}p{2.5in}c}
\und{Second semester} & & \und{17} \\
APMA 2120 & Multivariate Calculus & 4 \\
PHYS 1425 \& 1429 & Physics I \& Lab & 4 \\
%PHYS 1429 & Physics I Workshop & 1 \\
CS 1110 & Intro to Computer Science & 3 \\
SCI & Science elective$^1$ & 3 \\
HSS/UE & HSS or unrestricted elective$^{2,3}$ & 3 \\
\\
\end{tabular}

\noindent \begin{tabular}{p{1.25in}p{2.5in}c}
\und{Third semester} & & \und{16} \\
APMA & APMA elective$^4$ or APMA 3100 & 3 \\
CS 2110 & Software Development Methods & 3 \\
CS 2102 & Discrete Mathematics & 3 \\
PHYS 2415 \& 2419 & General Physics II \& Lab & 4 \\
%PHYS 2419 & General Physics Lab I & 1 \\ 
HSS/UE & HSS or unrestricted elective$^{2,3}$ & 3 \\
\\
\end{tabular}

\noindent \begin{tabular}{p{1.25in}p{2.5in}c}
\und{Fourth semester} & & \und{16} \\
CS 2150 & Program and Data Representation & 3 \\
CS/ECE 2330 & Digital Logic Design & 3 \\
CS 3102 & Theory of Computation & 3 \\
CS 2190 & CS Seminar & 1 \\
STS STS & 2xxx/3xxx elective$^6$ & 3 \\
HSS/UE & HSS or unrestricted elective$^{2,3}$ & 3 \\
\\
\end{tabular}

\noindent \begin{tabular}{p{1.25in}p{2.5in}c}
\und{Fifth semester} & & \und{18} \\
CS/ECE 3330 & Computer Architecture & 3 \\
CS 4102 & Algorithms & 3 \\
CS & CS elective$^5$ & 3 \\
APMA & APMA elective$^4$ or APMA 3100 & 3 \\
HSS/UE & HSS or unrestricted elective$^{2,3}$ & 3 \\
HSS/UE & HSS or unrestricted elective$^{2,3}$ & 3 \\
\\
\end{tabular}

\noindent \begin{tabular}{p{1.25in}p{2.5in}c}
\und{Sixth semester} & & \und{15} \\
CS 3240 & Advanced Software Development & 3 \\
CS & CS elective$^5$ & 3 \\
APMA & APMA elective$^4$ or APMA 3100 & 3 \\
HSS/UE & HSS or unrestricted elective$^{2,3}$ & 3 \\
HSS/UE & HSS or unrestricted elective$^{2,3}$ & 3 \\
\\
\end{tabular}

\noindent \begin{tabular}{p{1.25in}p{2.5in}c}
\und{Seventh semester} & & \und{15} \\
STS 4500 & Western Tech and Culture & 3 \\
CS & CS elective$^5$ & 3 \\
CS & CS elective$^5$ or CS 4970 & 3 \\
CS 4414 & Operating Systems & 3 \\
HSS/UE & HSS or unrestricted elective$^{2,3}$ & 3 \\
\\
\end{tabular}

\noindent \begin{tabular}{p{1.25in}p{2.5in}c}
\und{Eighth semester} & & \und{15} \\
STS 4600 & Engineer in Society & 3 \\
CS & CS elective$^5$ & 3 \\
CS 4971 or 4980 & Capstone course & 3 \\
HSS/UE & HSS or unrestricted elective$^{2,3}$ & 3 \\
HSS/UE & HSS or unrestricted elective$^{2,3}$ & 3 \\
\\
\end{tabular}

\mysection{Miscellaneous Information}
\subsection{CS 2190 Specific Details}

While students can take courses in any semester, there is an issue to
consider with CS 2190: this course should be taken in the second year
or (less preferably) the third year. If a student reaches his/her
fourth year without taking the course, then s/he must take a 3 credit
course in ethics and technology in its place (even though CS 2190 is
only 1 credit). This course taken in place of CS 2190 does not count
toward any other requirement except to replace CS 2190.


%\clearpage
%\mysection{Course Requirements Flowchart}
%\begin{figure}[h!]
%\epsfig{figure=flowcharts/Cs-course-flowchart-summer-2010.png,width=4.5in}
%\end{figure}

\clearpage
\mysection{Course Requirements Flowchart}

\begin{figure}[h!]
\epsfig{figure=diagrams/bs-cs.pdf,width=4.5in}
\end{figure}


\clearpage
\mychapter{Bachelor of Arts in Computer Science}{BA CS Degree}
\label{bacschapter}

\mysection{Introduction}

Computer Science is the study of information processes. Computer
scientists learn how to describe information processes, how to reason
about and predict properties of information processes, and how to
implement information processes elegantly and efficiently in hardware
and software. The Computer Science major concentrates on developing
the deep understanding of computing and critical thinking skills that
will enable graduates to pursue a wide variety of possible fields and
to become academic, cultural, and industrial leaders in areas that
integrate the arts and sciences with computing. The Computer Science
major is designed to provide students entering the University without
previous background in computing with an opportunity to major in
Computer Science, while taking courses in arts, humanities, and
sciences to develop broad understanding of other areas and their
connections to computing. Computing connects closely with a wide range
of disciplines including, but not limited to, the visual arts, music,
life sciences including biology and cognitive science, the physical
sciences, linguistics, mathematics, and the social sciences. The core
curriculum focuses on developing methods and tools for describing,
implementing, and analyzing information processes and for managing
complexity including abstraction, specification, and recursion. 


\mysection{Application Process}
\label{bacsapplicationprocess}

The Department of Computer Science is experiencing tremendous student
interest in our degree programs. Our goal is to accommodate as many BA
in Computer Science (BA CS) majors as possible.  However, because of
current resources, the Department has had to institute a cap on the
number of students who can declare the BA CS major.


{\bf Requirements to declare the major:} In order to apply for the
major, students must have taken one introductory computer science
course (either CS 1110, CS 1111, or CS 1112, or CS 1120) with a grade
of C+ or better, and must be enrolled in CS 2110 and CS 2102 (or must
have already completed CS 2110 and CS 2102 with a grade of C+ or
better). Students are accepted into the major in the spring semester
of their second year upon review of their applications. This is a
selective process which takes into account the applicant's GPA and
application essay, as well as other factors.

{\bf Application information:} Applications must be completed in the
spring semester (normally the student's fourth semester).  Deadlines
are posted in the Computer Science Department office and on the
departmental web site; the deadline will be on or about March 15.  Due
to prerequisite dependencies, it is difficult for rising third year
students who have not completed CS 2110 and CS 2102 to complete the
major in the 4 remaining semesters; however, in exceptional cases,
students in that situation may apply for the major by petition to the
Chair.

Students apply to the Computer Science major by completing a form
available on the departmental
web site\myurl{http://www.cs.virginia.edu/acad/ba/about.html}. Students
list all CS courses taken or in progress and discuss any career goals
or aspirations, computing-related extra-curricular activities,
internships or experience. The essay invites students to reflect on
their intellectual objectives in wishing to pursue the major and asks
students to consider their career goals and how the BA in CS advances
those goals. Applications are read and evaluated by the core faculty
in Computer Science. All applicants will be notified of admission
decisions by April 1.

{\bf Second majors:} College of Arts and Sciences students who wish to
declare the BA CS as a second major must follow the same application
process described here. Only College of Arts and Sciences students are
eligible to apply for the BA CS degree as a second major.

{\bf Transfer students from outside the University:} Students
transferring into the University from other institutions must apply to
the department to be allowed to declare the BA CS major. Qualified
applicants will be considered on a space-available basis, given our
target caps for each class year. Applications will be considered the
summer before a transfer student begins classes, and the application
process will be discussed during the summer orientation session. If an
incoming transfer does not attend summer orientation, they must meet
with a CS advisor before classes begin to discuss applying.  Due to
prerequisite dependencies, it is difficult for rising third year
students who have not completed CS 2110 and CS 2102 to complete the
BA CS in the 4 remaining semesters. It is important that students
transferring to the University as third years complete the equivalent
of these courses before coming to UVa. In exceptional cases, students
in this situation may apply for the major, but the ability to complete
the degree in a timely fashion is considered in determining if
you are accepted into the degree program.




\mysection{Curriculum}

\paragraph{Prerequisites:}

Before declaring the computer science major, all students should have
taken one introductory computer science course (either CS 1110, CS
1111, or CS 1112; CS 1120 is also allowed if the student has Java
experience) with a grade of C+ or better, or have comparable
experience. Students may be permitted to declare the major while they
are currently taking the introductory course.

The major requires the College Competency and Area
Requirements\myurl{http://artsandsciences.virginia.edu/college/requirements/index.html}
as well as at least 27 credits in Computer Science courses and 12
credits in Integration Electives.

\subsection{Required ``Core'' Courses}
\label{sec:bacs-corecourses}

The following courses are required for all BA CS majors.  Full
descriptions can be found in the Course Descriptions section
(section~\ref{sec:coursedesc}, page~\pageref{sec:coursedesc}).

\begin{itemlist}
\item CS 2110, Software Development Methods
\item CS 2102, Discrete Math
\item CS 2150, Program and Data Representation
\item CS 3330, Computer Architecture
\item CS 4102, Algorithms
\end{itemlist}

Note that any CS1 class, either CS 111x, Introduction to Programming,
or CS 1120, Introduction to Computing: Language, Logic, and Machines,
is required to enroll in CS 2110.

\subsection{CS Electives}
\label{sec:bacs-cselectives}

Four computing intensive electives are to be selected from a list of
approved courses. The list of approved courses will initially comprise
current Computer Science courses at 3000-level or above as well as CS
2330. Additional courses that may be jointly offered by CLAS and CS
departments will be added to the list of approved computing electives
based on approval by the BA committee.

There are a few restrictions on which upper-level CS courses can count
as a CS elective:

\begin{itemlist}
\item CS 4971 (Capstone Practicum II) does not count as a CS elective,
as it is part of the BS CS capstone requirement.  Note that its
pre-requisite (CS 4970, Capstone Practicum I) {\em does} count as a CS
elective.
\item CS 4980 (Capstone Research) does not count as a CS elective, as
it is part of the BS CS capstone requirement.
\item CS 4993 (Independent Study) credits can only count for at most 1
CS elective (i.e., 3 credits).
\item CS 4998 (Distinguished BA Majors Research) is a separate
  requirement for the DMP (see section~\ref{sec:badmp},
  page~\pageref{sec:badmp}), and thus does not count as a CS elective.
\end{itemlist}

\subsection{Integration Electives}
\label{sec:bacs-integrationelectives}

Four courses selected with the approval of the student's advisor from
the list of computing-related courses approved by the BA CS
committee. These courses are offered by departments other than
Computer Science, and should either provide fundamental computing
depth and background or explore applications of computing to arts and
sciences fields. 

This is a list of the courses that are generally approved as
integration electives. This list is not meant to be exhaustive: if you
find a course that is not on the list that appears to satisfy the
goals of an integration elective, discuss with your advisor or the BA
Program Director if it should count as an integration elective for
you.

Some of these courses are not offered regularly, and some courses may
have prerequisites. The list of integration electives may change
slightly from year to year.  You can always check the current list of
integration electives on SIS.  The list below is according to SIS as
of September 2013.

\paragraph{Anthropology}
\begin{itemlist}
\item ANTH 2430: Languages of the World
\item ANTH 3480: Language and Prehistory
\item ANTH 3490: Language and Thought
\item ANTH 5401: Linguistic Field Methods
\item ANTH 5410: Phonology
\item ANTH 5420: Theories of Language
\item ANTH 5440: Morphology
\end{itemlist}

\paragraph{Architecture}
\begin{itemlist}
\item ARCH 3450: Digital Moviemaking \& Animation
\item ARCH 5420: Digital Animation \& Storytelling
\item ARCH 5450: Digital Moviemaking \& Animation
\item ARCH 5470: Information Space
\item ARCH 5710: Photography and Digital Media
\item ARCH 6410: Advanced CAAD 3D Modeling \& Visualization
\end{itemlist}

\paragraph{Studio Art}
\begin{itemlist}
\item ARTS 2220: Introduction to New Media I
\item ARTS 2222: Introduction to New Media II
\item ARTS 3222: Intermediate New Media II
\item ARTS 4220: Advanced New Media I
\item ARTS 4222: Advanced New Media II
\end{itemlist}

\paragraph{Biochemistry}
\begin{itemlist}
\item BIOC 5080: Computer Analysis of DNA \& Protein
\end{itemlist}

\paragraph{Biology}
\begin{itemlist}
\item BIOL 3170: Introduction to Neurobiology
\item BIOL 3240: Introduction to Immunology
\item BIOL 4010: Macroevolution
\item BIOL 4020: Ecol \& Evolutionary Genetics
\item BIOL 4030: Evolutionary Biology Lab
\item BIOL 4050: Developmental Biology
\item BIOL 4080: Neuronal Organization of Behavior
\item BIOL 4130: Population Ecology and Conservation Biology
\item BIOL 4160: Functional Genomics Lab
\item BIOL 4170: Cellular Neurobiology
\item BIOL 4250: Human Genetics
\item BIOL 4480: Complex Macromolecules
\item BIOL 5080: Developmental Mechanisms
\item BIOL 5370: Epidemiology and Evolution of Infections Disease
\end{itemlist}

\paragraph{Biomedical Engineering}
\begin{itemlist}
\item BME 3310: Biomed Systems Analysis \& Design
\item BME 3315: Computational BME
\item BME 3636: Neural Network Models
\item BME 4783: Medical Imaging Modalities
\item BME 4784: Medical Image Analysis
\end{itemlist}

\paragraph{Chemistry}
\begin{itemlist}
\item CHEM 4411: Biological Chemistry Lab I
\end{itemlist}

\paragraph{Drama}
\begin{itemlist}
\item DRAM 2110: Lighting Technology
\item DRAM 2110: Lighting Technology
\item DRAM 2210: Scenic Technology
\item DRAM 2240: Digital Design: Re-making and Re-imagining
\item DRAM 2620: Sound Design
\item DRAM 2630: Production Laboratory: Sound
\item DRAM 3210: Scene Design I
%\item DRAM 3640: Sound Design:studio
\item DRAM 4110: Lighting Design
\item DRAM 4410: Acting III
\end{itemlist}

\paragraph{Electrical Engineering}
\begin{itemlist}
\item ECE 2066: Science of Information
\end{itemlist}

\paragraph{Economics}
\begin{itemlist}
\item ECON 4010: Game Theory
\item ECON 4020: Auction Theory and Practice
\item ECON 4720: Econometric Methods
\item ECON 4880: Seminar in Policy Analysis
\end{itemlist}

\paragraph{Environmental Science}
\begin{itemlist}
\item EVSC 3020: GIS Methods
\item EVSC 4010: Introduction to Remote Sensing
\item EVSC 4040: Climate Change: Science, Markets \& Policy
\item EVSC 4070: Advanced GIS
\item EVSC 5020: GIS Methods
\item EVSC 5030: Applied Statistics for Environmental Scientists
\item EVSC 5110: Systems Analysis in Environmental Sciences
\end{itemlist}

\paragraph{United States History}
\begin{itemlist}
\item HIUS 3162: Digitizing America
\end{itemlist}

\paragraph{Linguistics}
\begin{itemlist}
\item LING 3400: Structure of English
\item LING 5010: Synchronic Linguistics
\item LING 5060: Syntax and Semantics
\item LING 5070: Syntactic Theory
\end{itemlist}

\paragraph{General Linguistics}
\begin{itemlist}
\item LNGS 3250: Intro to Linguistic Theory
\end{itemlist}

\paragraph{Mathematics}
\begin{itemlist}
\item MATH 1160: Algebra, Number Systems, and Number Theory
\item MATH 3000: Transition to Higher Math
\item MATH 3100: Intro Mathematical Probability
\item MATH 3120: Intro Mathematical Statistics
\item MATH 3351: Elementary Linear Algebra
\item MATH 3354: Survey of Algebra
\item MATH 4080: Operations Research
\item MATH 4300: Elementary Numerical Analysis
\item MATH 4452: Algebraic Coding Theory
\item MATH 4750: Introduction to Knot Theory
\item MATH 5110: Intro to Stochastic Processes
\item MATH 5651: Advanced Linear Algebra
\item MATH 5653: Number Theory
\end{itemlist}

\paragraph{Media Studies}
\begin{itemlist}
\item MDST 2010: Introduction to Digital Media
\item MDST 3050: History of Media
\item MDST 3702: Computers and Languages
\item MDST 3703: Digital Liberal Arts
\item MDST 4700: Theory of New Media
\end{itemlist}

\paragraph{Music}
\begin{itemlist}
\item MUSI 2350: Technosonics: Digital Music and Sound Art Composition
\item MUSI 3390: Intro to Music \& Computers
\item MUSI 4535: Interactive Media
\item MUSI 4540: Computer Sound Generation
\item MUSI 4543: Sound Studio
\item MUSI 4545: Computer Applications in Music
\item MUSI 7350: Interactive Media
\end{itemlist}

\paragraph{Neruscience}
\begin{itemlist}
\item NESC 5330: Neural Network Models
\end{itemlist}

\paragraph{Philosophy}
\begin{itemlist}
\item PHIL 1410: Forms of Reasoning
\item PHIL 2330: Computers, Minds and Brains
\item PHIL 2420: Introduction to Symbolic Logic
\item PHIL 5420: Advanced Logic
\item PHIL 5450: Language and Logic
\end{itemlist}

\paragraph{Physics}
\begin{itemlist}
\item PHYS 2660: Fundamentals Scientific Computing
\item PHYS 5630: Computational Physics I
\item PHYS 5640: Computational Physics II
\end{itemlist}

\paragraph{Psychology}
\begin{itemlist}
\item PSYC 2150: Introduction to Cognition
\item PSYC 2200: Survey of the Neural Basis of Behavior
\item PSYC 2300: Introduction to Perception
\item PSYC 4110: Psycholinguistics
\item PSYC 4111: Language Development \& Disorders
\item PSYC 4125: Psychology of Language
\item PSYC 4150: Cognitive Processes
\item PSYC 4200: Neural Mechanisms of Behavior
\item PSYC 4290: Memory Distortions
\item PSYC 4300: Theories of Perception
\item PSYC 4330: Topics in Child Development
\item PSYC 4500: Special Topics: Psychology
\item PSYC 5150: Advanced Cognition
\item PSYC 5210: Developmental Psychobiology
\item PSYC 5260: Brain Systems Involved in Learning and Memory
\end{itemlist}

\paragraph{Statistics}
\begin{itemlist}
\item STAT 2120: Intro to Statistical Analysis
\item STAT 3010: Statist Computing \& Graphics
\item STAT 5000: Intro to Applied Statistics
\item STAT 5330: Data Mining
\end{itemlist}


\paragraph{Using other courses.}  If a student would like to use a
course not on the above list as an integration elective, they should
first contact their academic advisor.  Their advisor can work with the
student to come up with a good argument as to why the course should
qualify, and once the advisor approves it, send it to the BA CS
Director at \bacsdirectoremail.  Alternatively, if the advisor
prefers, s/he can just send the student to BA CS director to get
approval for a requirement exception.

\mysection{Miscellaneous Information}

%\subsection{Declaring the Major}

%Before declaring the computer science major, students should have
%taken one introductory computer science course (CS 111x (101, 101E,
%101x), Introduction to Programming, or CS 1120, Introduction to
%Computing: Language, Logic, and Machines) with a grade of C+ or
%better, or have comparable experience. Students may be permitted to
%declare the major while they are currently taking the introductory
%course.
%
%To declare the major:
%
%\begin{numlist}
%
%\item Satisfy the major prerequisite by taking one of the introductory
%  computer science courses. CS 1120 is the recommended course
%  for most BA CS majors, but the other introductory courses (CS 1110
%  CS1111, and CS 1112) can also be used to satisfy the
%  prerequisite. You may declare the major before completing the course
%  as long as you are on track to complete the course successfully. If
%  you believe you have comparable experience in some other way, you
%  may also be able to declare the major.
%
%\item Pick up a Major Declaration Form from the Dean's office, and
%  fill out the top half.
%
%\item Arrange to meet with \badup\ (\badupemail), Director of the
%  Undergraduate Program (DUP). You can email \baduppronountwo to arrange
%  a meeting time, or drop by during \baduppronoun office hours.
%
%\end{numlist}


\subsection{Distinguished Majors Program}
\label{sec:badmp}

%Bachelor of Arts Computer Science 
BA CS majors who have completed 18 credit hours toward their major
and who have a cumulative GPA of 3.4 or better may apply to the
Distinguished Majors Program. Students who are accepted must complete
a thesis based on two semesters of empirical or theoretical
research. The Distinguished Majors Program features opportunities for
students and advisors to collaborate on creative research; it is not a
lock-step thesis program with strict content requirements. Upon
successful completion of the program, students will likely be
recommended for a baccalaureate award of Distinction, High
Distinction, or Highest Distinction.

Students applying to the DMP must have a minimum cumulative GPA of 3.4
and have completed 18 credit hours toward their Computer Science major
by the end of the semester in which they apply. These 18 credit hours
can can come from any course used to fulfill the core course
requirement (section~\ref{sec:bacs-corecourses},
page~\pageref{sec:bacs-corecourses}), CS electives
(section~\ref{sec:bacs-cselectives},
page~\pageref{sec:bacs-cselectives}), or the integration electives
(section~\ref{sec:bacs-integrationelectives},
page~\pageref{sec:bacs-integrationelectives}).
%Interdisciplinary Major in Computer Science Curriculum.  curriculum.
Exceptions to the 18 credit hours rule may be granted at the
discretion of the Distinguished Majors Program Director.

In addition to the normal requirements for the computer science major,
they must register for two semesters of supervised research (CS 4998
for 3 credits each semester). Students may apply to the DMP before
completing this supervised research, but students must complete the
supervised research to complete the DMP. Based on their independent
research, students must complete, to the satisfaction of their advisor
and the Distinguished Major Program Director, a project at least one
month prior to graduation.

Please note: The CS 4998 DMP credits do not apply toward the credit
hours required for the major. That is, they cannot be used to fulfill
any requirement listed on the BA CS curriculum.

For more information on the DMP, see
online\myurl{http://www.cs.virginia.edu/acad/ba/distinguished-major.html}.
You may also contact the BA CS director at \bacsdirectoremail, who is
in charge of the BA DMP program.

\subsection{Double majors in CLAS}

From the CLAS web site on
majors\myurl{http://college.artsandsciences.virginia.edu/majortypes}
regarding double majors:

\begin{quotation}
\noindent You may major in two subjects, in which case the application
for a degree must be approved by both departments or inter\-departmental
programs. Students who double major must submit at least 18 credits in
each major; these credits may not be duplicated in the other
major. There is no triple major.
\end{quotation}

However, you should be aware of the application process described in
section~\ref{bacsapplicationprocess}
(page~\pageref{bacsapplicationprocess}).

\mysection{Course Requirements Flowchart}

\begin{figure}[h!]
\epsfig{figure=diagrams/ba-cs.pdf,width=4.5in}
\end{figure}

\noindent Notes:

\begin{itemlist}
\item CS 2102 requires {\em either} CS 111x or CS 1120 as a
  prerequisite.
\end{itemlist}


\clearpage
\mychapter{Bachelor of Science in Computer Engineering}{BS CpE Degree}
\label{bscpechapter}

\mysection{Introduction}

Computer Engineering is an exciting field that spans topics across
electrical engineering and computer science.  Students learn and
practice the design and analysis of computer systems, including both
hardware and software aspects and their integration. Careers in
Computer Engineering (CpE) are as wide and varied as computer systems
themselves, which range from embedded computer systems found in
consumer products or medical devices, to control systems for
automobiles, aircraft, and trains, to more wide-ranging applications
in entertainment, telecommunications, financial transactions, and
information systems.

%\begin{quotation}
%Computer Engineering gives you a great working knowledge
%  and balance in both CS \& ECE. With the freedom to choose electives
%  in either department, you are in full control of your educational
%  experience and how you wish to enhance your knowledge. -- Kevin Chang,
%  08
%\end{quotation}

\subsection{Program Objectives}

Graduates of the Computer Engineering program at the
University of Virginia utilize their academic preparation
to become successful practitioners and innovators in
computer engineering and other fields. They analyze,
design and implement creative solutions to problems
with computer hardware, software, systems and
applications. They contribute effectively as team
members, communicate clearly and interact responsibly
with colleagues, clients, employers and society.

Faculty from the Computer Science and Electrical \& Computer
Engineering departments jointly administer the CpE undergraduate
degree program at the University of Virginia.

The Computer Engineering program does not offer a minor.

%\begin{quotation}
%It's the future.  Everything is digitized and computer
%  engineering allows you to keep up with changing technology. It's a
%  complex field with many great opportunities for advancement.  -- Rob
%  Yip, '08
%\end{quotation}

\mysection{Application Process}
\label{bscpeapplicationprocess}

The application process for the Computer Engineering degree are the
exact same as with the BS Computer Science degree (section
\ref{bscsapplicationprocess}, page~\pageref{bscsapplicationprocess}),
and thus they are not repeated here.


\mysection{Curriculum} % called ``Disciplines'' in the original

The curriculum has been carefully designed to ensure that students
obtain an excellent background in both Computer Science and Electrical
Engineering, providing breadth across these disciplines as well as
depth in at least one. All Computer Engineering students work through
an extended sequence of introductory, intermediate and advanced
courses:

\begin{itemlist}
\item CS 1110 Introduction to Computer Science
\item CS 2110 Software Development Methods
\item CS 2102 Discrete Math
\item ECE 2630 Introductory Circuit Analysis
\item ECE 2660 Electronics I
\item CS 2150 Program and Data Representation
\item ECE/CS 2330 Digital Logic Design
\item ECE 3750 Signals \& Systems I
\item CS 3240 Advanced Software Development
\item CS 3430 Introduction to Embedded Computer Systems
\item CS 4414 Operating Systems
\item ECE 4435 Computer Architecture \& Design
\item ECE 4440 Embedded Systems Design
\item CS/ECE 4457 Computer Networks
\end{itemlist}

%\paragraph{Please Note:} Course numbers changed in 2009. The old 
%course numbers are shown in parentheses.

In addition to providing breadth across the two areas,
this core of the Computer Engineering program provides
depth in the following areas:

\paragraph{Circuits}
\begin{itemlist}
\item ECE 2630: Introductory Circuit Analysis
\item ECE 2660: Electronics I
\end{itemlist}

\paragraph{Software Engineering}
\begin{itemlist}
\item CS 2110: Software Development Methods
\item CS 3240: Advanced Software Development
\end{itemlist}

\paragraph{Digital Logic}
\begin{itemlist}
\item ECE/CS 2330: Digital Logic Design
\item CS 2102: Discrete Math
\end{itemlist}

\paragraph{Computer Systems}
\begin{itemlist}
\item CS 2150: Program and Data Representation
%\item CS/ECE 3330: Computer Architecture (see \S\ref{embedded})
\item CS 3430 Introduction to Embedded Computer Systems
\item CS 4414: Operating Systems
\item ECE 4435: Computer Architecture \& Design
\item ECE 4436: Embedded Systems Design
\item CS/ECE 4457: Computer Networks
\end{itemlist}

\subsection{Grade Requirement}
To complete their program of study, computer engineering majors must
achieve a C average or better in their Computer Science and Electrical
Engineering courses.


%\begin{quotation}
%I decided to major in CPE because it gave me an opportunity
%  to combine two majors into one. I came to UVA interested in computer
%  science, but decided that I wanted to know more about the hardware
%  and doing CPE was the perfect choice for me.  In the long run having
%  this major can make the student more marketable because he or she
%  can take on careers in many paths. -- Alla Aksel, '04
%\end{quotation}

\subsection{Sample BS CpE Course Schedule}

Below is the recommended course of study for the bachelor's degree. If
one has already completed some of these classes (through AP credit,
for example), then your course of study would deviate from what is
shown below~-- consult your academic advisor for details.

\vspace{0.15in}

\noindent \begin{tabular}{p{1.25in}p{2.5in}c}
\und{First semester} & & \und{15} \\
APMA 1110 & Single Variable Calculus & 4 \\
CHEM 1610 \& 1611 & Chemistry for Engineers \& Lab & 4 \\
%CHEM 1611 & Chemistry Lab & 1 \\
ENGR 1620 & Introduction to Engineering & 3 \\
ENGR 1621 & Introduction to Engineering Lab & 1 \\
STS 1500 & Engineering, Technology \& Society & 3 \\
\\
\end{tabular}

\noindent \begin{tabular}{p{1.25in}p{2.5in}c}
\und{Second semester} & & \und{17} \\
APMA 2120 & Multivariate Calculus & 4 \\
PHYS 1425 \& 1429 & Physics I \& Lab & 4 \\
%PHYS 1429 & Physics I Workshop & 1 \\
CS 1110 & Intro to Computer Science & 3 \\
SCI & Science elective$^2$ & 3 \\
HSS & HSS elective$^1$ & 3 \\
\\
\end{tabular}

\noindent \begin{tabular}{p{1.25in}p{2.5in}c}
\und{Third semester} & & \und{16} \\
APMA 2130 & Ordinary Differential Equations & 4 \\
CS 2110 & Software Development Methods & 3 \\
CS 2102 & Discrete Mathematics & 3 \\
ECE 2630 & Introductory Circuit Analysis & 3 \\
HSS & HSS elective$^1$ & 3 \\
\\
\end{tabular}

\noindent \begin{tabular}{p{1.25in}p{2.5in}c}
\und{Fourth semester} & & \und{16} \\
CS 2150 & Program and Data Representation & 3 \\
CS/ECE 2330 & Digital Logic Design & 3 \\
ECE 2660 & Electronics I & 4 \\
UE & Unrestricted elective$^3$ & 3 \\
STS & 2xxx/3xxx elective & 3 \\
\\
\end{tabular}

\noindent \begin{tabular}{p{1.25in}p{2.5in}c}
\und{Fifth semester} & & \und{16} \\
ECE 3430 & Intro to Embed.\ Systems (see \S\ref{embedded}, p.~\pageref{embedded})& 3 \\
ECE 3750 & Signals \& Systems & 3 \\
CS/ECE & CS/ECE elective $^4$ & 3 \\
PHYS 2415 & General Physics II \& Lab & 4 \\
%PHYS 2419 & General Physics Lab I & 1 \\ 
UE & Unrestricted elective$^3$ & 3 \\
\\
\end{tabular}

\noindent \begin{tabular}{p{1.25in}p{2.5in}c}
\und{Sixth semester} & & \und{15} \\
APMA 3100 & Probability & 3 \\
ECE 4435 & Computer Architecture \& Design & 4.5 \\
CS 3240 & Advanced Software Development & 3 \\
CS/ECE & CS/ECE elective $^4$ & 3 \\
HSS & HSS elective$^1$ & 3 \\
\\
\end{tabular}

\noindent \begin{tabular}{p{1.25in}p{2.5in}c}
\und{Seventh semester} & & \und{16.5} \\
ECE 4440 & Embedded Systems Design & 4.5 \\
CS/ECE 4457 & Computer Networks & 3 \\
CS/ECE & CS/ECE elective $^4$ & 3 \\
UE & Unrestricted elective$^3$ & 3 \\
STS 4500 & Western Tech and Culture & 3 \\
\\
\end{tabular}

\noindent \begin{tabular}{p{1.25in}p{2.5in}c}
\und{Eighth semester} & & \und{16.5} \\
CS 4414 & Operating Systems & 3 \\
CS/ECE & CS/ECE elective $^4$ & 3 \\
UE & Unrestricted elective$^3$ & 3 \\
UE & Unrestricted elective$^3$ & 3 \\
STS 4600 & Engineer in Society & 3 \\
\end{tabular}

\paragraph{Footnotes:}

\begin{numlist}
\item Chosen from the approved list available in A122 Thornton Hall.
\item Chosen from: among BIOL 2010, BIOL 2020, CHEM 1620, ECE 2066, ENGR
  2500, MSE 2090, and PHYS 2620.
\item Unrestricted electives may be chosen from any graded course in
  the University except mathematics courses below MATH 1310 including
  STAT 1100 and 1120, and courses that substantially duplicate any
  others offered for the degree, including PHYS 2010, 2020; CS 1010,
  1020; or any introductory programming course. Students in doubt as
  to what is acceptable to satisfy a degree requirement should get the
  approval of their advisor and the dean's office, located in Thornton
  Hall, Room A122. APMA 1090 counts as a three-credit unrestricted
  elective.
\item Chosen from CS/ECE course at the 3000 level or higher. Two
  CS/ ECE electives must be 4000 level or above.
\end{numlist}

\subsection{Embedded Systems Course Requirement}
\label{embedded}

As of the spring semester of 2013, the computer engineering curriculum
was modified to require ECE 3430 (Introduction to Embedded Computing
Systems).  This course takes the place of CS/ECE 3330 (Computer
Architecture), which is no longer required for a computer engineering
degree.  In addition, ECE 4435 (Computer Architecture \& Design) was
restructured to include many of the topics in CS 3330.  Note that this
embedded systems requirement change also affects the Electrical
Engineering curriculum, but that is not being covered here.

For any computer science classes that require CS/ECE 3330 as a
pre-requisite, ECE 4435 is also allowed.  The one exception is CS/ECE 4457
(Computer Networks), which has either CS/ECE 3330 or ECE 3430 as
pre-requisites (i.e., not CS 4435).

%Questions to resolve by asking Joanne:
%\begin{itemlist}
%\item what are the name changes?  they are listed differently in
%  different places
%\item is that pre-requisite structure really correct? what about
%  cross-listed courses?  what about cs-only courses?
%\item what other changes to 4435 and 4440 are there?  name changes
%  also?
%\item how does the when-they-take-what table change?  i can replace
%  3330 with 3430, but does it move up a semester?  what about 4435 and
%  4440?
%\item where does ece 3430 go a few pages back? (page 26) i have it in
%  the general category
%\end{itemlist}

\mysection{Miscellaneous Information}

There are three CS capstone courses: CS 4970 (Capstone Practicum I), 
CS 4971 (Capstone Practicum II), and CS 4980 (Capstone Research).
Only the first one (CS 4970) counts as a CS/ECE elective for the
Computer Engineering degree; the other two can only count as an
unrestricted elective.

Students who are dual-majoring in BS CS and Computer Engineering may
have ECE 4435 count as their computer architecture (CS 3330)
requirement, although this will require a manual SIS exception to do
so.

Please refer to the Undergraduate
Record\myurl{http://records.ureg.virginia.edu/} for detailed
information about SEAS Academic Rules and Regulations including HSS
electives. In addition, guidelines such as course load, academic
probation and academic suspension can also be found in the Record.

%The Registrar web site provides a Course Renumbering Crosswalk to
%assist with the transition from 3 to 4 digit course
%numbers\myurl{http://www.virginia.edu/registrar/search.php}.


%\clearpage
%\mysection{Course Requirements Flowchart}
%\begin{figure}[h!]
%\epsfig{figure=flowcharts/cpe-flowchart.png,width=4.5in}
%\end{figure}

\clearpage
\mysection{Course Requirements Flowchart}

\begin{figure}[h!]
\epsfig{figure=diagrams/bs-cpe.pdf,width=4.5in}
\end{figure}



\clearpage
\mychapter{Minor in Computer Science}{CS Minor}
\label{sec:csminor}

%{\Large\em Warning: Due to limited resources, the department is not
%accepting any minor applications from non-SEAS students, and only a
%limited number of applications from SEAS students are accepted.}

\mysection{Introduction}

The Department of Computer Science provides a minor program for
qualified students. The courses in the minor program provide a solid
foundation in computer science. The minor program is a six course,
eighteen credit curriculum. The curriculum consists of the four
required courses and two elective courses. Full course descriptions
are at the end of this document in section~\ref{sec:coursedesc}
(page~\pageref{sec:coursedesc}).

%In the past, there were separate requirements for the minor for SEAS
%students and non-SEAS students.  These requirements have been
%streamlined into a single set of requirements for everybody.
 
\mysection{Application Process}
\label{minorapplicationprocess}

The department can only allow a limited number of SEAS students to
declare a minor in Computer Science due to a rapidly growing demand
for computing courses.  Unfortunately, at this time the University is
only able to accept SEAS students as CS minors. This situation will be
re-evaluated before each year's application deadline, and if a change
is made the CS minor
page\myurl{http://www.cs.virginia.edu/acad/cs_minor.html} will be
updated to reflect the change. The CS department continues to work
with the University to obtain resources that will allow more students
to declare the Computer Science minor.

Students wishing to declare the minor normally apply in the spring of
their first or second year. Applications from third and fourth year
students will be considered only if there are still available spaces
that were not taken earlier. The normal deadline is March 1. The
application form can be found
online\myurl{http://www.cs.virginia.edu/~horton/cs-minor-apply}.  All
applicants will be notified if they have been accepted as a CS minor
by April 1.

{\bf BS in Computer Engineering majors:} When the CpE program was created,
it was decided by the two departments that CpE students could not
declare the minor in CS. Because the CpE combines CS and EE, graduates
with this degree will automatically have the equivalent of the minor
in CS.

\mysection{Curriculum}

All SEAS (School of Engineering and Applied Science) students are
required to take (or place out of) CS 1110, as part of the SEAS
first-year curriculum. This course is also the first required course
for the minor.

The following are the first four courses required for the minor.

\begin{itemlist} 
\item CS 1110, CS 1111, or CS 1112: Introduction
  to Computer Science
\item CS 2110: Software Development Methods
\item CS 2102: Discrete Mathematics
\item CS 2150: Program and Data Representation
\end{itemlist}

Note that CS 1120 (From Ada and Euclid to Quantum Computing and the
World Wide Web) can replace the CS 111x requirement.  However, all
SEAS students are required to take a CS 111x course regardless, so
courses, so taking CS 1120 would not help at all.

Note that if you place out of CS 1110 via the placement exam,
you still have to take 6 CS courses; if you receive course credit for
it via the AP exam or transfer credit, then you need not substitute a
course in its place.

Furthermore, two additional computer science electives are
required. The elective courses must be computer science courses at the
3000 level or above. The only restriction on elective courses is a
limit to how many independent study courses one can count toward a
minor~-- contact the minor advisor for details at
\csminoradvisoremail.

Computer science courses typically build upon each other. In
particular, CS 1110 is a prerequisite of both CS 2110 and
CS 2102. CS 2110 and CS 2102 are both prerequisites
of CS 2150. In addition, CS 2150 is a prerequisite for
almost all of the computer science electives. The Department of
Computer Science also requires that its courses be passed at a certain
level (typically a C- or higher) in order to take successive
courses. Be aware that the department strictly enforces its
prerequisite policy.

 
%\mysection{Miscellaneous Information}
%\subsection{Declaring the Minor}
%
%To declare the minor:
%
%\begin{numlist}
%
%\item A student should have completed CS 1110 or 1120, CS 2110, and CS
%  2102. Furthermore, the student should have completed, or at least be
%  enrolled in, CS 2150.
%
%\item Complete the minor declaration form, which is available in the
%  Computer Science department's front office (Rice Hall, room 527).
%  The form has the title, ``School of Engineering and Applied Science,
%  Minor Declaration''~-- this is the form for everybody; the SEAS
%  school title at the top is because the CS department is in SEAS.
%
%\item Meet with the CS department's minor advisor(s), currently
%  \csminoradvisor\ (\csminoradvisoremail).  Bring a the minor
%  declaration form and your transcript (unofficial is fine).
%
%\item Assuming the form is approved, it will be processed by the
%  department.
%
%\end{numlist}


\clearpage
\mychapter{Master's in Computer Science}{Masters in CS}

% From a conversation with Jack Davidson on 8-20-13, the warning is
% being removed, as the department is now accepting masters students

%{\Large\em Warning: The department typically does not admit more than
%  one or two UVa students per year into the five-year Master's program
%  described in this chapter. Thus students may not be able to complete
%  this program and should discuss it with an advisor when planning.}

% this information is via Wes who spoke with Kathy Thornton in January 2011

\mysection{Introduction}

There are multiple ways that one can pursue a graduate degree.
Typically, undergraduates are interested in completing a Master's
program in 5 years, 1 year beyond that for a Bachelors.

The department maintains graduate program information
online\myurl{http://www.cs.virginia.edu/acad/graduate_program/}; that
web site contains more complete information than this chapter.  This
section pertains solely to obtaining a Master's in 5 years.  This type
of Masters degree is the ``terminal'' Masters degree discussed in the
graduate handbook; the other type of Masters degree (what one can earn
while working towards a Ph.D.) is not discussed here.

Students are often better served going to a different school for
graduate work.  Every school has biases and ways of doing things, and
if one spends all of their academic career at one institution, then
they don't see any other way.  This is especially true for
Ph.D.\ degrees, but also important (although less so) for Master's.

Students that are aiming for a 5-year Master's must still follow the
same rules and guidelines for all Computer Science and SEAS graduate
students.  These rules and guidelines can be found in the graduate
handbook (see above), and
online\myurl{http://www.seas.virginia.edu/admissions/graduate.php}.

We would like to stress again that this chapter focuses solely on
earning a Master's degree in 5 years, where the Bachelors also was
earned at UVa.


\subsection{When to Apply}

Although you are earning both degrees in 5 years, there is still will
be a formal switch between undergraduate and graduate student status.
Also see the graduate applications web
page\myurl{http://www.seas.virginia.edu/admissions/graduate.php}.

A student applies to the UVa graduate program in CS like anybody else,
but mentions in the application packet (specifically, in the statement
of purpose) that they are going for a terminal 5-year Master's.  Here,
'terminal' means that you are not pursuing a degree beyond that (i.e.,
a Ph.D.).

Note that in the application process, you are NOT considered a
transfer student, even if you already have taken some graduate courses
at UVa.

The easiest time to apply is in the fall of your 4th year (i.e. during
your 7th semester).  This would allow one to finish up 8 full
semesters as an undergraduate, and have 2 full semesters (plus
summers, potentially) as a graduate student.

Often students will use the summer (either before or after their 5th
year) to complete some Master's requirements.  Note that it is very
possible to complete a Master's in 1 year, but it will be a heavy work
load if classes are not taken during the summers.  Check the course
availability to see what courses, if any, are being offered.

One can certainly take more than 5 years to complete the Master's~--
you are paying tuition, after all~-- but typically students aim to
complete their Master's after 5 full years.

However, nothing requires that you apply at that time.  You can apply
in the spring of your 3rd year (i.e. 6th semester) or even earlier.
Should one decide to apply then (6th semester), they may graduate
early in 3.5 years (i.e. 7 semesters)~-- thus, they graduate 1
semester early from undergraduate, then enroll in their 8th semester
as a Master's student.

The benefit of applying a semester early is that one can have an
additional semester to work on graduate courses.  This must be
balanced with the concern of completing their undergraduate degree in
7 semesters.

Note that if you are a SEAS student and are graduating one semester
early, you must still write a senior thesis, and take STS 4500 and STS
4600.

One can apply earlier as well, such as in the fall of one's 3rd year
(i.e. in your 5th semester).  This would mean that one would complete
the undergraduate degree in 3 years (following the same time line as
completion of the undergraduate degree in 8 semesters, but accelerated
by 1 year), and complete the Master's in their 4th year.


\subsection{Degrees Offered}

The Department of Computer Science offers two different Master's
degrees.  The first is a Master's of Computer Science (MCS), and the
second is a Master's of Science in Computer Science (MS).  Both may be
obtained in 5 years, although most students will opt for the MCS.

From the perspective of employers, the two degrees are, for the most part,
equivalent.  The primary difference is that a MS requires a full
Master's thesis, with a complete faculty committee that looks for a
significant amount of work to have been accomplished.  A MS requires a
3-credit project, and that is judged only by the student's advisor.  A
faculty committee looking at a MS thesis will look for significantly
more work than what is required for an MCS project.  As a result, a
MCS is an easier degree to earn.

Students tend to complete the MCS instead of the MS, as the MS
requires a significant thesis, and the MCS requires a project that is
smaller in scope.


\mysection{Curriculum}

The full curriculum for a Master's degree is listed in the graduate
handbook\myurl{http://www.cs.virginia.edu/acad/graduate_program/grad-handbook.pdf}.
This section is only intended as a summary.

A Master's curriculum generally consists of 30 credits (i.e. 10
courses) in computer science.  One or two of these courses will be the
MCS project course or the MS thesis course(s).

Any course that counts toward the graduation requirements for your
undergraduate degree may {\em\bf NOT} count toward the graduation
requirements for your Master's~-- even if it is as an unrestricted
elective.  Thus, if you want to take graduate class(es) as an
undergraduate, and you want it to count toward your Master's
graduation requirements, you must ensure that you take enough classes
so that you could have graduated {\em\bf without} those graduate
class(es).  Thus, one must carefully work out which courses will count
for which degree.

Master's students do not need to take (or pass) the qualification
exams, as those are required for the Ph.D.\ degree only.  If you
decide to later transfer into the Ph.D.\ degree, then you will need to
take (or have taken) the qualification exams.

\mysection{Miscellaneous Information}

Generally, Master's students are not funded.  Thus, students will pay
tuition (and room/board, as appropriate).  The costs are analogous to
undergraduate rates: lower for in-state residents, and higher for
out-of-state residents.

The ``easiest''~-- and most typical~-- path to a 5-year Master's is to
apply in 7th semester, and already have some graduate classes that are
NOT counting toward your undergraduate degree.  You will have had to
have talked to your undergraduate advisor about who you are going to
work with for your Master's.  You would then complete the degree in 1
full additional year (summer, fall, and spring), aiming for a May
graduation date.


\clearpage
\mychapter{Common Information}{Common Information}

\mysection{Major Focal Paths}

A focal path is a selection of courses that a student can take to
fulfill the various elective requirements, which are described in
detail in the sections on elective information for the various
majors. They do not change any of the requirements, and students are
not required to follow a focal path. They are included simply to give
prospective majors an idea about various classes that they can take to
fulfill an interest that they may have in computing. Not all focal
paths have classes to fulfill each elective requirement. And some will
have more classes than are needed for the given requirement.

In an effort to keep down the space for each listing, the reason for
each class is not listed~-- if interested, speak to a CS faculty
member in that particular area. Also, as BA CS students may be
interested in these focal paths, a line listing the BA CS requirements
is also shown below.

There are a number of other areas for which focal paths are being
developed, and we expect to include them in future editions of this
handbook. Those areas are: Systems, Parallel \& Distributed Computing,
Graphics, Languages \& Compilers, Software Engineering, Hardware, and
Security \& Privacy.


\subsection{Game Design}
\begin{itemlist}
\item Science elective (1): N/A
\item HSS electives (5): anything relating to asset development
  (sound, images, video, etc.)
\item Unrestricted elective (5): digital art classes, such as ARTS
  2220, 2222, 3220, 3222, 4220, and 4222; sound design courses, such
  as DRAM 2620 and DRAM 3640; modeling classes such as ARCH 3410.
  Also consider additional CS electives.
\item APMA electives (2): linear algebra (APMA 3080)
\item CS electives (5): game development courses (offered as special
  topics courses, CS 4501); graphics (CS 4810), artificial
  intelligence (CS 4710), networks (CS 4457), databases (CS 4750),
  parallel computing (CS 4444)
\item STS 2xxx/3xxx elective (1): N/A
\item Notes: You will need a lot of C++ experience upon graduation
\end{itemlist}

\subsection{Theory}
\begin{itemlist}
\item Science elective (1): ECE 2066 (Science of Information)
\item HSS electives (5): mathematical economics (ECON 3090),
  psycho-linguistics (PSYC 4110)
\item Unrestricted electives (5): game theory (ECON 401), various math
  courses (MATH 4452, MATH 5700, STAT 3010)
\item Unrestricted elective (1): N/A
\item APMA electives (2): linear algebra (APMA 3080)
\item CS electives (5): programming languages (CS 4610), artificial
  intelligence (CS 4710), cryptography (offered as special topics
  courses, CS 4501)
\item STS 2xxx/3xxx elective (1): N/A
\item Notes: BA students need to take CS 302 which is critical for a
  theory focal path, but is not (at this time) a required course for
  the BA
\end{itemlist}

\subsection{Networks (including wireless networks)}
\begin{itemlist}
\item Science elective (1): ECE 2066 (Science of Information)
\item HSS (5): N/A
\item Unrestricted electives (5): ECE 2630 (Circuits), ECE 3750
  (Signals), ECE 4710 (Communications), ECE 4290 (Wireless Systems),
  ECE 4785 (Optical Communications)
\item APMA electives (2): APMA 3120 (Statistics)
\item CS electives (5): CS 4457 (Networks), CS 4458 (Internet
  Networks), all wireless sensor networks courses offered as special
  topics, CS 4753 (Electronic Commerce), CS 4720 (Web and Mobile
  Systems), CS 4444 (Parallel Computing)
\item STS 2xxx/3xxx elective (1): N/A
\item Notes: The wireless networking class is often offered as a graduate class (called wireless sensor networks) and can be added with instructor permission. 
%\item Science elective (1): N/A
%\item HSS electives (5): N/A
%\item Unrestricted electives (5): electronic commerce (SYS 2057),
%  hardware based communications (ECE 4710, ECE 4784)
%\item Unrestricted elective (1): electronic commerce (COMM 4240)
%\item APMA electives (2): N/A
%\item CS electives (5): networks (CS 4457), Internet networks (CS
%  4458), wireless networks (offered as special topics courses, CS
%  4501), electronic commerce (CS 4753), parallel computing (CS 4444)
%\item STS 2xxx/3xxx elective (1): N/A
%\item Notes: the wireless networking class is often offered as a
%  graduate class (called wireless sensor networks), and can be added
%  with instructor permission
\end{itemlist}

\subsection{Web technologies}
\begin{itemlist}
\item Science elective (1): ECE 2066 (Science of Information)
\item HSS (5): Digital art classes, ECON 2010 (Micro Economics), ECON 2020 (Macro Economics), PSYC 1010 (Intro to Psychology)
\item Unrestricted electives (5): STS 4110 (Business of New Product Development)
\item APMA electives (2): N/A
\item CS electives (5): CS 4753 (Electronic Commerce), CS 4457 (Networks), CS 4720 (Web and Mobile Systems), CS 4750 (Database Systems), CS 4240 (Software Design)
\item STS 2xxx/3xxx elective (1): STS 2160 (Intellectual Property)
\item Notes: There are a number of IT classes that are relevant, including courses in web design, technology, and marketing.  However, these are not allowed as unrestricted electives per SEAS policy. 
%\item Science elective (1): ECE 200
%\item HSS and unrestricted electives (10): digital art classes
%\item Unrestricted electives (5): COMM 424 (if no 453), TMP 351
%\item APMA electives (2): N/A
%\item CS electives (5): Electronic commerce (CS 5753); networks (CS
%  4457); Internet networks (CS 4458); web-based courses (offered as
%  special topics courses, CS 4501)
%\item STS 2xxx/3xxx elective (1): STS 2160
%\item Notes: There are a number of IT classes that are relevant,
%  including IT 323 (Web design), IT 332 (Web Tech), IT 334 (Web
%  marketing). However these are not allowed as unrestricted electives
%  as per SEAS policy.
\end{itemlist}

\subsection{Software Engineering}
\begin{itemlist}
\item Science elective (1): Science of Information (ECE 2066)
\item HSS electives (5): See note below regarding the Engineering
  Business Minor.
\item Unrestricted elective (5): See note below regarding the
  Engineering Business Minor.
\item APMA electives (2): any choices are suitable
\item CS electives (5): Principles of Software Design (CS 4240) and
  HCI in Software Development (CS 3205) have a strong software
  engineering focus. Other courses that include significant development
  projects would be appropriate, such as Databases (CS 4750), Web and
  Mobile Systems (CS 4720), and Computer Game Design (CS 4730).
\item STS 2xxx/3xxx elective (1): Any offerings related to technology
  in society or ethics would be appropriate
\item Capstone sequence: the Practicum Capstone track
\item Notes: If special topics courses were offered in software
  testing, software quality, or formal methods, these would be good
  choices for this focal path. Also, the Engineering Business Minor
  would be a good addition for this focus. Finally, experience within
  a software company through a summer internship will increase your
  understanding of this area.
\end{itemlist}

\mysection{Frequently Asked Questions}

\subsection{What computer science student groups exist?}

There are five computer science student groups at UVa.

The {\bf Association for Computing Machinery} Chapter at the
University of Virginia is a student chapter of the national parent
Association for Computing Machinery. The Chapter is a Contracted
Independent Organization (CIO) at the University of Virginia, and
serves students, faculty, and staff of the University. Numerous events
are held each year, including technical talks, workshops, social
events, hack-a-thons, etc.  Any student at UVa may become a member of
the chapter.  Also see their web
site\myurl{http://acm.cs.virginia.edu/}.

The {\bf Programming Contest Teams} at UVa participates in the
International Collegiate Programming Contest (ICPC)~-- there is a
regional contest in the fall, and potentially the World Finals in the
late spring.  UVa has had great success recently with earning a World
Finals berth.  The same group of students also host UVa's High School
Programming Contest in the spring, which is the largest such contest
in the mid-Atlantic region.  More information about the programming
contest teams can be found
online\myurl{http://www.cs.virginia.edu/~asb/icpc/}.

{\bf ACM-W} is the ACM committee on Women in Computing. It celebrates,
informs and supports women in computing, and works with the ACM-W
community of computer scientists, educators, employers and policy
makers to improve working and learning environments for women. Also
see their web site\myurl{http://www.cs.virginia.edu/~acm-w/}.

The {\bf Student Game Developers} seeks to bring together students who
are interested in learning and experiencing the art of computer game
development. They have resources available for programmers as well as
non-programmers, weekly informative meetings, and many industry
contacts for lectures, resume building, and networking. Also see their
web site\myurl{http://gamedevatuva.blogspot.com/}.

{\bf PARFAIT} is a student-run club focused on mobile application
development. It provides a way for student developers to meet and
create project teams so that they can improve their mobile development
skills, develop a portfolio of apps, and provide the University with a
much needed mobile development resource. Find out more using the "contact
owners" link at the parfait mailing list
page\myurl{https://lists.virginia.edu/sympa/info/parfait}.

\subsection{What is ICPC, the International Collegiate Programming
  Contest, and how do I get involved?}

The International Collegiate Programming Contests, abbreviated ICPC,
is a world-wide contest of computer programming for college
students. UVa has a very active programming contest team. Regional
contests occur in the fall.  Our region is the nearest 6 (or so)
states and Washington, D.C. The top team(s) from each regional contest
advance to the world finals, which consists of the top 100 teams from
around the world. UVa has qualified for the world finals twice in the
recent years: for the 2009 world finals in Stockholm, Sweden, and the
2010 world finals in Harbin, China. We typically have seven teams (of
three students each) compete in the regional contest. Our programming
contest teams practice throughout the year. If you are interested in
more information, you can either contact UVa's local ACM
chapter\myurl{http://acm.cs.virginia.edu} or ACM's advisor,
\acmadvisor\ (\acmadvisoremail).

\subsection{What kind of advanced placement credit is available?}
\label{applacement}

Advanced placement (AP) credit is awarded by the University for most
AP tests in which the grade is a 4 or a 5. This section only deals
with the AP Computer Science test. A student's SIS report will
always list which courses qualify for the AP test scores (both in
computer science and in other fields).

Students who receive a 4 or 5 on the Computer Science A test will
receive credit for CS 1110.  Students who took the higher-level Computer
Science course through International Baccalaureate will get credit for
CS 1110 with a 5 and for both CS 1110 and CS 2110 with a 6 or 7.

Note that CS 2110 is required for other majors: computer engineering,
systems engineering, and electrical engineering. There is also a
placement exam before the fall semester that will allow the student to
place out of CS 1110, but does not allow credit to be received for the
course~-- the student must then take another 3 hour CS or technical
course (see your advisor for details about a 'technical course')
instead. See the next question and answer for information about the CS
1110 placement exam.

\subsection{Can I place out of CS 1110? What about CS 2110 ?}
\label{101placement}

There is a placement exam for CS 1110, which covers all the topics
taught in the course. For the current semester's syllabus, see the CS
1110 course web site. Successful completion will allow a student to
place out of the course, but does NOT give course credit~-- only a
sufficient score on the AP test or transfer credit can give course
credit for CS 1110. A student must still take CS 2110 or a technical
course (see your advisor for details about a 'technical course') to
fulfill the SEAS CS 1110 requirement (note that since computing majors
have to take CS 2110 anyway, they can not use that course for both
requirements, and must thus use a technical elective instead). The
test is offered before the beginning of the fall semester. Note that
any student who has enrolled in CS 1110 or equivalent (CS 1111 or CS
1112) and got a letter grade~-- including a 'W'~-- is not allowed to
take the placement exam (in other words, if you enroll and then drop
the course without a 'W', you may still take the placement exam). The
exam may be taken by visiting the departmental office in Rice Hall,
room 527.

More information about the CS 1110 placement exam may be found online
\myurl{http://www.cs.virginia.edu/cs1110/placement.php}.

Computing majors may place out of CS 2110, but they must take another
CS course of a greater number (i.e., greater than CS 2110) instead.
For information about the placement exam for CS 2110, please contact
the current CS 2110 instructor.

\subsection{How does SEAS handle transfer credit?}

The Engineering School handles transfer credit, such as from an AP
course or transfer from another school. The credit will appear on your
SIS report, along with the UVa courses that you received credit
for. Note that the credit amounts need to match - so if you are
getting credit for APMA 2120 (Multivariate Calculus), which is a
4 credit course, the number of credits you transfer in should
(ideally) also be 4 credit hours. If it does not (your equivalent
course at another school was only 3 credits), you will have to take
another math or technical course (see your advisor for details about a
'technical course') to make up for the discrepancy. Note that placing
out of a course (such as CS 1110, APMA 2120, etc.) through
the respective placement exam does not give credit – and thus the
credits need to be made up through other courses (in the case of CS
1110, 3 credits of a technical course will fill that spot; in
the case of APMA 2120, 4 credits of math or a technical elective
will fill that spot). AP exams do give course credit.

Note that half of the 128 credits that one uses to graduate must be
earned at UVa. Thus, if you transfer with more than 64 credits, you
must still take 64 credits at UVa.

\subsection{Can CS courses from another college receive credit?}

We officially discourage taking major courses elsewhere. This policy
is especially true for the lab-based and required courses. If, in
spite of this departmental policy, you still want to take a course
elsewhere, then the student needs an advisor signature AND the
signature of the current instructor of that course from UVa. To
receive the required signatures, you must bring in a detailed
syllabus, so that faculty can make informed decisions. Note that to
receive credit for CS 2150 elsewhere, you need a course (or
multiple courses) that cover(s) data structures, C++, and assembly
language programming.

\subsection{What are the Rodman Scholar requirements?}

Rodman Scholars\myurl{http://seas.virginia.edu/students/rodmans/} have
slightly different
requirements\myurlremember{rodmanreqs}{http://seas.virginia.edu/students/rodmans/curriculum.php}
for graduation.  While those requirement differences are included
here, one should check at that URL\myurlrecall{rodmanreqs} for the
most up-to-date information, as that URL will supersede the
information in this section.

\begin{itemlist}
\item 
\item Introduction to Engineering: Instead of ENGR 1620 and ENGR 1621,
  Rodman Scholars take ENGR 1410 (Synthesis Design I) in the fall of
  their first year, and ENGR 1420 (Synthesis Design II) in the spring
  of their first year.  Note that this sequence is 6 credits, whereas
  the ENGR 1620 and ENGR 1621 sequence is only 4 credits.  Because
  ENGR 1410 and ENGR 1420 this is a fall-spring sequence, mid-year
  applicants are not required to take it, and can use ENGR 1620/1621
  credit instead.
\item A total of four 1-credit seminars (ENGR 3580) must be taken
  prior to graduation.
\item The requirements in place of STS 1500 are in a state of flux.
  For the classes of 2014 and earlier, STS 1500 is not required, but
  an HSS elective (or appropriate AP credit) must be taken in its
  stead.  For the classes of 2015 and beyond, STS 1500 is required,
  but with a separate Rodman-only discussion section.
\item Physics: Instead of PHYS 1425 and PHYS 2415 (and their
  associated labs), Rodmans may choose
  to take a three-course sequence for Physics majors: PHYS 1610, PHYS
  1620, and PHYS 2610.
\end{itemlist}

\subsection{Why are ECE 4435 and ECE 4440 not showing up in my list of
  fulfilled CS electives?}
\label{sec:sisece4435issue}

This has to do with a restriction in how SIS handles the CS elective
requirements.  While ECE 4435 (Computer Architecture \& Design) and
ECE 4440 (Embedded Systems Design) can both count as a CS elective
each, in order for this to happen a SIS exception will need to be
entered.  Your academic advisor can request to have such an exception
entered by contacting the SEAS undergraduate office.  Note that both
of those courses only count as one elective each, even though they are
4.5 credits per course (meaning that taking both of those classes~--
worth 9 credits total~-- does not count as three CS electives, but
only as two).

\subsection{Why do the SIS requirements for the BS CS major list 6 HSS
  electives, and not 5?}
\label{sec:sishssissue}

This has to do with how SIS (the Student Information System, UVa's
system for handling academic requirements and registration) handles
major requirements, and is done to allow for people to place out of
STS 1500 (previously STS 101, and STS 1500). If one does not
place out of STS 1500, then STS 1500 will list both in the STS 1500
requirement, and in the HSS requirement, thus requiring students to
take 5 additional HSS courses. If one does place out of STS 1500, they
need to take an additional HSS course in its place. So the credit to
place out of STS 1500 will appear in the STS 1500 requirements, and
will still require 6 (not 5) HSS courses. We think this is all a bit
bizarre as well, but that is how SIS handles requirements.

%A sample of the BS CS requirements can be found
%online\myurl{http://www.cs.virginia.edu/bscs/bscs-reqs-in-sis.pdf}~--
%your individual one can be found via SIS.

\subsection{Can CS students study abroad?}

Yes! To get more information about studying abroad, see
online\myurl{http://www.cs.virginia.edu/~horton/study_abroad/}
for more details.
 
\subsection{How do I transfer into the computing program?}

Students must decide which school they want to receive their degree
in: either in the Engineering School (which will yield a BS CS or BS
CpE degree; chapter~\ref{bscschapter} (page~\pageref{bscschapter})
details the BS CS degree, and chapter~\ref{bscpechapter}
(page~\pageref{bscpechapter}) details the BS CpE degree) or the
College of Arts and Sciences (which will yield a BA CS degree,
detailed chapter~\ref{bacschapter} (page~\pageref{bacschapter}).
Students must then apply for the degree in that school; the
application process is described in
section~\ref{bscsapplicationprocess}
(page~\pageref{bscsapplicationprocess}) for the BS CS and CpE degrees,
or section~~\ref{bacsapplicationprocess}
(page~\pageref{bacsapplicationprocess}) for the BA CS degree.

%Like other SEAS students, transfer students must formally apply to,
%and be approved by, the Department of Computer Science to enroll in
%the computer science program of study. To minimize loss of credit upon
%transfer, students must take a rigorous program in mathematics and the
%sciences. The School of Engineering and Applied Science expects a
%minimum of 63 credits in the first two years, instead of the 60-credit
%minimum that is customary in the College of Arts and Sciences. The
%additional credits are often completed through summer
%courses. Detailed information on curriculum requirements may be
%obtained from the Office of the Dean of the School of Engineering and
%Applied Science.
%
%There is also the Bachelor of Arts in Computer Science, offered
%through the College; also see their main
%web page\myurl{http://www.cs.virginia.edu/ba/}. Students
%outside of the School of Engineering and Applied Science with an
%interest in obtaining a BS (as opposed to a BA) degree in computer
%science must transfer to the Engineering school.


\subsection{Where can I find out about the Business minor?}

The courses for the Engineering Business Minor can be found on the web
page for the Technology Entrepreneurship Program.  These courses can
be worked into the various electives for the BS CS. More details can
be found online\myurl{http://techentrepreneurship.seas.virginia.edu/}.


\subsection{What CS electives can be taken without having completed CS
  2150?}

There are a few CS electives that one can take having only taken CS
2110.  They include:
\begin{itemlist}
\item CS 3102, Theory of Computation
\item CS 3205, Human-Computer Interaction
\item For the BA CS, CS 2330, Digital Logic Design, counts as an
  elective
\item CS 3330, Computer Architecture.
\end{itemlist}

There are a few courses that require CS 3330 as a prerequisite, but do
not also require CS 2150:
\begin{itemlist}
\item CS 4434, Dependable Computing (note that this course
    has other prerequisites)
\item CS 4457, Computer Networks
\end{itemlist}

\subsection{Why is CS 2330, Digital Logic Design, not offered
  in the spring?}
\label{cs2330}

CS 2330, Digital Logic Design, is cross-listed with ECE 2330.  Either
course counts for this requirement, and it does not matter which one
you take.  For unknown reasons, it is not cross-listed with CS 2330 in
the spring, but it is in the fall.  We don't understand why, either.
But you can take ECE 2330 to fulfill this requirement, as it's all the
same course.



\mysection{Course Descriptions}
\label{sec:coursedesc}

These course listings in this section are from the undergraduate
record\myurl{http://records.ureg.virginia.edu/content.php?catoid=11\&navoid=189}.
Table~\ref{table:course-offering-history}
(page~\pageref{table:course-offering-history}) gives this history of
which courses were taught when for the last few years.  The department
generally cannot guarantee that electives can be taught at least once
a year, as when an elective is offered is dependent on the available
faculty.  However, many electives are still offered regularly.
Lower-level causes are typically taught every semester.

\subsection{Course Offering History}

\begin{table}[h!]
\begin{center}
\begin{tabular}{l|cccccccccccc}
\label{table:course-offering-history}
Course & 08 & 09 & 09 & 10 & 10 & 11 & 11 & 12 & 12 & 13 & 13 \\
& F & S & F & S & F & S & F & S & F & S & F \\ \hline\hline
CS 1010 & X & X & X & X & X & X & X & X & X & X & X \\
CS 1110 & X & X & X & X & X & X & X & X & X & X & X \\
CS 1111 & X & X &   & X & X & X & X & X & X & X & X \\
CS 1112 & X & X & X & X & X & X & X & X & X & X & X \\
CS 1120 &   & X & X & X & X & X & X & X & X & X &   \\
CS 2102 & X & X & X & X & X & X & X & X & X & X & X \\
CS 2110 & X & X & X & X & X & X & X & X & X & X & X \\
CS 2150 & X & X & X & X & X & X & X & X & X & X & X \\
CS 2190 &   & X &   & X &   & X &   & X &   & X &   \\
%CS 2220 & X &   & X &   & X &   & X &   &   &   &   \\
CS 2330 & X & X & X & X & X & X & X & X & X & X & X \\
CS 2501 & X &   &   &   & X &   & X &   &   &   & X \\
CS 3102 &   & X & X & X & X & X & X & X & X & X & X \\
CS 3205 & X &   &   &   &   & X &   & X &   & X & X \\
CS 3240 &   & X &   & X &   & X &   & X & X & X & X \\
CS 3330 & X & X & X & X & X & X & X & X & X &   & X \\
CS 4102 & X &   & X & X & X & X & X & X & X & X & X \\
CS 4240 &   &   & X &   & X &   & X &   & X &   & X \\
CS 4330 &   & X &   &   &   &   &   &   &   &   &   \\
CS 4414 & X & X & X & X & X & X & X & X & X & X & X \\
CS 4434 &   &   &   & X &   &   &   & X &   &   &   \\
CS 4444 & X &   & X &   & X &   & X &   &   & X &   \\
CS 4457 & X & X & X & X & X & X & X & X & X & X & X \\
CS 4458 & X &   &   &   &   &   &   &   &   &   &   \\
CS 4501 &   &   &   &   & X & X &   &   & X & X &   \\
CS 4610 &   &   & X &   &   &   &   & X &   &   &   \\
CS 4620 &   &   &   &   &   &   &   &   &   &   & X \\
CS 4630 &   & X & X & X &   & X &   & X &   & X & X \\
CS 4710 &   & X &   &   &   & X &   & X &   & X &   \\
CS 4720 &   &   & X &   & X &   & X &   & X &   & X \\
CS 4730 &   &   &   & X &   &   & X &   &   &   &   \\
CS 4740 &   &   &   &   &   &   &   & X &   & X &   \\
CS 4750 & X &   &   & X &   & X &   & X &   & X & X \\
CS 4753 & X &   & X &   & X &   & X &   & X &   & X \\
CS 4810 & X &   &   & X &   &   & X &   & X &   &   \\
CS 4970 &   &   &   &   &   &   &   & X & X &   & X \\
CS 4971 &   &   &   &   &   &   &   &   &   & X &   \\
\end{tabular}
\caption{Computer Science Course Offering History}
\end{center}
\end{table}

\subsection{1000 Level CS Courses}

\course{CS 1010}{Introduction to Information Technology}{3
  credits}{Provides exposure to a variety of issues in information
  technology, such as computing ethics and copyright. Introduces and
  provides experience with various computer applications, including
  e-mail, newsgroups, library search tools, word processing, Internet
  search engines, and HTML. Not intended for students expecting to do
  further work in CS. Cannot be taken for credit by students in SEAS
  or Commerce.}{}

%\course{CS 1020}{Introduction to Business Computing}{3
%  credits}{Overview of modern computer systems and introduction to
%  programming in Visual Basic, emphasizing development of programming
%  skills for business applications. Intended primarily for
%  pre-commerce students. May not be taken for credit by students in
%  SEAS.}{}

\clearpage

\course{CS 1110}{Introduction to Programming}{3 credits}{Introduces
  the basic principles and concepts of object-oriented programming
  through a study of algorithms, data structures and software
  development methods in Java. Emphasizes both synthesis and analysis
  of computer programs.}{}

\course{CS 1111}{Introduction to Programming}{3 credits}{Introduces
  the basic principles and concepts of object-oriented programming
  through a study of algorithms, data structures and software
  development methods in Java. Emphasizes both synthesis and analysis
  of computer programs.}{Prerequisite: Prior programming experience.}

\course{CS 1112}{Introduction to Programming}{3 credits}{Introduces
  the basic principles and concepts of object-oriented programming
  through a study of algorithms, data structures and software
  development methods in Java. Emphasizes both synthesis and analysis
  of computer programs. Note: No prior programming experience
  allowed.}{}

\course{CS 1120}{Introduction to Computing: Explorations in Language,
  Logic, and Machines}{3 credits}{This course is an introduction to
  the most important ideas in computing. It focuses on the big ideas
  in computer science including the major themes of recursive
  definitions, universality, and abstraction.  It covers how to
  describe information processes by defining procedures using the
  Scheme and Python programming languages, how to analyze the costs
  required to carry out a procedure, and the fundamental limits of
  what can be computed.}{}

\course{CS 1501}{Special Topics in Computer Science}{1
  to 3 credits}{Content varies annually, depending on students needs
  and interests. Recent topics include the foundations of computation,
  artificial intelligence, database design, real-time systems,
  Internet engineering, wireless sensor networks, and electronic
  design automation.}{Prerequisite: Instructor permission.}

\subsection{2000 Level CS Courses}

\course{CS 2102}{Discrete Mathematics}{3 credits}{Introduces discrete
  mathematics and proof techniques involving first order predicate
  logic and induction. Application areas include finite and infinite
  sets and elementary combinatorial problems. Development of tools and
  mechanisms for reasoning about discrete problems.}{Prerequisite: CS
  1110, 1111, 1112 or 1120 with a grade of C- or higher.}

\course{CS 2110}{Software Development Methods}{3 credits}{A
  continuation of CS 1010, emphasizing modern software development
  methods. An introduction to the software development life cycle and
  processes. Topics include requirements analysis, specification,
  design, implementation, and verification. Emphasizes the role of the
  individual programmer in large software development
  projects.}{Prerequisite: CS 1010, 1111, or 1112 with a grade of C-
  or higher.}

\course{CS 2150}{Program and Data Representation}{3 credits}
{Introduces programs and data representation at the machine level.
  Data structuring techniques and the representation of data
  structures during program execution. Operations and control
  structures and their representation during program execution.
  Representations of numbers, arithmetic operations, arrays, records,
  recursion, hashing, stacks, queues, trees, graphs, and related
  concepts.}{Prerequisite: CS 2102 and CS 2110, both with grades of C-
  or higher.}

\course{CS 2190}{Computer Science Seminar}{1 credit}{Provides cultural
  capstone to the undergraduate experience. Students make
  presentations based on topics not covered in the traditional
  curriculum. Emphasizes learning the mechanisms by which researchers
  and practicing computer scientists can access information relevant
  to their discipline, and on the professional computer scientist's
  responsibility in society.}{Prerequisite: CS 2102 and 2110, both
  with a grade of C- or higher.}

\course{CS/ECE 2330}{Digital Logic Design}{3 credits}{Includes number
  systems and conversion; Boolean algebra and logic gates;
  minimization of switching functions; combinational network design;
  flip-flops; sequential network design; arithmetic networks.
  Introduces computer organization and assembly language. Cross-listed
  as ECE 2330.}

\course{CS 2501}{Special Topics in Computer Science}{1
  to 3 credits}{Content varies annually, depending on students needs
  and interests. Recent topics include the foundations of computation,
  artificial intelligence, database design, real-time systems,
  Internet engineering, wireless sensor networks, and electronic
  design automation.}{Prerequisite: Instructor permission.}

\subsection{3000 Level CS Courses}

\course{CS 3102}{Theory of Computation}{3 credits}{Introduces
  computation theory including grammars, finite state machines and
  Turing machines; and graph theory.}{Prerequisites: CS 2102 and CS
  2110 both with grades of C- or higher.}

\course{CS 3205}{HCI in Software Development}{3
  credits}{Human-computer interaction and user-centered design in the
  context of software engineering. Examines the fundamental principles
  of human-computer interaction. Includes evaluating a system's
  usability based on well-defined criteria; user and task analysis, as
  well as conceptual models and metaphors; the use of prototyping for
  evaluating design alternatives; and physical design of software
  user-interfaces, including windows, menus, and commands.}
{Prerequisite: CS 2110 with a grade of C- or higher.}

\course{CS 3240}{Advanced Software Development Techniques}{3
  credits}{Analyzes modern software engineering practice for
  multi-person projects; methods for requirements specification,
  design, implementation, verification, and maintenance of large
  software systems; advanced software development techniques and large
  project management approaches; project planning, scheduling,
  resource management, accounting, configuration control, and
  documentation.}{Prerequisite: CS 2150 with a grade of C- or higher.}

\course{CS 3330}{Computer Architecture}{3 credits}{Includes the
  organization and architecture of computer systems hardware;
  instruction set architectures; addressing modes; register transfer
  notation; processor design and computer arithmetic; memory systems;
  hardware implementations of virtual memory, and input/output control
  and devices.}{Prerequisite: CS 2110 with a grade of C- or higher. CS
  2330 recommended. Students may not receive credit for both CS 3330
  and ECE 3430.}

\course{CS 3501}{Special Topics in Computer Science}{1
  to 3 credits}{Content varies annually, depending on students needs
  and interests. Recent topics include the foundations of computation,
  artificial intelligence, database design, real-time systems,
  Internet engineering, wireless sensor networks, and electronic
  design automation.}{Prerequisite: Instructor permission.}

\subsection{4000 Level CS Courses}

\course{CS 4102}{Algorithms}{3 credits}{Introduces the analysis of
  algorithms and the effects of data structures on them. Algorithms
  selected from areas such as sorting, searching, shortest paths,
  greedy algorithms, backtracking, divide- and-conquer, and dynamic
  programming. Data structures include heaps and search, splay, and
  spanning trees. Analysis techniques include asymtotic worst case,
  expected time, amortized analysis, and reductions between problems.}
{Prerequisite: CS 2102 and 2150 with grades of C- or higher.}

\course{CS 4240}{Principles of Software Design}{3 credits}{Focuses on
  techniques for software design in the development of large and
  complex software systems. Topics will include software architecture,
  modeling (including UML), object-oriented design patterns, and
  processes for carrying out analysis and design. More advanced or
  recent developments may be included at the instructor's discretion.
  The course will balance an emphasis on design principles with an
  understanding of how to apply techniques and methods to create
  successful software systems.}{Prerequisite: CS 2150 with grade of C-
  or higher.}

\course{CS 4330}{Advanced Computer Architecture}{3 credits}{Provides
  an over\-view of modern microprocessor design. The topics covered in
  the course will include the design of super-scalar processors and
  their memory systems, and the fundamentals of multi-core processor
  design.}{Prerequisite: CS 2150 and CS 3330, both with grades of C-
  or higher.}

\course{CS 4414}{Operating Systems}{3 credits}{Analyzes process
  communication and synchronization; resource management; virtual
  memory management algorithms; file systems; and networking and
  distributed systems.}{Prerequisite: CS 2150 with grade of C- or
  higher, CS/ECE 2330 with a grade of C- or higher, and CS 3330 or ECE
  3430 with a grade of C- or higher.}

\course{CS 4434}{Dependable Computing Systems}{3 credits}{Focuses on
  the techniques for designing and analyzing dependable computer-based
  systems. Topics include fault models and effects, fault avoidance
  techniques, hardware redundancy, error detecting and correcting
  codes, time redundancy, software redundancy, combinatorial
  reliability modeling, Markov reliability modeling, availability
  modeling, maintainability, safety modeling, trade-off analysis,
  design for testability, and the testing of redundant digital
  systems.}

\course{CS 4444}{Introduction to Parallel Computing}{3
  credits}{Introduces the student to the basics of high-performance
  parallel computing and the national cyber-infrastructure. The course
  is targeted for both computer science students and students from
  other disciplines who want to learn how to significantly increase
  the performance of applications.}{Prerequisites: CS 2110 with grade
  of C- or higher, CS/ECE 2330 with a grade of C- or higher, CS3330 or
  ECE 3430 with a grade of C- or higher, APMA 3100 and APMA 3110.}

\course{CS/ECE 4457}{Computer Networks}{3 credits}{ Topics include the
  design of modern communication networks; point-to-point and
  broadcast network solutions; advanced issues such as Gigabit
  networks; ATM networks; and real-time communications.  Cross-listed
  as ECE 4457.}{Prerequisites: CS 2110 with grade of C- or higher, and
  CS 3330 or ECE 3430 with a grade of C- or higher.}

\course{CS 4458}{Internet Engineering}{3 credits}{An advanced course
  on computer networks on the technologies and protocols of the
  Internet. Topics include the design principles of the Internet
  protocols, including TCP/IP, the Domain Name System, routing
  protocols, and network management protocols. A set of laboratory
  exercises covers aspects of traffic engineering in a wide-area
  network.}{Prerequisite: CS 4457 with a grade of C- or better.}

\course{CS 4501}{Special Topics in Computer Science}{1
  to 3 credits}{Content varies annually, depending on students needs
  and interests. Recent topics include the foundations of computation,
  artificial intelligence, database design, real-time systems,
  Internet engineering, wireless sensor networks, and electronic
  design automation.}{Prerequisite: Instructor permission.}

\course{CS 4610}{Programming Languages}{3 credits}{Presents the
  fundamental concepts of programming language design and
  implementation. Emphasizes language paradigms and implementation
  issues. Develops working programs in languages representing
  different language paradigms. Many programs oriented toward language
  implementation issues.}{Prerequisite: CS 2150 with grade of C- or
  higher.}

\course{CS 4620}{Compilers}{3 credits}{Provides an introduction
  to the field of compilers, which translate programs written in
  high-level languages to a form that can be executed. The course
  covers the theories and mechanisms of compilation tools. Students
  will learn the core ideas behind compilation and how to use software
  tools such as lex/flex, yacc/bison to build a compiler for a
  non-trivial programming language.}{Prerequisite: CS2150 with grade
  of C- or higher. CS3330 recommended.}

\course{CS 4630}{Defense Against the Dark Arts}{3 credits}{Viruses,
  worms, and other malicious software are an ever-increasing threat to
  computer systems. There is an escalating battle between computer
  security specialists and the designers of malicious software. This
  course provides an essential understanding of the techniques used by
  both sides of the computer security battle.}{Prerequisite: CS 2150
  with a grade of C- or above.}

\course{CS 4710}{Artificial Intelligence}{3 credits}{Introduces
  artificial intelligence. Covers fundamental concepts and techniques
  and surveys selected application areas. Core material includes state
  space search, logic, and resolution theorem proving. Application
  areas may include expert systems, natural language understanding,
  planning, machine learning, or machine perception. Provides exposure
  to AI implementation methods, emphasizing programming in Common
  LISP.}{Prerequisite: CS 2150 with grade of C- or higher.}

\course{CS 4720}{Web and Mobile Systems}{3 credits}{With advances in
  the Internet and World Wide Web technologies, research on the
  design, implementation and management of web-based information
  systems has become increasingly important. In this course, we will
  look at the systematic and disciplined creation of web-based
  software systems. Students will be expected to work in teams on
  projects involving mobile devices and web
  applications.}{Prerequisite: CS 2150 with a grade of C- or higher.}

\course{CS 4730}{Computer Game Design}{3 credits}{This course will
  introduce students to the concepts and tools used in the development
  of modern 2-D and 3-D real-time interactive computer video games.
  Topics covered in this include graphics, parallel processing,
  human-computer interaction, networking, artificial intelligence, and
  software engineering.}{Prerequisite: CS 2150 with a grade of C- or
  higher.}

\course{CS 4740}{Cloud Computing}{3 credits}{Investigates the
  architectural foundations of the various cloud platforms, as well as
  examining both current cloud computing platforms and modern cloud
  research.  Student assignments utilize the major cloud
  platforms.}{Prerequisite: CS 2150 with a grade of C- or higher.}

\course{CS 4750}{Database Systems}{3 credits}{Introduces the
  fundamental concepts for design and development of database systems.
  Emphasizes relational data model and conceptual schema design using
  ER model, practical issues in commercial database systems, database
  design using functional dependencies, and other data models.
  Develops a working relational database for a realistic
  application.}{Prerequisite: CS 2150 with grades of C- or higher.}

\course{CS 4753}{Electronic Commerce Technologies}{3 credits}{History
  of Internet and electronic commerce on the web; case studies of
  success and failure; cryptographic techniques for privacy, security,
  and authentication; digital money; transaction processing; wired and
  wireless access technologies; Java; streaming multimedia; XML;
  Bluetooth. Defining, protecting, growing, and raising capital for an
  e-business.}{Prerequisite: CS 2150 with a grade of C- or higher.}

\course{CS 4810}{Introduction to Computer Graphics}{3
  credits}{Introduces the fundamentals of three-dimensional computer
  graphics: rendering, modeling, and animation. Students learn how to
  represent three-dimensional objects (modeling) and the movement of
  those objects over time (animation). Students learn and implement
  the standard rendering pipeline, defined as the stages of turning a
  three-dimensional model into a shaded, lit, texture-mapped
  two-dimensional image.}{Prerequisite: CS 2150 with a grade of
  C- or higher.}

%\course{CS 4820}{Real Time Rendering}{3 credits}{Examines real-time
%  rendering of high-quality interactive graphics. Studies the advances
%  in graphics hardware and algorithms that are allowing applications
%  such as video games, simulators, and virtual reality to become
%  capable of near cinematic-quality visuals at real-time rates. Topics
%  include non-photorealistic rendering, occlusion culling, level of
%  detail, terrain rendering, shadow generation, image-based rendering,
%  and physical simulation. Over several projects throughout the
%  semester students work in small teams to develop a small 3-D game
%  engine incorporating some state-of-the-art
%  techniques.}{Prerequisite: Grade of C- or better in CS 4810 or
%  equivalent working knowledge.}

%\course{CS 4830}{Computer Animation}{3 credits}{Introduces both
%  fundamental and advanced computer animation techniques. Discusses
%  such traditional animation topics as keyframing, procedural
%  algorithms, camera control, and scene composition. Also introduces
%  modern research techniques covering dynamic simulation, motion
%  capture, and feedback control algorithms. These topics help prepare
%  students for careers as technical directors in the computer
%  animation industry and assist in the pursuit of research
%  careers.}{Prerequisite: Grade of C- or better in CS 4810 or
%  equivalent working knowledge.}

\course{CS 4970}{Capstone Practicum I}{3 credits}{This course is one
  option in the CS Senior Thesis track. Under the Practicum track,
  students will take two 3-credit courses, CS 4970 and CS 4971. These
  courses would form a year-long group-based and project-based
  practicum class. There would be an actual customer, which could be
  either internal (the course instructor, other CS professors, etc.)
  or external (local companies, local non-profits,
  etc.).}{Prerequisite: CS 2150 with a grade of C- or higher.}

\course{CS 4971}{Capstone Practicum II}{3 credits}{This course is one
  option in the CS Senior Thesis track and is the continuation from CS
  4970. Under the Practicum track, students will take two 3-credit
  courses, CS 4970 and CS 4971. These courses would form a year-long
  group-based and project-based practicum class. There would be an
  actual customer, which could be either internal (the course
  instructor, other CS professors, etc.) or external (local companies,
  local non-profits, etc.).}{Prerequisite: CS 4970.}

\course{CS 4980}{Capstone Research}{1 to 3 credits}{This course is one
  option in the CS Senior Thesis track. Students will seek out a
  faculty member as an advisor, and do an independent project with
  said advisor. Instructors can give the 3 credits across multiple
  semesters, if desired. This course is designed for students who are
  doing research, and want to use that research for their senior
  thesis. Note that this track could also be an implementation
  project, including a group-based project.}{Prerequisite: CS 2150
  with a grade of C- or higher.}

\course{CS 4993}{Independent Study}{1 to 3 credits}{In-depth study of
  a computer science or computer engineering problem by an individual
  student in close consultation with departmental faculty. The study
  is often either a thorough analysis of an abstract computer science
  problem or the design, implementation, and analysis of a computer
  system (software or hardware).}{Prerequisite: Instructor
  permission.}

\course{CS 4998}{Distinguished BA Majors Research}{3 credits}{Required
  for Distinguished Majors completing the Bachelor of Arts degree in
  the College of Arts and Sciences. An introduction to computer
  science research and the writing of a Distinguished Majors
  thesis.}{Prerequisites: CS 2150 with a grade of C- or higher AND a
  CLAS student.}
 
\subsection{Selected ECE Courses}

This section is not meant to be an exhaustive list of all courses in
the Electrical Engineering department.  Instead, it is meant to list the
required courses for the Computer Engineering majors.  Information
about the other Electrical Engineering courses offered can be found
online\myurl{http://www.ece.virginia.edu/}.  Note that cross-listed
courses (CS/ECE 2330 (Digital Logic Design), CS/ECE 3330 (Computer
Architecture), and CS/ECE 4457 (Networks)) are only listed above.

\vspace{0.25in}

\course{ECE 2630}{Introductory Circuit Analysis}{3 credits}{Elementary
  electrical circuit concepts and their application to linear circuits
  with passive elements; use of Kirchhoff's voltage and current laws
  to derive circuit equations; solution methods for first- and
  second-order transient and DC steady-state responses; AC
  steady-state analysis; frequency domain representation of signals;
  trigonometric and complex Fourier series; phasor methods; complex
  impedance; transfer functions and resonance; Thevenin / Norton
  equivalent models; controlled sources. Six laboratory
  assignments.}{Prerequisite: APMA 1110.}

\course{ECE 2660}{Electronics I}{3 credits}{Studies the modeling,
  analysis, design, computer simulation, and measurement of electrical
  circuits which contain non-linear devices such as junction diodes,
  bipolar junction transistors, and field effect transistors. Includes
  the gain and frequency response of linear amplifiers, power
  supplies, and other practical electronic circuits. Three lecture and
  three laboratory hours.}{Prerequisite: ECE 2630.}


\course{ECE 3430}{Introduction to Embedded Computing Systems}{3
  credits}{ An embedded computer is designed to efficiently and
  (semi-) au\-ton\-o\-mous\-ly perform a small number of tasks, interacting
  directly with its physical environment. This lab-based course
  explores architecture and interface issues relating to the design,
  evaluation and implementation of embedded systems . Topics include
  hardware and software organization, power management, digital and
  analog I/O devices, memory systems, timing and
  interrupts.}{Prerequisite: CS/ECE 2330, CS 2110, ECE 2660--if ECE
  3430 offered in spring}

\course{ECE 3750}{Signals and Systems I}{3 credits}{Develops tools for
  analyzing signals and systems operating in continuous-time, with
  applications to control, communications, and signal processing.
  Primary concepts are representation of signals, linear
  time-invariant systems, Fourier analysis of signals, frequency
  response, and frequency-domain input/output analysis, the Laplace
  transform, and linear feedback principles. Practical examples are
  employed throughout, and regular usage of computer tools (Matlab,
  CC) is incorporated. Students cannot receive credit for both this
  course and BIOM 3310.}{Prerequisite: ECE 2630 and APMA 2130.}

\course{ECE 4435}{Computer Architecture \& Design}{3 credits}{
  Introduces computer architecture and provides a foundation for the
  design of complex synchronous digital devices, focusing on: 1)
  Established approaches of computer architecture, 2) Techniques for
  managing complexity at the register transfer level, and 3) Tools for
  digital hardware description, simulation, and synthesis. Includes
  laboratory exercises and significant design activities using a
  hardware description language and simulation.}{Prerequisite: ECE
  3430}

\course{ECE 4440}{Embedded System Design}{3 credits}{ Modeling,
  analysis and design of embedded computer systems. Tradeoff analysis
  and constraint satisfaction facilitated by the use of appropriate
  analysis models. Includes a semester-long design of an embedded
  system to meet a specific need. Counts as MDE (major design
  experience) for both electrical and computer engineering
  students.}{}


\mysection{Course Numbering}
\label{course-numbering}

Starting with the fall 2009 semester, the University of Virginia
changed all course numbers to 4-digit numbers from the old 3-digit
number system. Earlier versions of this handbook listed both the both
the 3-digit number and the 4-digit number, in the form of ``CS 1110
(101)'' to aid the transition, as well as a full table mapping the
3-digit course numbers to the 4-digit course numbers.  The current
version no longer lists the courses that way.  This handbook no longer
lists the old 3-digit course numbers, but they can be found
online\myurl{http://www.virginia.edu/registrar/atoz.html\#CS}.

The new 4-digit course numbers follow a system developed by the
department. The first digit is the year that the course is expected to
be taken. The second digit specifies the type of course, as shown
below. The third and fourth digits attempted to keep the previous last
two digits of the 3-digit course number, although that was not always
possible.


The 2nd digit numbering scheme is:

\begin{itemlist}
\item x000: service courses, courses for non-majors, general interest
\item x100: core, fundamentals, theoretical (a broad category)
\item x200: software development-oriented courses (note in ECE, this will
 be for electronics courses)
\item x300: hardware, architecture, etc.
\item x400: computer systems
\item x500: by University rule: ``special-topics and variable one-time
 offerings''
\item x600: languages, compilation, etc.
\item x700: application areas including AI, databases, etc.
\item x800: computer graphics
\item x900: by University rule: thesis, dissertation, independent
 study, capstone, etc.
\end{itemlist}

Note that currently cross-listed courses with ECE fall in the x300 and
x400 categories.

\mysection{Degree Requirement Revisions}
\label{sec:degreerevisions}

Computer science is an evolving field, and our undergraduate
curriculum reflects this. The department sometimes makes changes to
the requirements for the bachelor's degree. Note that you are allowed
to graduate using ANY SINGLE set of requirements that were in effect
when you were a declared computer science major~-- thus, if the
requirements change, you are allowed to complete the degree using the
older version of the requirements. You cannot ``mix and match''
requirements from the different sets. Whatever set of requirements is
completed, it must be {\em all} the requirements from that set.

Any changes to the requirements will typically occur after the spring
semester and before the following fall semester, unless the change is
considered minor. A minor change is something that does not in any way
restrict the degree requirements. Examples of minor changes would be
expanding the allowed courses for one of the elective types, or
clarifying what counts as a given elective. Note that unless the
change to the requirements directly affects the third semester (i.e.
the first semester of the second year), a student cannot choose to
graduate using a set of requirements that were in effect during his or
her first year at UVa but that were not in effect during his or her
second year, as they were not a declared computer science major during
their first year.

The requirement revisions below describe which major changes occurred
during the previous years, and what courses students must complete to
graduate using that set of requirements. Note that the older sets are
kept for historical reasons, even though there may not be any more
students who are eligible to graduate with those sets.

The current set of requirements, which this document reflects, became
effective in the spring of 2013 (and before then-first year SEAS
majors declared the BS CS and CpE majors).

\subsection{Requirements revision from spring 2013}

The inclusion BS CS capstone requirement added 3 additional credits
(CS 4971 (Capstone Practicum II) or CS 4980 (Research Capstone))
needed to complete the BS CS degree.  In addition, CS 4970 (Capstone
Practicum I) was added as a CS elective.  Details about the capstone
can be found in section~\ref{capstone-section}
(page~\pageref{capstone-section}).

For the BS CpE, the embedded systems requirement
(section~\ref{embedded}, page \pageref{embedded}) was added, as that
was approved in the spring of 2013.

CS 2330 (Digital Logic Design) was removed as a pre-requisite for
CS/ ECE 3330 (Computer Architecture), and added as a pre-requisite for
CS 4330 (Advanced Computer Architecture) and CS 4434 (Dependable
Computing).  Note that because of the Embedded Systems requirement
change for CpE, BS CpE majors no longer will be taking CS/ECE 3330.
This requirement change also affects the BA CS, as they no longer need
to take CS 2330 as a CS elective in order to enroll in CS/ECE 3330.

UVa moved to 4-digit course numbers in the fall of 2009, and previous
versions of this handbook listed both the 3-digit and 4-digit numbers
for the courses.  The current version no longer lists the 3-digit
versions throughout the document.  However, the course numbering
section (section~\ref{course-numbering},
page~\pageref{course-numbering}) still lists the mapping, for
reference.  Anybody interested in the original 3-digit course numbers
can find the mapping
online\myurl{http://www.virginia.edu/registrar/atoz.html\#CS}.

\subsection{Requirements revision from spring 2010}

In January of 2010, the elective structure was changed. Previously,
majors were required to take 3 HSS electives, 3 general education
electives, 3 technical electives, and 1 unrestricted elective. With
the change, these 10 elective courses are now split into 5 HSS
electives and 5 unrestricted electives. Students wishing to graduate
using the old rules (if you were a declared major prior to 2010)
should see the previous editions of this handbook for the description
of what constitutes general education electives and technical
electives. However, the new requirements are more general, and we
expect most students to graduate using these updated requirements. The
old versions of this handbook are available from the department.

\subsection{Requirements revision from fall 2009}

In addition to the course numbering change, the change in the
requirements was that the computer architecture elective was replaced
with an additional CS elective, to bring the total number of required
CS electives to 5. The previous computer architecture requirement had
the students take one class from a set of 3: CS 4444
(Introduction to Parallel Computing), CS 4330 (Advanced Computer
Architecture) or ECE 4435 (Computer Architecture and
Design). Since all of those three courses count as CS electives,
students who have already fulfilled this old requirement will still
fulfill the CS elective that replaced it.

Focal paths were also added to the undergraduate handbook, although
they do not change the major requirements.

\cleardoublepage
\pagestyle{empty}

\vspace*{2in}

\begin{center}
\parbox{2.5in}{{\em Enlighten the people generally, and tyranny and
    oppression of body and mind will vanish like evil spirits at the
    dawn of day … the diffusion of knowledge among the people is to be
    the instrument by which it is to be effected.\linebreak\linebreak
    --~Thomas Jefferson, 1816}}
\end{center}


\clearpage

\vspace*{2.75in}

\begin{center}
Department {\em of} Computer Science \linebreak
School of Engineering and Applied Science \linebreak
The University {\em of} Virginia \linebreak
85 Engineer's Way \linebreak
P.O. Box 400740 \linebreak
Charlottesville, Virginia 22904-4740 \linebreak
434.982.2200 \linebreak
\url{http://www.cs.virginia.edu} \linebreak
\end{center}

\begin{figure}[h!]
\begin{center}
\epsfig{figure=images/uva-seal.png,width=2in}
\end{center}
\end{figure}

\end{document}
